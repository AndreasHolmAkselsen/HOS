%\documentclass[a4paper,11pt]{texMemo}
\documentclass[internal]{sintefmemo}
% Alternative Options:
%	Paper Size: a4paper / a5paper / b5paper / letterpaper / legalpaper / executivepaper
% Duplex: oneside / twoside
% Base Font Size: 10pt / 11pt / 12pt


%% Language %%%%%%%%%%%%%%%%%%%%%%%%%%%%%%%%%%%%%%%%%%%%%%%%%
\usepackage[english]{babel} %francais, polish, spanish, ...
%\usepackage[T1]{fontenc} %[T1]
%\usepackage[utf8]{inputenc} % [utf8], [ansinew] for non
%\usepackage{enumerate}
\usepackage{amsmath,amsthm,amsfonts,amssymb}

\usepackage[shortlabels]{enumitem}
\usepackage[round]{natbib}
\usepackage{bm}

%% Packages for Graphics & Figures %%%%%%%%%%%%%%%%%%%%%%%%%%
\usepackage{graphicx} %%For loading graphic files
%\usepackage[center,tight]{subfigure} %%Subfigures inside a figure
\usepackage{subfig} %%Subfigures inside a figure
\usepackage{xcolor}
\usepackage{hyperref}
\hypersetup{colorlinks=true,
			linkcolor=blue,
			filecolor=blue,
			urlcolor=blue,
			citecolor=blue}
%\hypersetup{colorlinks=false}

%\usepackage{alltt} % writing computer code
%\usepackage{matlab} % writing matlab code specifically
%\usepackage[straightquotes]{newtxtt} % fixes the quotation marks (') such that they are straight in the code envirorment.
\usepackage{float}

%% Math Packages %%%%%%%%%%%%%%%%%%%%%%%%%%%%%%%%%%%%%%%%%%%%

%\usepackage[margin=1.5in]{geometry}

%\setcounter{secnumdepth}{1}
%\usepackage{chngcntr}

\usepackage{caption}

\counterwithout{figure}{section}
\counterwithout{table}{section}


\newcommand{\mr}{\mathrm}
\newcommand{\mc}{\mathcal}
\newcommand{\dd}{\mr d}
\newcommand{\br}[3]{\left#1#2\right#3}
\newcommand{\Prob}{\mr{Prob}}
\newcommand{\comm}[1]{{\color{red}#1}}
\newcommand{\ii}{\mr i}
\newcommand{\ee}{\mr e}


%\memoto{Whom it may concern\ldots}
%\memofrom{Andreas H. Akselsen}
%\memosubject{Investigation of directional spreading in wave tank using the the Minimum Entropy Method (MEM) --- Irregular waves.}
%\memodate{\today}
%%\logo{\includegraphics[width=0.3\textwidth]{wp}}
%%Evaluation of the Maximum Entropy Method (MEM) when applied to three- and fifteen-probe sensor arrays.




\title{Wave field disturbances expected from a steep or discontinuous ramp within the wave tank}
\author{Andreas H. Akselsen}
\project{302006355-3 (OSC pre-project)}
\year{2021}

\recipient[information]{SINTEF employees}
%\recipient[information,agreed]{SVV}
%\recipient[comments]{\LaTeX\ hackers}
%\recipient[attention,agreed]{\LaTeX\ users}
%\recipient[attention]{Word users}
%\address{Postboks 4762 Torgarden NO-7465 Trondheim, NORWAY}
%\institute{SINTEF Ocean AS}
%\vat{}

\begin{document}
\frontmatter

\section{Abstract}
This memo applies the theory of the twin papers \citet{li_2021_step1} and \citet{li_2021_step2} to dimensions relevant for the SINTEF Ocean Basin with the aim of estimating the impact of abrupt changes in water depth on wave quality.
The two papers are summarised in what follows.
The study identifies a wide range of operation where measurements are likely to be affected by parasitic waves generated by abrupt depth change, both from a ramp in front of the wavemaker and from a partly steep beach.

\section{Introduction}

The two papers \citet{li_2021_step1} and \citet{li_2021_step2} concern a model derived for wave packets travelling over a bathymetry step discontinuity  as illustrated in figure \ref{fig:sketch_Yan}.
The model solves the problem to second order in Stokes expansion making no critical assumption on water depth or on the narrowness of the wave packet band.
The water is assumed deep enough for the Stokes expansion to hold.
\\

Several new wave groups are created when a wave packet crosses a step discontinuity. 
At first order of accuracy, these are
\begin{itemize}
	\item a transmitted free wave packet, travelling with the group velocity of the downstream ($x>0$) depth.
	\item a reflected free wave packet, travelling with the group velocity of the upstream ($x<0$) depth.
\end{itemize}
At second order, the following groups are identified:
\begin{itemize}
	\item Two bound wave groups, corresponding to the Stokes solution of the transmitted and reflected first order free waves.
	\item Two spurious superharmonic free wave packets travelling in each direction. These are double the frequency of the original incoming wave packet any they propagate with the the corresponding free wave group velocity computed at their respective depths. These therefore travel more slowly than their bound wave counterparts.
	\item Two spurious subharmonic free wave packets travelling in each direction. These travel with approximately the deep water group velocity $\pm\sqrt{gh}$ without dispersing, sign and depth being side-dependent. Accordingly, these waves travel faster than the other wave groups.
\end{itemize}
An illustration, taken from an example in \citet{li_2021_step1}, is shown in \autoref{fig:groupSplit_Yan}.


\begin{figure}[h!ptb]%
\centering
\includegraphics[width=.75\columnwidth]{./figures/sketch_Yan.png}%
\caption{Sketch of problem solved to second order in \citet{li_2021_step1}.}%
\label{fig:sketch_Yan}%
\end{figure}
 
\begin{figure}[h!tbp]%
\centering
\includegraphics[width=\columnwidth]{./figures/groupSplit_Yan.png}%
\caption{Figures form \citet{li_2021_step1}---A wave packet travelling over a step discontinuity at $x=0$ is split into three groups propagating with different group velocities.
The figures show the second-order components of surface elevation.
Visible are
a bound transmitted group that travels together with the first-order group (not shown) in the transmitted direction, as wall as
two spurious free wave group travelling in opposite direction with double carrier frequency. }%
\label{fig:groupSplit_Yan}%
\end{figure}






\section{Quantifying the magnitude of spurious waves in relation to wave tank design}
\label{sec:quantifying}
We will here attempt to give a relative measure of the spurious waves that can be expected from abrupt changes in wave tank bathymetry.
Figure~5 in \citet{li_2021_step1}, showing contour plots of solution transfer functions, are used as a basis for this inquiry. 

The two wave groups which are of most interest to us in this respect are the second-order transmitted free superharmonic spurious waves and the first-order reflected waves.
Spurious transmitted waves disturb the wave field downstream of an abrupt depth change. Such a transition will be present when an adjustable tank bed ramp is located in front of the wavemaker. 
Reflection from an upstream ramp can also agitate the downstream wave field after first being re-reflected off of the wavemaker. 
Reflected first order wave magnitudes can give an indication of disturbances arising from short beaches that do not extend all the way to the tank bed.

Spurious subharmonic wave groups can also disturb experiments. Although these waves are small in magnitude, they can influence the measured slow-drift responses.
The transmitted non-spurious free+bound waves we wish to preserve, and it is relative to these that we  compare spurious wave magnitudes.

\subsection{The transmitted superharmonic free wave packet}
Let us briefly quote equations $(2.25d)$ and $(2.25e)$ in \citet{li_2021_step1} for bound and free second order transmitted waves:
\begin{subequations}
\begin{align}
\Phi_{T,b}^{(22,0)} &= \frac{3\omega_0}{8} |T_0|^2A_I^2\bigg(\frac{c_{g0}}{c_{g0s}}X -c_{g0}T\bigg)
\frac{\cosh 2 k_{0s}(z+h_s)}{\sinh^4k_{0s}h_s} \sin(2k_{0s}z - 2\omega_0 t +2\mu_0+2\mu_T)
\\
\Phi_{T,f}^{(22,0)} &= \omega_0 |T_{20}|A_I^2\bigg(\frac{c_{g0}}{c_{g20s}}X -c_{g0}T\bigg)
\frac{\cosh k_{20s}(z+h_s)}{\cosh k_{20s}h_s} \sin(k_{20s}z - 2\omega_0 t +2\mu_0+2\mu_{20s})
\end{align}%
\label{eq:Phi}%
\end{subequations}%
Suffixes `T' `b', `f' and `s' here respectively indicate `transmitted', `bound', `free' and `shallow (side)'. $c_{g\dots}$ are group velocities, $X$ and $T$ are slow-scale space and time coordinates and $A_I()$ is the amplitude envelope of the wave packet. Superscript `$(22,0)$' indicates a second order quantity of frequency $2\omega_0$ to zeroth order in wave packet band width.
The transfer functions $T_0$ and $T_{20}$ are provided as contour plots in the same paper.
A practical measure of relative magnitude is the ratio of amplitudes between second order bound and free waves as it is independent of carrier amplitude. 
It is however not appropriate to compare the ratio of potential functions $\Phi$ since $\Phi_{T,b}^{(22,0)}$ vanish at large depths while $\Phi_{T,f}^{(22,0)}$ does not. 
We evaluate instead the ratio of surface elevations $\zeta_{T,f}^{(22,0)}$ and $\zeta_{T,b}^{(22,0)}$. % even though expressions for these are not provided in the paper.
%We can however make an educated guesses as to what they look like;
The bound transmitted wave is essentially the second order component of the Stokes wave solution\footnote{\url{https://en.wikipedia.org/wiki/Stokes_wave}} with amplitude $|T_0|$ times the amplitude envelope. 
The free wave follows the solution of the liner components such that $\zeta_t = \Phi_z$. Accordingly, we define
%\begin{subequations}
\begin{align}
\big|\zeta_{0T,b}^{(22,0)}\big| &= k_{0s}\frac{3-\tanh^2k_{0s}h_s}{4\tanh^3k_{0s}h_s} |T_0|^2 A_0^2 ,
&
\big|\zeta_{0T,f}^{(22,0)}\big| &= \frac{2\omega_0}{g}|T_{20}| A_0^2.
\end{align}%
\label{eq:zeta}%
%\end{subequations}%
Yan Li (first author of cited papers) has through private communication provided us with code that allows us to compute the transfer functions $T_0$, $R_0$ and $T_{20}$.
In \autoref{fig:T_contour}, we compare the output of this code with figure 5 in \citet{li_2021_step1}  as verification.
We have in the case of $T_{20}$ removed contours of magnitude $T_{20}>25$ to highlight that $T_{20}$ does indeed take on values larger than 20, even though 20 was made the upper contour in the paper for the purpose of visibility. 
We also see that  $T_{20}$ starts increasing again for larger values of $k_0 h_d$, possibly due to numerical inaccuracies in the computation;
we know intuitively that $T_{20}$ should vanish in deep waters.
%We do not have direct access to the data for transfer functions $T_0$ and $T_{20}$ and computing these appears to be involved. As a coarse estimate, we instead sample a number from figure 5 in \citet{li_2021_step1} and interpolate our own contours. These are shown in \autoref{fig:T_contour}.

The resulting relative magnitudes $\big|\zeta_{0T,f}^{(22,0)}\big|\big/\big|\zeta_{0T,b}^{(22,0)}\big|$ are shown in \autoref{fig:zeta_f__b}.
The ratio drops off for short periods as expected;
the wavelength associated with period $T=2.5$ seconds is approximately $10$ meters, which means that the wave packet will not `feel' anything deeper than about $5$ meters.
We have excluded a region corresponding to the aforementioned region where $T_{20}$ increases as the results in that region are questionable.
%There are two peaks in the contour plot where similar magnitudes are of similar magnitude.
%One occurs at intermediate relative depth with a large step, another in shallow waters for a small step. 
We should keep in mind that the theory, which is based on a Stokes expansion, is inaccurate in shallow water (large periods).

\begin{figure}[H]%
\centering
\subfloat[Figures 5c and 5f in \citet{li_2021_step1}]{
%\includegraphics[width=.4\columnwidth]{./figures/fig_5c.png}%
%\hspace{.1\columnwidth}
%\includegraphics[width=.4\columnwidth]{./figures/fig_5f.png}%
\includegraphics[width=.33\columnwidth]{./figures/fig_5a.png}%
\includegraphics[width=.33\columnwidth]{./figures/fig_5c.png}%
\includegraphics[width=.33\columnwidth]{./figures/fig_5f.png}%
}\\
\subfloat[Coarse representation by sampling a number of points \textit{by eye}.]{
%\includegraphics[width=\columnwidth]{./figures/T_contour.pdf}%
\includegraphics[width=\columnwidth]{./figures/T_contour_Yan.pdf}%
}
\caption{Transfer functions $T_0$ and $T_{20}$.}%
\label{fig:T_contour}%
\end{figure}


\begin{figure}[H]%
\center
%\includegraphics[width=.5\columnwidth]{./figures/zeta_f__b.pdf}%
\includegraphics[width=.5\columnwidth]{./figures/zeta_f__b_Yan_hd6.pdf}%
\includegraphics[width=.5\columnwidth]{./figures/zeta_f__b_Yan_hd10.pdf}%
\caption{Contour plot of the ratio $\big|\zeta_{0T,f}^{(22,0)}\big|\big/\big|\zeta_{0T,b}^{(22,0)}\big|$ as defined in \eqref{eq:zeta} as function of dimensional period and relative downstream depth.
The upstream depth is fixed at $h_d = 6$\,m and $h_d = 10$\,m}%
\label{fig:zeta_f__b}%
\end{figure}


\subsection{The reflected first-order wave packet}
\autoref{fig:zeta_R} shows a similar contour plot for the ratio $\big|\zeta_{0R}^{(11,0)}\big|\big/\big|\zeta_{0T}^{(11,0)}\big| = |R_{0}|/|T_0|$.
This ratio increases monotonically with decreasing downstream depth $h_s$ and should approach infinity as $h_s\to 0$.

\begin{figure}[h!pbt]%
\center
%\includegraphics[width=.5\columnwidth]{./figures/zeta_R.pdf}%
\includegraphics[width=.5\columnwidth]{./figures/zeta_R_Yan_hd6.pdf}%
\includegraphics[width=.5\columnwidth]{./figures/zeta_R_Yan_hd10.pdf}%
\caption{Contour plot of the ratio $\big|\zeta_{0R}^{(11,0)}\big|\big/\big|\zeta_{0T}^{(11,0)}\big| = |R_{0}|/|T_0|$ as function of dimensional period and relative downstream depth.
The upstream depth is fixed at $h_d = 6$\,m and $h_d = 10$\,m.}%
\label{fig:zeta_R}%
\end{figure}



\subsection{The subharmonic free wave packets}
%\includegraphics[width = 3cm]{./figures/Pikachu.jpg}

A shallow-water assumption is adopted for subharmonics in \citet{li_2021_step1} which lead to a solution on the form
\begin{align}
\zeta_{R,b}^{(20,1)} &= k_0 B_d |R_0|^2A_I^2\Bigg(-X-c_{g0}T\Bigg),
&
\zeta_{T,b}^{(20,1)} &= k_{0s} B_s |T_0|A_I^2\Bigg(\frac{c_{g0}}{c_{g0s}}X-c_{g0}T\Bigg),
\\
\zeta_{R,f}^{(20,1)} &= k_0 B_R^f A_I^2\Bigg(-\frac{c_{g0}}{\sqrt{gh_d}}X-c_{g0}T\Bigg),
&
\zeta_{T,f}^{(20,1)} &= k_{0s} B_T^f A_I^2\Bigg(\frac{c_{g0}}{\sqrt{gh_s}}X-c_{g0}T\Bigg).
\end{align}
with all transfer functions being relatively simple algebraic expressions.

We compare in \autoref{fig:zeta201} the reflected and transmitted free spurious subharmonics to their bound subharmonic counterparts. 
The transmitted ratio $|\zeta_{0T,f}^{(20,1)}|/|\zeta_{0T,b}^{(20,1)}| = B_T/B_s$ is of order unity and increases steadily with increasing step height (Figure~\ref{fig:zeta201:T}).
The reflected free subharmonic (Figure~\ref{fig:zeta201:R}) is of magnitude comparable to the reflected bound subharmonic only for large step heights and moderate periods.

\begin{figure}[h!ptb]%
\centering
\subfloat[Transmitted]{\includegraphics[width=.5\columnwidth]{./figures/zetaT201.pdf}\label{fig:zeta201:T}}%
\subfloat[Reflected]{\includegraphics[width=.5\columnwidth]{./figures/zetaR201.pdf}\label{fig:zeta201:R}}%
\caption{Free relative to bound subharmonc as function of dimensional period and relative downstream depth. The upstream depth is fixed at $h_d = 10$\,m.}%
\label{fig:zeta201}%
\end{figure}


\section{Discontinuous vs.\ sloping depth transitions}
How does a sloping depth transition compare to a discontinuous step?
According to \citet{li_2021_step2}, the two are quite similar.
Experiments are presented in \citet{li_2021_step2} where two gradual depth transitions,1:1 ($45$\textdegree) and 1:3 ($\sim$18.4\textdegree), were tested together with the step discontinuity. 
A cut-out from a figure in \citet{li_2021_step2} showing amplitudes of bound and free waves is given in \autoref{fig:slopingAmplitudes}. 
The difference between a step transition and a 1:3 slop transition does not amount to more than about 15\%.
Further details and examples are given in that paper.
We note that the dimensional wavelengths in these experiments (campaign A,C,D) ranges from $1.68--3.35$\,m on the deep side and $1.28--2.18$\,m on the shallow side. 
For comparison, the 

relative depths in the presented experiments ranges from 

\begin{figure}[H]%
\center
\fbox{\includegraphics[width=.9\columnwidth]{./figures/parameters_Yan.png}}
\\\fbox{\includegraphics[width=.9\columnwidth]{./figures/amplitudesSloping_Yan.png}}%
\caption{Cut-out from \citet{li_2021_step2}.}%
\label{fig:slopingAmplitudes}%
\end{figure}


\section{Other studies}
A wide overview of experimental and numerical studies on depth transitions is given in the introduction fo \citet{li_2021_step2} and references within.
We will here consider some of their findings, although we do not provide a complete overview.


\subsection{\citet{trulsen_2020_rampKurtosis}}
A short and informative paper has been provided by \citet{trulsen_2020_rampKurtosis}.
\autoref{fig:Trulsen_sketch} shows a sketch of Trulsen et.\ al.'s experimental setup. 
The shoal of this setup is $42$\,cm heigh, giving a fixed slope $\approx 0.26$.
The horizontal lengths of ramp and shoal are equal.
This setup is similar to the planned design of the towing tank (\autoref{fig:plan_slepetanken}),
for which the slope approaches $0.2$ for the shallowest depths and the shoal is twice the horizontal length of the ramp.

\citet{trulsen_2020_rampKurtosis} measured skewness and kurtosis in a JONSWAP spectrum at multiple spatial locations.
Their measurements show a clear spatial dependency of these properties along the length of their shoal---a phenomenon which they attribute to a relaxation process from the depth transition. 
We have in \autoref{fig:Trulsen_kurtosis} copied some of these results for kurtosis to illustrate. 
Kurtosis statistics does not return to spatial steadiness within the span of the shoal, and it is not clear that it would within twice that shoal length (as in the towing tank design plan).
It is well understood, or at least strongly believed within the scientific community, that extreme statistics are relatable to kurtosis \citep[e.g.][]{mori_2006_kurtosis_and_BFI}.
It therefore seems plausible, if not likely, that a ramped depth transition will affect observed extreme statistics far within the shoaling region.
The shoaling region extends about 1.5--3 characteristic wavelengths in the results shown. 

We also remark on the paper \citet{trulsen_2012_weakSlope}, where a significantly weaker slope, 1:30, was tested. Here the investigators again observed increased skewness and kurtosis close to the shallower side of the slope and found experimental evidence of a large probably wave envelope at the same location.


\begin{figure}[h!ptb]%
\includegraphics[width=\columnwidth]{./figures/Trulsen_sketch.png}%
\caption{Experimental setup, copied from \citet{trulsen_2020_rampKurtosis}.
The shoal is $42$\,cm heigh, giving a fixed slope $\approx 0.26$ on each side.
Water depth was varied.
The horizontal lengths of ramp and shoal are equal.}%
\label{fig:Trulsen_sketch}%
\end{figure}

\begin{figure}[h!ptb]%
\includegraphics[width=\columnwidth]{./figures/plan_slepetanken.png}%
\caption{Design sketch for the towing tank.}%
\label{fig:plan_slepetanken}%
\end{figure}

\begin{figure}[h!ptb]%
\centering
\includegraphics[width=.65\columnwidth]{./figures/Trulsen_table.png}\\
\includegraphics[width=.65\columnwidth]{./figures/Trulsen_kurtosis.png}%
\caption{Kurtosis results, copied from \citet{trulsen_2020_rampKurtosis}.
Compare to \autoref{fig:Trulsen_sketch} 
Figures showing similar trends for skewness are given in the cited reference.}%
\label{fig:Trulsen_kurtosis}%
\end{figure}




\subsection{\citet{viotti_2014_slope}}
A similar study using a fully nonlinear numerical solver was conducted by \citet{viotti_2014_slope}.
A sketch of their domain is shown in \autoref{fig:Viotti_sketch}.
A smooth ramp was used in their study, such that their bathymetry is described by
\[
b(x) = -H_1 + \tfrac12(H_1-H_1)\bigg[\tanh \frac{x-x_1}{\delta_1}-\tanh\frac{x-x_2}{\delta_2}\bigg].
\]
Used values are $H_1=1.8$, $H_2\in\{0.8,0.7,0.6,0.55\}$, $\delta_1 = 2$, $x_2-x_1=200$. 
The steepest part of this slope is at $x=x_1$ where the gradient is $(H_1-H_1)/\delta_1$, although replaced with a straight ramp the gradient would be about a quarter of this; 
\citet{li_2021_step2} cites the equivalent slope of being 0.16--0.20.
Relative depths are $k_p H_1\approx1.8$ on the deep side and $k_p H_1\approx1.03,0.92,0.81,0.78$ on the shallow side. 
The wave spectrum generated form the wavemaker is Gaussian with peak frequency 1.0 and bandwidth 0.2. Average deep-side steepness is $\approx 0.068$.


\begin{figure}[h!pbt]%
\centering
\includegraphics[width=.75\columnwidth]{./figures/Viotti_sketch.png}%
\caption{Sketch of numerical setup used in \citet{viotti_2014_slope}.}%
\label{fig:Viotti_sketch}%
\end{figure}

\autoref{fig:viotti_results} displays \citeauthor{viotti_2014_slope}' results taken from their paper.
We note that the shoal length shown in the plots corresponds to about 25 wavelengths.
As seen in \citet{trulsen_2020_rampKurtosis}, there is a relaxation time corresponding to a couple of wavelengths before an equilibrium state is reached atop the shoal. 
The spectral properties of the shallower water state differs from those of the deep water side as bound harmonics become more apparent in shallow waters. 
As demonstrated in \citet{li_2021_step1,li_2021_step2}, this change of state is also associated with a discharge of energy in the form of spurious waves. 


\begin{figure}[h!ptb]%
\centering
%\subfloat[Characteristic local frequency and wavenumber for the four shoal heights $H_2\in\{0.8,0.7,0.6,0.55\}$ where decreasing heights corresponds to increasing peaks. ]{%
%%
%}
\subfloat[Characteristic local frequency, wavenumber, skewness and excess of kurtosis for the four shoal heights.]{% $H_2\in\{0.8,0.7,0.6,0.55\}$ where decreasing heights corresponds to increasing peaks. ]{%
\includegraphics[width=.5\columnwidth]{./figures/Viotti_k.png}\includegraphics[width=.5\columnwidth]{./figures/Viotti_kurtosis.png}%
}\\
\subfloat[Normalized probability density functions at three locations indicated in leftmost panel for the four different shoal heights.]{%
\includegraphics[width=\columnwidth]{./figures/Viotti_S.png}%
}\\
\subfloat[Local power spectra with varying location and shoal height. Each plot is up-shifted by $10^4$ relative to the previous plot.]{%
\includegraphics[width=\columnwidth]{./figures/Viotti_S_x.png}%
}
\caption{Results taken from \citet{viotti_2014_slope}.
The shoal length shown in the plots corresponds to about 25 wavelengths.}%
\label{fig:viotti_results}%
\end{figure}

Similar, more extensive fully-nonliear simulation studies are provided by \citet{Zheng_2020_slope} with similar conclusions. We do not display these results here.

\section{Conclusions}
Steep and moderately inclined ramps within the wave tank will for a large range of depths and wave periods generate additional free spurious (parasitic) waves of magnitudes similar to the bound harmonics. (See \autoref{fig:zeta_f__b} and \ref{fig:zeta201:T}.)  
The presence of these waves should be evident when waves are steep.
Likewise, free spurious wave will be generated and reflected off of beaches unless these are made sufficiently long and shallow. 
Beaches that do not reach all the way to bottom level generate for long periods wave reflection at linear order (\autoref{fig:zeta_R}). Such reflection can be significant also for non-steep waves.


In terms of bathymetry effects in irregular wave fields, studies show that depth transitions for not-very-shallow slopes will be associated with a concentration of krutosis and skewness near the shoal edge. This will in other words affect the wave statistics observed in the upstream end of the shoal. Wave statistics return to being spatially uniform some relaxation length downstream of the shoal edge. This relaxation length appears to be a couple of wavelengths, for which the proposed bathymetry design appears insufficiently long (see \autoref{fig:fiugre_Nwavelengths_at_various_depths} below.)


\begin{figure}[H]%
\centering
\includegraphics[width=.5\columnwidth]{./figures/fiugre_Nwavelengths_at_various_depths.pdf}%
\caption{Periods corresponding to a fixed number of linear wavelengths spanning the suggested shoal length of 50\,m. Various shoal depths shown. States in the deep water regime are not relevant ($\tanh\pi/2 = 0.917$, $\tanh\pi =0.996$).}%
\label{fig:fiugre_Nwavelengths_at_various_depths}%
\end{figure}

\bibliographystyle{plainnat} % abbrvnat,plainnat,unsrtnat
\bibliography{sintef_bib} %You need a file 'literature.bib' for this.

\end{document}