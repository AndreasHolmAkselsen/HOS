\documentclass[a4paper,12pt]{article}
%\documentclass{amsart}

\usepackage[a4paper, total={17cm, 25cm}]{geometry}
\usepackage[utf8]{inputenc}
\usepackage[english]{babel}
\usepackage{amsmath,bm}
\usepackage{xcolor}
\usepackage{amsmath}%
\usepackage{amsfonts}%
\usepackage{amssymb}%
\usepackage{graphicx}
\usepackage{natbib}
\usepackage{mathtools}
\newcommand{\mr}{\mathrm}
\renewcommand{\S}{^\mr{S}}
\newcommand{\ii}{\mr{i}}
\newcommand{\ee}{\mr{e}}
%\newcommand{\phit}{\psi}
\newcommand{\phit}{\tilde\phi}

\begin{document}
\title{The Higher Order Spectral method extended with a background potential flow field}
\author{Andreas H. Akselsen}
\date{\today}
\maketitle

\section{HOS model extension}
\subsection{Boundary conditions}
The kinematic and dynamic boundary conditions respectively read
\begin{subequations}
\begin{align}
\frac{\mr D \xi}{\mr D t}  = \eta_t + \nabla \phit \cdot\nabla \xi &= 0  \quad \text{at }z = \eta, 
\\
\phit_t + \frac12|\nabla\phit|^2 + g\eta &= 0  \quad  \text{at }z = \eta.
\end{align}%
\label{eq:BC0}%
\end{subequations}%
%\begin{equation}
%%\left.
%\begin{aligned}
%\eta_t &= \nabla \phit \cdot\nabla \xi\\
%\phit_t &= -\frac12|\nabla\phit|^2 - g\eta
%%\quad \text{at }z = \eta.
%\end{aligned}
%%\right\}
%\qquad \text{at }z = \eta.
%\end{equation}
Here, $\phit(x,y,z,t)$ is the full velocity potential and $\xi = \eta(x,y,t)-z$, $z = \eta$ being the free surface.
The full velocity potential is split in two parts, $\phit = \phi + \Phi$, where $\Phi$ is a known background potential flow field and $\phi$ is the wave field enforcing the boundary conditions \eqref{eq:BC0}.
$\Phi(x,y,z,t)$, being user-defined, is known everywhere and its derivatives can be evaluated directly at $z=\eta$.
For $\phi$ it is beneficial for convergence to express the boundary conditions in terms of a potential defined at the free surface, and so we introduce 
$\phi\S(x,y,t) \equiv \phi[x,y,\eta(x,y,t),t]$.%
\footnote{This idea originates from Zakharov and it's beneficial convergence demonstrated by \citet{west1981deep}.}
The chain rule yields the following relationships
\begin{align*}
\phi_t\big|_{z=\eta} &= \phi_t\S - W\eta_t,\\
(\nabla\phi)\big|_{z=\eta}  &= \nabla\phi\S - W\nabla\xi , 
\end{align*}
where $W\equiv \phi_z\big|_{z=\eta}$.
Introducing these relationships into \eqref{eq:BC0} we find after a bit of algebra
\begin{subequations}
\begin{align}
\eta_t &= - \nabla \xi\cdot\big[\nabla\phi\S + (\nabla\Phi)_{z=\eta} - W\nabla\xi\big], 
\\
\phi\S_t &= -\Phi_t\big|_{z=\eta} - \frac12\big|\nabla\phi\S +(\nabla\Phi)_{z=\eta}\big|^2 + \frac12|W\nabla\xi|^2 - g\eta.
\end{align}
\label{eq:BCS}
\end{subequations}
Expanded for the sake of comparison, we have
\begin{align*}
\eta_t &=   - \nabla\eta\cdot\nabla\phi\S  - \nabla\eta\cdot(\nabla\Phi)_{z=\eta} + (\Phi_z)_{z=\eta}   + \big(1+|\nabla\eta|^2\big)W, 
\\
\phi\S_t &= -\Phi_t\big|_{z=\eta} - \frac12\big|\nabla\phi\S +(\nabla\Phi)_{z=\eta}\big|^2 + \frac12\big(1+|\nabla\eta|^2\big)W^2 - g\eta.
\end{align*}
%
We have chosen to use the background velocities evaluated at $z=\eta$ as these are everywhere available.%
\footnote{Note that all derivatives of $\Phi$ are also available so it is possible to instead use a complete surface potential function $\phit\S\equiv\phit[x,y,\eta(x,y,t),t]$, $\tilde W\equiv \phit_z\big|_{z=\eta}$, and solve
%\begin{subequations}
\begin{align*}
\eta_t &= - \nabla \xi\cdot\big[\nabla\phit\S - \tilde W\nabla\xi\big], 
&
\phit\S_t &= - \frac12\big|\nabla\phit\S\big|^2 + \frac12|\tilde W\nabla\xi|^2 - g\eta.
\end{align*}
%\end{subequations}
%with $\tilde W\equiv \phit_z\big|_{z=\eta}$.
}

\subsection{The background flow}
Classical potential flow constructs is here adopted to generate a potential flow field that satisfies the conditions
\begin{subequations}
\begin{alignat}{2}
\nabla^2\Phi &= 0 &\quad&\\
\Phi_z &=0 & &\text{at } z=0\\
\Phi_x, \Phi_y &= 0\text{ or periodic} & &\text{at } \{x,y\} = 0,L_{\{x,y\}} \text{ depending on domain.}
\label{eq:}%
\end{alignat}%
\end{subequations}%
These flows can be constructed in three-dimensional domains, either as three-dimensional flows or as two-dimensional flows uniform an a third dimension.
We will here consider demonstrate two-dimensional flows only.

We construct the background flow by superposing elementary complex potential flow solutions $f(\zeta)$.
These are defined by $f(\zeta) = \Phi(x,z) + \ii\Psi(x,z)$, $\zeta = x + \ii z$, $\Psi$ being the stream function.
We mirror each solution about the plane $z=0$ by adding the conjugate solution $f^*(\zeta^*)$. This ensures that no streamlines crosses the plane $z=0$.
\begin{equation}
f(\zeta) = \sum_j \big[ A_j\ln(\zeta-\zeta_j) + A_j^*\ln(\zeta-\zeta_j^*) \big]
\label{eq:f}
\end{equation}
where real positive/negative $A_j$ constitutes a source/sink at $\zeta_j$ and imaginary positive/negative $A_j$ constitutes a clockwise/counter-clockwise rotating line vortex at $\zeta_j$.
The  derivative
\begin{equation}
f'(\zeta) = \sum_j \bigg[ \frac{A_j}{\zeta-\zeta_j} + \frac{A_j^*}{\zeta-\zeta_j^*} \bigg] = U^*(\zeta)
\label{eq:df}
\end{equation}
directly yields the velocity field $U$. In real coordinates, $\Phi_x = \rm{Re}[f'(x+\ii z)$], $\Phi_z = -\rm{Im}[f'(x+\ii z)]$.
We normalize the intensities $A_j$ according to the velocity intensity brought about at $z=0$ by the element relative to the phase velocity:
\[
F_j = \frac{2A_j}{c_p |\mr{Im}\zeta_j|}
\]
One can also impose doublets
\begin{equation}
f_j^\mr{d} = \frac{B_j}{\zeta-\zeta_j}+\frac{B_j^*}{\zeta-\zeta_j^*},\quad (f^\mr{d}_j)' = -\frac{B_j}{(\zeta-\zeta_j)^2}-\frac{B_j^*}{(\zeta-\zeta_j^*)^2},
\label{eq:doublet}
\end{equation}
where the angle of $B_j$ dictates the doublet direction and its magnitude the intensity.
Periodic boundaries in the horizontal plane are closely approximated by repeating the domain a number of times in the horizontal direction:
%$\{A_j\}:=\{\ldots,\{A_j\},\{A_j\},\{A_j\},\ldots\}$, $\{\zeta_j\}:=\{\ldots,\{\zeta_j-L_x\},\{\zeta_j\},\{\zeta_j+L_x\},\ldots\}$
$\{A_j\}\coloneqq\bigcup_{n=-N}^N \{A_j\}$, $\{\zeta_j\}\coloneqq\bigcup_{n=-N}^N\{\zeta_j + n L_x \}$.

\bibliographystyle{plainnat} % abbrvnat,plainnat,unsrtnat
\bibliography{bib} %You need a file 'literature.bib' for this.


\end{document}
