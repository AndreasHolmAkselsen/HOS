




\section{A direct mapping technique for matching the surface potential.}
\label{sec:directMapping}
Even though the map \eqref{eq:map_double} cannot be inverted analytically, a numerical inversion function $\zzmap$ based on interpolation can easily be constructed. 
Crucially, the bathymetry mapping is constant throughout any simulation, meaning that the minor computation associated with this inverse mapping is done in advance, 
and the surface in the $\zz$-plane obtained at $\zz\S(\x)=\zzmap[\x+\ii h(\x)]$.
The inverse map $\zzmap$ is composed from scattered interpolation of a structured rectangle in the $\zz$-plane, fitted to contain the surface $\zz\S$.
The main challenge and goal of such a construct is then to be able to construct a complex potential $\ww(\zz)$ from \eqref{eq:ww_logstrip} or \eqref{eq:ww_double}, depending on map, whose real part coincides with $\phi\S(\x)$ at $\zz=\zz\S(\x)$.

As discussed in \autoref{sec:finiteDepth}, the modes $\h\ww_j$ cannot explicitly be related to $\phi\S$ because the complex potential is also constructed to account for the impermeable lower boundary. 
We can however express the system for the discrete points $\z_i,\zz_i$ which can easily be solved numerically.
Incorporating the condition $\h\ww_{-j}=\h\ww_{j}^*$, the system is written
\begin{equation}
\h\phi_0 + 2\sum_{j=1}^J (\h\phi_j \cos\kk_j \xx_i - \h\psi_j \sin\kk_j \xx_i )\frac{\cosh \kk_j(\yy_i+\yy_0)}{\cosh\kk_j\tilde H} = \phi\S_i,
%\Re\ww_0 + 2\sum_{j=1}^J \Re\left\{\h\ww_j \ee^{\ii \kk_j \xx_i}\right\}\frac{\cosh \kk_j(\yy_i+\yy_0)}{\cosh\kk_j\tilde H} = \phi\S_i,
\label{eq:syst_ww}
\end{equation}
to be solved for the real and imaginary parts of 
%$\h\ww_j$; $\Re\h\ww_j \ee^{\ii \kk_j \xx_i} = \h\phi_j \cos\kk_j \xx_i - \h\psi_j \sin\kk_j \xx_i $
$\h\ww_j = \h\phi_j + \ii \h\psi_j$. 
$\yy_0$ and $\tilde H$ both equal $\pi$ in the extended maps \eqref{eq:map_logstrip} and \eqref{eq:map_double}, and equals respectively zero and some representative height (e.g., $\Im \zzmap(0)$) with the simpler map \eqref{eq:mapSC}.
The imaginary part of $\h\ww_0$ is left arbitrary.

It is found numerically stabilizing to integrate $\phi\S(x)$ such that $\xx_i$ are equidistant.  
This ensures that the system \eqref{eq:syst_ww} corresponds to a normal Fourier transform in the case of a flat surface $\yy_i=0$, such that $\h\ww_j = \h\phi_j\S \ee^{\ii k_j \x_0}$ ($\h\phi_j\S$ being the Fourier coefficients of $\phi\S(\x)$ after interpolation).
Robustness of the method then gradually decreases with increasing surface amplitudes as the components in \eqref{eq:syst_ww} become less orthogonal.
It is possible that robustness can be improved further by other interpolation choices or by over-determining system \eqref{eq:syst_ww}, including fewer wave modes than there are spatial points.
\\

Examples of thus computed potential fields are presented in this section for an arbitrary wave $h(\x) = a\cos(k_0 \x + \varphi)$ with an arbitrary surface potential $\phi\S(\x) = b\sin(k_0 \x + \varphi_1) + c\sin(2k_0 \x + \varphi_2)$ in the $\z$-plane.
\autoref{fig:res:simple} shows results for the simpler transformation \eqref{eq:mapSC}.
Dashed black lines indicate the interpolation domain included in the inverse mapping $\zzmap$.
We see that a smooth potential field has been found in the right half of the domain. 
The left half shows strong high-frequency variation. 
This is because to solution obtained form solving \eqref{eq:syst_ww} relied on high-wavenumber modes in this region; it matches $\phi\S$ exactly at the interpolated discrete points, but fluctuates wildly in their neighbourhood.

Better robustness is seen in \autoref{fig:res:logstip} showing the extended map \eqref{eq:map_logstrip} with the coinciding flat planes $\yy=0$ where $\y=0$. 
Fairly large amplitudes can be imposed before fluctuations at the surface solution are observed. 
Furthermore, robustness is fairly insensitive to the degree of depth transition.

Several examples more are presented in \autoref{fig:res:double1}--\ref{fig:res:double3}, here for the double step \eqref{eq:map_double}.
This map has the additional advantage that the domain becomes periodic in both $\zz$ and $\z$ planes.






\begin{figure}[h!ptb]%
\centering
\subfloat[Long waves]{\includegraphics[width=.5\columnwidth]{../conformalMapping/figures/simple_nWaves2_h1_1_h2_0p5_nx101_aEta0p06_kCutF1_L5.pdf}}%
{\vrule width 1pt}%
\subfloat[Short waves]{\includegraphics[width=.5\columnwidth]{../conformalMapping/figures/simple_nWaves4_h1_1_h2_0p5_nx101_aEta0p03_kCutF1.pdf}}%
\caption{Simple Schwartz-Christoffel geometry \eqref{eq:mapSC}}%
\label{fig:res:simple}%
\end{figure}


\begin{figure}[h!ptb]%
\centering
\subfloat[Long waves]{\includegraphics[width=.5\columnwidth]{../conformalMapping/figures/logstip_nWaves2_h1_1_h2_0p5_nx101_aEta0p075_kCutF1_L5.pdf}}%
{\vrule width 1pt}%
\subfloat[Short waves]{\includegraphics[width=.5\columnwidth]{../conformalMapping/figures/logstip_nWaves4_h1_1_h2_0p5_nx101_aEta0p03_kCutF1_L5.pdf}}%
\caption{Extended Schwartz-Christoffel geometry \eqref{eq:map_logstrip}}%
\label{fig:res:logstip}%
\end{figure}

\begin{figure}[h!ptb]%
\centering
\subfloat[Long waves]{\includegraphics[width=.5\columnwidth]{../conformalMapping/figures/double_nWaves2_h1_0p5_h2_1_nx101_aEta0p05_kCutF1.pdf}}%
{\vrule width 1pt}%
\subfloat[Short waves] {\includegraphics[width=.5\columnwidth]{../conformalMapping/figures/double_nWaves4_h1_0p4_h2_1_nx101_aEta0p03_kCutF1.pdf}}
\caption{Extended Schwartz-Christoffel geometry \eqref{eq:map_double} (1)}%
\label{fig:res:double1}%
\end{figure}
\begin{figure}[h!ptb]%
\centering
{\includegraphics[width=.33\columnwidth]{../conformalMapping/figures/double_nWaves2_h1_0p25_h2_1_nx101_aEta0p05_kCutF1.pdf}}%
{\includegraphics[width=.33\columnwidth]{../conformalMapping/figures/double_nWaves2_h1_0p25_h2_1_nx201_aEta0p075_kCutF1_L10.pdf}}
{\includegraphics[width=.33\columnwidth]{../conformalMapping/figures/double_nWaves4_h1_0p25_h2_1_nx101_aEta0p03_kCutF1.pdf}}
\caption{Extended Schwartz-Christoffel geometry \eqref{eq:map_double} (2)}%
\label{fig:res:double2}%
\end{figure}
\begin{figure}[h!ptb]%
\centering
\subfloat[Long waves] {\includegraphics[width=.5\columnwidth]{../conformalMapping/figures/double_nWaves2_h1_1_h2_0p25_nx101_aEta0p05_kCutF1.pdf}}
{\vrule width 1pt}%
\subfloat[Short waves] {\includegraphics[width=.5\columnwidth]{../conformalMapping/figures/double_nWaves4_h1_1_h2_0p5_nx101_aEta0p1_kCutF1.pdf}}%
\caption{Extended Schwartz-Christoffel geometry \eqref{eq:map_double} (3)}%
\label{fig:res:double3}%
\end{figure}



