 %\documentclass[a4paper,12pt]{article}
%%\documentclass{amsart}
%\usepackage[a4paper, total={17cm, 25cm}]{geometry}
%\usepackage[utf8]{inputenc}
%\usepackage[english]{babel}
%\usepackage{amsmath,bm,amsfonts,amssymb}
%\usepackage{xcolor}
%\usepackage{graphicx}
%
%\usepackage{graphbox}
%
%\usepackage[round]{natbib}
%\usepackage{mathtools}
%\usepackage[font=normalsize]{subfig}
%\usepackage{float}
%\usepackage{hyperref}
%\usepackage{enumitem}
%\hypersetup{colorlinks=true, 	
			%linkcolor=blue,
			%filecolor=blue,
			%urlcolor=blue,
			%citecolor=blue}
%\newcommand{\mr}{\mathrm}
%\newcommand{\mc}{\mathcal}
%\let\SSS\S
%\renewcommand{\S}{^\mr{S}}
%\newcommand{\ii}{\mr{i}\,}
%\newcommand{\ee}{\mr{e}}
%%\newcommand{\phit}{\psi}
%\newcommand{\phit}{\tilde\phi}
%\newcommand{\br}[3]{\left#1#2\right#3}
%\let\underscore\_
%\renewcommand{\_}[1]{_\mr{#1}}
%\newcommand{\oo}[1]{^{(#1)}}
%\newcommand{\rr}{\bm r}%{x,y}
%\newcommand{\cp}{c\_p}
%\let\Re\relax
%\let\Im\relax
%\DeclareMathOperator\Re{Re}
%\DeclareMathOperator\Im{Im}
%\newcommand{\w}{w}
%\newcommand{\bU}{\bm U}
%\newcommand{\h}{\hat}
%\newcommand{\rbr}[1]{\left(#1\right)}
%\newcommand{\sbr}[1]{\left[#1\right]}
%\newcommand{\cbr}[1]{\left\{#1\right\}}
%\newcommand{\z}{z}
%\newcommand{\x}{x}
%\newcommand{\y}{y}
%\newcommand{\zz}{\zeta}
%\newcommand{\xx}{\xi}
%\newcommand{\yy}{\sigma}
%%\newcommand{\k}{k}
%\newcommand{\kk}{\kappa}
%
%\newcommand{\zmap}{f}
%%\newcommand{\zzmap}{\zmap^{-1}}
%\newcommand{\zzmap}{\zmap^{\raisebox{.2ex}{$\scriptscriptstyle-1$}}}
%
%\newcommand{\ww}{\omega}
%\renewcommand{\w}{w}
%\newcommand{\surf}{\eta}
%\newcommand{\dd}[2]{\frac{\mr d #1}{\mr d #2}}
%
%\begin{document}
%

Results shown here pertain to HOS simulations using the algebraic conformal maps described in \autoref{sec:SC} and the approach of \autoref{sec:zz-planeApproach}.
Simulation over the step shown in \autoref{fig:res:map_logstrip} (equation \ref{eq:map_logstrip}) is presented in \autoref{fig:res:logstrip1} to \ref{fig:res:logstrip3}.
These simulations take about five minutes each to run on a piece-of-shit laptop.

\begin{figure}[h!ptb]%
\centering
\subfloat[Full domain]{\includegraphics[width=.5\columnwidth,align=c]{../HOS_bathymetry/figures/map/map_logstrip_SSGW_ka0p05_H1p00_0p50_Nw60.pdf}}%
\subfloat[Corner, 1--to--1.]{\includegraphics[width=.5\columnwidth,align=c]{../HOS_bathymetry/figures/map/mapZoom_logstrip_SSGW_ka0p05_H1p00_0p50_Nw60.pdf}}
\caption{Conformal map, single step; $H\_d = 1.0$, $H\_s = 0.5$}%
\label{fig:res:map_logstrip}%
\end{figure}

\begin{figure}[h!ptb]%
\centering
\includegraphics[width=1\columnwidth]{../HOS_bathymetry/figures/logstrip_SSGW_ka0p05_M5_H1p00_0p75_Nw60_dt5T_nx3840_pad0_ikCutInf_Md0p5_r0p25.pdf}%
\caption{Surface elevation, $(ka)\_L = 0.05$, $(kH)\_L = 1.00$ ($H\_d=1.0$\,m corresponds to $T\approx2.30$\,s); single step, $H\_s/H\_d = 0.75$. Dashed line indicates $\x$-location of step transition.}%
\label{fig:res:logstrip1}%
\end{figure}
\begin{figure}[h!ptb]%
\centering 
\includegraphics[width=1\columnwidth]{../HOS_bathymetry/figures/logstrip_SSGW_ka0p05_M5_H1p00_0p50_Nw60_dt5T_nx3840_pad0_ikCutInf_Md0p5_r0p25.pdf}%
\caption{Similar to \autoref{fig:res:logstrip1}, but with $H\_s/H\_d = 0.50$.}%
\label{fig:res:logstrip2}%
\end{figure}
\begin{figure}[h!ptb]%
\centering 															  
\includegraphics[width=1\columnwidth]{../HOS_bathymetry/figures/logstrip_SSGW_ka0p05_M5_H1p00_0p35_Nw60_dt5T_nx3840_pad0_ikCutInf_Md0p5_r0p25.pdf}%
\caption{Similar to \autoref{fig:res:logstrip1}, but with $H\_s/H\_d = 0.35$.}%
\label{fig:res:logstrip3}%
\end{figure}


\begin{figure}[h!ptb]%
\centering
\includegraphics[width=\columnwidth]{../HOS_bathymetry/linearStepTheory/contourPlot.pdf}%
\caption{Example of linear theory solution by discontinuity matching. ($H\_d = 1.0$, $H\_s = 0.50$, $kH\_d = 1.00$.) Solid black and dashed red lines are values at left and right side of discontinuity $x=0$, respectively.}%
\label{fig:linearReflection:contour}%
\end{figure}

\begin{figure}[h!ptb]%
\centering
\includegraphics[width=.5\columnwidth]{../HOS_bathymetry/linearStepTheory/R0.pdf}%
\includegraphics[width=.5\columnwidth]{../HOS_bathymetry/linearStepTheory/R0_k.pdf}%
\caption{Reflection coefficients form linear theory}%
\label{fig:linearReflection:R0}%
\end{figure}



A comparison to linear theory and to the second-order nonlinear theory of \citet{li_2021_step1} is made in \autoref{tab:compareTheory} for the simulations shown in \autoref{fig:res:logstrip1} to \ref{fig:res:logstrip3}.
(See also the memo \citet{AHA_2021_LiTheory} summarizing \citeauthor{li_2021_step1}'s second-order theory in relation to design decisions of SINTEF Ocean Space Center.)
The relative depth is $(kH)\_L=1.00$,  corresponding to $T/\sqrt{H\_d/g}=7.20$.
Both linear reflected and second-order free transmitted wave packet amplitudes correspond reasonably well to observations.
Discrepancies should likely be attributed to wave packet dispersion, wave packet stability and numerical damping, all of which contributes to the observed amplitudes being smaller than the theoretical ones.
Inaccuracies also arise from estimating the initial primary mode amplitude as half the wave height and form a small portion of the initial energy travelling in the negative direction.

Third and possibly forth-order transmitted wave packets are discernible in the shallowest simulation shown in \autoref{fig:res:logstrip3}.
We can also see in the final panel the emergence of what is likely the second-order sub-harmonic transmitted wave packet. 
Note that wave packet stability and also the numerical resolution and damping may affect results somewhat. 
\\


\begin{table}[h!ptb]%
\centering
\begin{tabular}{c|ccc}
$H\_s/H\_d$ & 0.75 & 0.50 & 0.35\\\hline
$R_0$ & 0.054 & 0.135 & 0.208 \\
$R\_{HOS}^{(1)}$ &  0.052 & 0.132 & 0.20\\\hline
$T_{20}$ & 1.93 & 7.57& 18.58 \\
$T^{(2)}\_{HOS}$ & 1.6 & 6.32 & 14.4
\end{tabular}
\caption{
Comparison with theory of simulation results for $(kH)\_L=1.00$ shown in \autoref{fig:res:logstrip1} to \ref{fig:res:logstrip3}.
Reflection coefficients $R_0$ from linear theory (\autoref{fig:linearReflection:contour}, \ref{fig:linearReflection:R0})
and coefficient $T_{20}$ for second-order transmitted free wave packet from the theory by \citet{li_2021_step1}.
Observed linear reflected and second-order free transmitted wave amplitudes are estimated by visual inspection of the fifth and eight plot panel from below, respectively.
$T^{(2)}\_{HOS} = A_{20}/(2A_0^2 \omega/g)$ is adopted for estimating the second-order coefficients from plots, $A_0$ being the incident packet amplitude and $A_{20}$ the observed second-order free transmitted packet amplitude. 
}
\label{tab:compareTheory}
\end{table}



Wave propagation over the plateau configuration (equation \ref{eq:map_double}) is considered in \autoref{fig:res:double}.
We here have linear reflections associated with both steps of the plateau, as well as the excretion of second-order reflected and transmitted packets.
The main spurious wave packets observed across the plateau are the linear reflected packet from the rear step and the second-order free transmitted packet from the front step. 
These will in turn partially reflect when reaching the opposite step, albeit less intensely. 
Also distinct is the second-order sub-harmonic free wave travelling ahead of the main wave group after interacting with the bathymetry. 
Amplitudes of  sub-harmonic packets can also be estimated with the theory of \citet{li_2021_step1}, although we do not do so here. 
Passive absorption of sub-harmonic  (surge-type) waves is inefficient such that these may persist in a wave tank for a long time.

Images also contain some pollution form a wave packet associated with imprecision in the initial conditions. This packet travels in the negative direction through the periodic boundary.
The simulation has been terminated at the point in time when the linear wave reflected at the up-step reaches the down-step via the periodic boundary. 


\begin{figure}[h!ptb]%
\centering
\subfloat[Map]{\includegraphics[width=1\columnwidth]{../HOS_bathymetry/figures/map/map2_double_SSGW_ka0p05_H1p00_0p50_Nw60.pdf}}\\
\subfloat[Surface elevation]{\includegraphics[width=1\columnwidth]{../HOS_bathymetry/figures/double_SSGW_ka0p05_M5_H1p00_0p50_Nw60_dt7p5T_nx3840_pad0_ikCutInf_Md0p5_r0p25.pdf}}%
\caption{Double step, $(ka)_0 = 0.05$, $(kH)_0 = 1.00$, $H\_s/H\_d = 0.50$. Plateau length is 35\% of domain length.}%
\label{fig:res:double}%
\end{figure}




%\big|\zeta_{0T,f}^{(22,0)}\big| &= \frac{2\omega_0}{g}|T_{20}| A_0^2.


%\end{document}