%\documentclass[a4paper,12pt]{article}
%%\documentclass{amsart}
%\usepackage[a4paper, total={17cm, 25cm}]{geometry}
%\usepackage[utf8]{inputenc}
%\usepackage[english]{babel}
%\usepackage{amsmath,bm,amsfonts,amssymb}
%\usepackage{xcolor}
%\usepackage{graphicx}
%
%\usepackage{stmaryrd}
%\usepackage[round]{natbib}
%\usepackage{mathtools}
%\usepackage[font=normalsize]{subfig}
%\usepackage{float}
%\usepackage{hyperref}
%\usepackage{enumitem}
%\hypersetup{colorlinks=true, 	
			%linkcolor=blue,
			%filecolor=blue,
			%urlcolor=blue,
			%citecolor=blue}
%\newcommand{\mr}{\mathrm}
%\newcommand{\mc}{\mathcal}
%\let\SSS\S
%\renewcommand{\S}{^\mr{S}}
%\newcommand{\ii}{\mr{i}\,}
%\newcommand{\ee}{\mr{e}}
%%\newcommand{\phit}{\psi}
%\newcommand{\phit}{\tilde\phi}
%\newcommand{\br}[3]{\left#1#2\right#3}
%\let\underscore\_
%\renewcommand{\_}[1]{_\mr{#1}}
%\newcommand{\oo}[1]{^{(#1)}}
%\newcommand{\rr}{\bm r}%{x,y}
%\newcommand{\cp}{c\_p}
%\let\Re\relax
%\let\Im\relax
%\DeclareMathOperator\Re{Re}
%\DeclareMathOperator\Im{Im}
%\newcommand{\w}{w}
%\newcommand{\bU}{\bm U}
%\newcommand{\h}{\hat}
%\newcommand{\rbr}[1]{\left(#1\right)}
%\newcommand{\sbr}[1]{\left[#1\right]}
%\newcommand{\cbr}[1]{\left\{#1\right\}}
%\newcommand{\z}{z}
%\newcommand{\x}{x}
%\newcommand{\y}{y}
%\newcommand{\zz}{\zeta}
%\newcommand{\xx}{\xi}
%\newcommand{\yy}{\sigma}
%%\newcommand{\k}{k}
%\newcommand{\kk}{\kappa}
%
%
%\newcommand{\zmap}{f}
%%\newcommand{\zzmap}{\zmap^{-1}}
%\newcommand{\zzmap}{\zmap^{\raisebox{.2ex}{$\scriptscriptstyle-1$}}}
%
%\newcommand{\ww}{\omega}
%\renewcommand{\w}{w}
%\newcommand{\surf}{\eta}
%\newcommand{\dd}{\mr d}
%\newcommand{\ddfrac}[2]{\frac{\dd #1}{\dd #2}}
%
%\begin{document}










Let us assume that our mapping is time dependent, i.e.,
\begin{equation}
z \mapsfrom f(\zz,t).
\label{eq:}
\end{equation}
Such a map can for example account for the action of a wavemaker on the body of water.
We now derive the associated system equations in terms of $\zz$-variables.
The kinematic boundary condition states that a fluid particle follows the surface, also in the $\zz$-plane. 
Fluid particle velocity as seen from the $\zz$-plane is
\[
\mc U\_p = \ddfrac{\zz\_p}{t} = (\zzmap)_z \ddfrac{\z\_p}{t} + (\zzmap)_t = \frac{\w_z^*}{\zmap_\zz} + (\zzmap)_t =  \frac{\ww_\zz^*}{|\zmap_\zz|^2} + (\zzmap)_t,
\]
subscript `p' indicating fluid particle position.
The kinematic boundary condition thus becomes
\begin{equation}
\eta_t + \Re \mc U\_p \eta_\xx = \Im \mc U\_p.
\label{eq:BC_kin_transient}
\end{equation}

The kinematic condition requires an the inverse map time derivative $(\zzmap)_t$. 
Utilizing the property $(\zzmap)_\z = 1/\zmap_\zz$, we find
\begin{equation}
(\zzmap)_t = \int \!\big(1/\zmap_\zz\big)_t \,\dd z = -\int\! \frac{\zmap_{\zz t}}{\zmap_\zz} \,\dd\zz.
\label{eq:zzmap_t}
\end{equation}
Integrating \eqref{eq:zzmap_t} is the same challenge as we face with the Schwartz--Cristoffel transform where
analytic expressions exist for some maps but not all. Numerical integration can be performed where analytic expressions are not forthcoming. 
\\

Deriving the dynamic boundary condition is straight-forwart when adopting the complex potentials $\ww(\zz,t)=\w[\zmap(\zz,t),t]$; $\phi=\Re\w$, $\varphi = \Re\ww$.
This yields
\begin{equation}
\varphi_t - \Re\sbr{\ww_\zz \frac{\zmap_t}{\zmap_\zz}} + \frac12|\zmap_\zz|^{-2}|\ww_\zz|^2 + gh = 0.
\label{eq:}
\end{equation}
Further introducing $\varphi\S(\xx,t)=\varphi[\xx,\eta(\xx,t),t]$ to eliminate $\varphi_t$ and $\varphi_\xx$, we get 
\begin{equation*}
\varphi\S_t - \Re\sbr{\ww_\zz \frac{\zmap_t}{\zmap_\zz}}-\varphi_\yy \eta_t
 + \frac12|\zmap_\zz|^{-2}\sbr{\rbr{\varphi\S_\xx-\varphi_\yy\eta_\xx}^2+\varphi_\yy^2} + gh = 0.
\end{equation*}
%
Eliminating $\eta_t$ with \eqref{eq:zzmap_t} finally produces
\begin{equation}
\varphi\S_t - \Re\sbr{\ww_\zz \frac{\zmap_t}{\zmap_\zz}}
+ \varphi_\yy\sbr{\Re(\zzmap)_t-\eta_\xx\Im(\zzmap)_t}
 + \frac12|\zmap_\zz|^{-2}\sbr{\rbr{\varphi\S_\xx}^2-\rbr{1+\eta_\xx^2}\varphi_\yy^2} + gh = 0.
\label{eq:BC_dyn_transient}
\end{equation}
\\

An example of a Schwartz-Christoffel type transform of a flap wavemaker is
\begin{equation}
\zmap_\zz(\zz,t) = \prod_{j=-\infty}^\infty \rbr{1+\frac{d^2}{\rbr{\zz+2\ii j h}^2}}^{\theta(t)/\pi}
= \sbr{1+\frac{\sin^2\!\frac{\pi d}{2h}}{\sinh^2\!\frac{\pi \zz}{2h}}}^{\theta(t)/\pi}
\label{eq:df_flap}
\end{equation}
This map is a $\theta$-bend in the vertical axis mirrored about the horizontal axis.
This is then mirrored again infinitely many times for every $2\ii h$-period---a product series which expressible in terms of trigonometric functions.
\eqref{eq:df_flap} is unfortunately to our knowledge not integrable, so integration must be carried out numerically. 
We note that $\zmap_\zz\to1$ as $\xx\to\infty$ and so the $\z$ plane will scale with the $\zz$ plane far away from the wavemaker, and the two planes will match everywhere for $\theta=0$. $\zz=-\ii d$, which is the hinge position in the $\zz$-plane, will be in proximity to the hinge position in the $\z$-plane, but will shifted vertically with extending flap angles. We perform a numerical iteration in order to determine the value of $d$ to achieve the descried hinge depth $D$ in the physical plane. This is done by measuring the distance from the bottom to the hinge;
let
\begin{equation}
\mc L(d/h,\theta) = \int_{-1}^{-d/h} \dd \yy \sbr{1-\frac{\sin^2\!\frac{\pi}{2}\frac{d}{h}}{\sin^2\!\frac{\pi}{2}\yy}}^{\theta/\pi} = \frac{f(-\ii d)-f(-\ii h)}h
\label{eq:L}
\end{equation}
We iterate by assigning
\[
d\leftarrow d + h(1-\mc L) - D.
\]

The map here given has a horizontal line of symmetry along $\y=0$. The wavemaker forms a wedge at this level. 
For this not to interfare with the water level, we propose shifting the waterline down a small distance $\y=-\hbar$.
This is easily done by substituting 
%\begin{align*}
%h &\leftarrow h + \hbar, & D &\leftarrow D + \hbar, & \z &\leftarrow \z + \ii \hbar.
%\end{align*}
$h \leftarrow h + \hbar$, $D \leftarrow D + \hbar$ and $\z \leftarrow \z + \ii \hbar$, right-hand side lengths being lengths with respect to the waterline.
\\

\autoref{fig:f_flap} shows an example of \eqref{eq:df_flap} integrated numerically.
%We note that the floor near the wavemaker is with this map somewhat distorted which will cause some error in non-deep water.
\\
\begin{figure}[h!ptb]%
\centering
\subfloat[$\theta = +30$\textdegree]{\includegraphics[width=.5\columnwidth]{../conformalMapping/conformalWavemaker/mapFlap3_positive.pdf}}%
\subfloat[$\theta = -30$\textdegree]{\includegraphics[width=.5\columnwidth]{../conformalMapping/conformalWavemaker/mapFlap3_negative.pdf}}%
\caption{The conformal map \eqref{eq:df_flap}, integrated numerically.
$D=1.0$, $h=2.0$ and $\hbar = 0.25$.
}%
\label{fig:f_flap}%
\end{figure}



We may for deep water ignore the periodic mirroring in \eqref{eq:df_flap}.
This expression is integrable in terms of the  ordinary hypergeometric function ${_2F}_1$:
\begin{equation}
\tilde\zmap(\zz,t) = \int\!\dd\zz\rbr{1+\frac{b^2}{\zz^2}}^{\theta(t)/\pi}=\frac{d}{1-2\frac\theta\pi}\,{_2F}_1\rbr{-\frac{\theta}{\pi},\frac12-\frac{\theta}{\pi},\frac32-\frac{\theta}{\pi},-\frac{\zz^2}{d^2}}\rbr{\frac{\zz}{d}}^{1-2\frac\theta\pi} + c.
\label{eq:f_flap}
\end{equation}
Equation \ref{eq:df_flap} can be integrated numerically if the hypergeometric functions are not available.\footnote{${_2F}_1$ is in MATLAB only part of the symbolic toolbox.}

The condition that the flap hinge depth be a vertical distance $D$ from the quiescent waterline at $\x=0$, and that the map return be unaffected by the wavemaker motion far away ($\lim_{\xx\to\infty}\zmap(\xi,t) = \xi$) yields the coefficients
\begin{align*}
%c &= -(D+\hbar)\tan\theta + \ii \hbar,&
%d &=  \frac{\sqrt\pi(D+\hbar)\sec\theta}{\Gamma\rbr{1+\frac\theta\pi}\Gamma\rbr{\frac12-\frac\theta\pi}}.
c = -D\tan\theta,\qquad
d =  \frac{\sqrt\pi D\sec\theta}{\Gamma\rbr{1+\frac\theta\pi}\Gamma\rbr{\frac12-\frac\theta\pi}}.
\end{align*} 
$\hbar$ is here the distance from the quiescent waterline to the line of symmetry around which the map is horizontally mirrored.

Expression \eqref{eq:zzmap_t} is integrable with for the non-mirrored domain, producing
\begin{equation*}
(\tilde\zzmap)_t = -\sbr{2d\arctan\frac\zz d + \zz \ln\rbr{1+\frac{d^2}{\zz^2}}}\frac{\theta_t}{\pi}.
\end{equation*}
It is also integrable with \eqref{eq:df_flap} itself, although the expression we get is a mess not quoted here. 


%\end{document}