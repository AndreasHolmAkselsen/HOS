\documentclass[a4paper,12pt]{article}
%\documentclass{amsart}

\usepackage[a4paper, total={17cm, 25cm}]{geometry}
\usepackage[utf8]{inputenc}
\usepackage[english]{babel}
\usepackage{amsmath,bm,amsfonts,amssymb}
\usepackage{xcolor}
\usepackage{graphicx}
\usepackage[round]{natbib}
\usepackage{mathtools}
\usepackage[font=normalsize]{subfig}
\usepackage{float}
\usepackage{hyperref}
\usepackage{stmaryrd}%\mapsfrom
%\usepackage{cprotect}%for \verb in captions
%\usepackage{enumerate}
\usepackage{enumitem}
\hypersetup{colorlinks=true,
			linkcolor=blue,
			filecolor=blue,
			urlcolor=blue,
			citecolor=blue}
\newcommand{\mr}{\mathrm}
\newcommand{\mc}{\mathcal}
\let\SSS\S
\renewcommand{\S}{^\mr{S}}
\newcommand{\ii}{\mr{i}\,}
\newcommand{\ee}{\mr{e}}
%\newcommand{\varphit}{\psi}
\newcommand{\varphit}{\tilde\varphi}
\newcommand{\br}[3]{\left#1#2\right#3}
\let\underscore\_
\renewcommand{\_}[1]{_\mr{#1}}
\newcommand{\oo}[1]{^{(#1)}}
\newcommand{\rr}{\bm r}%{x,y}
\newcommand{\cp}{c\_p}
\let\Re\relax
\let\Im\relax
\DeclareMathOperator\Re{Re}
\DeclareMathOperator\Im{Im}
\newcommand{\w}{w}
\newcommand{\bU}{\bm U}
\newcommand{\h}{\hat}
\newcommand{\rbr}[1]{\left(#1\right)}
\newcommand{\sbr}[1]{\left[#1\right]}
\newcommand{\cbr}[1]{\left\{#1\right\}}
%\newcommand{\bU}{(\nabla\phi)_{z=\eta}}

\newcommand{\refCS}{C\&S}

\input{listingsEnvirorment.sty}

\newcommand{\lref}[2]{\href{#2}{#1~\ref*{#2}}}%

%\newcommand{\zz}{z}
%\newcommand{\xx}{x}
%\newcommand{\yy}{y}
%\newcommand{\z}{\zeta}
%\newcommand{\x}{\xi}
%\newcommand{\y}{\sigma}
\newcommand{\z}{z}
\newcommand{\x}{x}
\newcommand{\y}{y}
\newcommand{\zz}{\zeta}
\newcommand{\xx}{\xi}
\newcommand{\yy}{\sigma}
%\newcommand{\k}{k}
\newcommand{\kk}{\kappa}

\newcommand{\zmap}{f}
%\newcommand{\zzmap}{\zmap^{-1}}
\newcommand{\zzmap}{\zmap^{\raisebox{.2ex}{$\scriptscriptstyle-1$}}}

%\newcommand{\ww}{w}
%\renewcommand{\w}{\ww^\mr{P}}
\newcommand{\ww}{\omega}
\renewcommand{\w}{w}

\newcommand{\Hz}{H}
\newcommand{\Hzz}{\tilde H}

\newcommand{\hzz}{\eta}
\newcommand{\hz}{h}
%\newcommand{\w}{\varpi}
\newcommand{\dd}[2]{\frac{\mr d #1}{\mr d #2}}

\newcommand{\Lsin}{\mc S}
\newcommand{\Lcos}{\mc C}
\newcommand{\FF}{\mc F}
\newcommand{\surf}{\eta}
\newcommand{\xS}{\x\S}
\newcommand{\tf}{\tilde \zmap}

 \newcommand{\phib}{\Phi}
 \newcommand{\wb}{W}

\DeclareMathOperator{\sech}{sech}
\DeclareMathOperator{\csch}{csch}
\begin{document}
\title{Conformal mapping in the The Higher Order Spectral method}
\author{Andreas H. Akselsen}
\date{\today}
\maketitle

%\tableofcontents

\section{Aim}
This document presents a conformal mapping approach to evaluating potential function derivatives in two dimensions at a modulated free surface.
We have largely followed \citet{chalikov2005modeling}, hereafter referred to as \refCS{}, in our derivation. 
The problem is encountered within free-surface hydrodynamics whence boundary conditions require a vertical velocity component $v=\phi_y$ evaluated at the free surface $(\x,\surf(\x))$ itself.
A standard approach for such an evaluation, e.g.\ adopted in numerical higher order spectral methods, is to related the free surface $\y=\surf$ to the horizontal line $\y=0$ through Taylor expansions.
The downside of such an approach is that the Taylor series convergence is limited \citep{west1981deep} and that it requires evaluation of a number of functional derivatives.
The number is proportional to the square of the expansion order, each requiring a pair of Fourier transformations.
\\

As will be shown, the conformal mapping approach presented here provides and explicit expression for the surface velocities for a given surface potential.
The expression does not entail any series expansions and is not subject to convergence limitations.
What's more, it requires only six Fourier transformations.

\section{Derivation}
\subsection*{Mapping}
We map the complex coordinate
$\z = \x+\ii\y$
of the physical $\z$-plane to a rectangular plane $\zz=\xx+\ii\yy$ using the conformal transformation function
\[\z\mapsfrom \zmap(\zz).\]
We wish this function to map the free surface at $\z=\x+\ii\hz(\x,t)$ to the ordinate $\zz=\xx$ of the $\zz$-plane while maintaining orthogonality to a flat bottom at $\y=-\Hz$.
For this purpose we introduce the following two transformation kernels:
\newcommand{\fun}{\mu}
\begin{subequations}
\begin{align}
\Lcos[\fun](\zz) &= \sum_{j=-M}^M \FF_j(\fun) \frac{\ee^{\ii \kk_j(\zz+\ii \Hzz)}}{\cosh(\kk_j\Hzz)},% = \sum_{j=-M}^M \FF_j(\fun) \frac{2\ee^{\ii \kk_j \zz}}{\ee^{2\kk_j\Hzz}+1}
\label{eq:Lcos}\\
\Lsin[\fun](\zz) &= \sum_{j=-M}^M \FF_j(\fun) \frac{-\ee^{\ii \kk_j(\zz+\ii \Hzz)}}{\sinh(\kk_j\Hzz)-\delta_j}.% = \sum_{j=-M}^M \FF_j(\fun) \frac{-2\ee^{\ii \kk_j \zz}}{\ee^{2\kk_j\Hzz}-1-2\delta_j}.
\label{eq:Lsin}%
 \end{align}%
\label{eq:L}%
\end{subequations}%
$\FF_j(\fun)$ is here the Fourier transform such that $\fun=\sum_{j=-\infty}^\infty \FF_j(\fun)\ee^{\ii \kk_j \xx}$, and $\delta_j=1$ for $j=0$ and zero otherwise. 
Wavenumbers are $\kk_j=2\pi j/L$, $L$ being the domain length in the $\zeta$-plane.
Assuming $\fun(\xx)$ is real, $\Lcos$ and $\Lsin $ have the following properties:
\begin{subequations}
\begin{align}
\Lcos[\fun](\xx) &= \fun(\xx)  -2\,\ii \sum_{j=1}^M \Im\Big[\FF_j(\fun) \ee^{\ii \kk_j\zz}\Big]\tanh\kk_j\Hzz,\label{eq:LProps:C0}\\
\Lsin[\fun](\xx) &= \fun(\xx)  -2\,\ii \sum_{j=1}^M \Im\Big[\FF_j(\fun) \ee^{\ii \kk_j\zz}\Big]\coth\kk_j\Hzz;\label{eq:LProps:S0}\\
\Lcos[\fun](\xx-\ii\Hzz) &= \FF_0(\fun)  +2\,    \sum_{j=1}^M \Re\Big[\FF_j(\fun) \ee^{\ii \kk_j\zz}\Big]\sech\kk_j\Hzz,\label{eq:LProps:CH}\\
\Lsin[\fun](\xx-\ii\Hzz) &= \FF_0(\fun)  -2\,\ii \sum_{j=1}^M \Im\Big[\FF_j(\fun) \ee^{\ii \kk_j\zz}\Big]\csch\kk_j\Hzz;\label{eq:LProps:SH}\\
\lim_{\Hzz\to\infty}\Lcos[\fun](\zz) &=\lim_{\Hzz\to\infty}\Lsin[\fun](\zz)= \FF_0(\fun) + 2 \sum_{j=-M}^{-1} \FF_j(\fun) \ee^{\ii \kk_j \zz}.\label{eq:LProps:lim}
\end{align}%
\label{eq:LProps}%
\end{subequations}%
That is to say, the real part of both kernels equal the input function itself at the surface line $\yy=0$, and $\Lcos$ is pure real while $\Lsin$ is pure imaginary at the bed line $\yy = -\Hzz$, save for a constant.

We require an undulating surface and a flat bed in the physical plane 
 %the property \eqref{eq:LProps:CH} at the bed for our conformal map 
and so we choose the map
\begin{equation}
\z\mapsfrom \zmap(\zz,t) = \zz+\ii\Lsin[\hzz](\zz).
\label{eq:zmap}
\end{equation}
The free surface thus obtained in the physical plane is
\[\x+\ii\hz(\x,t)\mapsfrom \zmap(\xx)=\xS(\xx,t)+\ii\hzz(\xx,t).\]
Note the stretching of the $\x$ coordinate which means that 
\begin{equation}
\hzz(\xx,t) = \hz[\xS(\xx,t),t]
\label{eq:hzz}
\end{equation}
 with 
\[\xS(\xx,t)=\x_0(t)+\xx+2\sum_{j=1}^M \Im\Big\{\FF_j[\hzz(t)] \ee^{\ii \kk_j\xx}\Big\}\coth\kk_j\Hzz.\] 
Property \eqref{eq:LProps:SH} further reveals that $\Hzz = \Hz+\langle\eta\rangle$.

An example for a single $\zz$-plane mode is shown in \autoref{fig:conformalH}.
Note that a monochromatic wave in the $\zz$-plane is not monochromatic in the physical plane.
\\

Similar to the surface elevation, we also need to map the surface potential onto the free surface.
We introduce the complex surface potentials  
\begin{subequations}
\begin{gather}
\w(\zz,t)=\phi(\x,\y,t)+\ii\psi(\x,\y,t),
\qquad
\ww(\zz,t)=\varphi(\xx,\yy,t)+\ii\tilde\psi(\xx,\yy,t);
\\
\ww(\zz,t)=\w[\zmap(\zz,t),t] \label{eq:ww:match}
\end{gather}%
\end{subequations}%
in the two planes.
%$\ww(\zz,t)$ in the $\zz$ plan, whose 
%The real part is the potential function and imaginary part the stream function. 
Accordingly, 
$\Re \ww(\xx,t)=\varphi(\xx,0,t)=\varphi\S(\xx,t)=\phi[\xS(\xx,t),\hzz(\xx,t),t]$
 and $\Im \ww(\xx-\ii \Hzz,t)=\mr{const}$, and so
\begin{equation}
\ww(\zz,t) = \Lcos[\varphi\S](\zz).
\label{eq:ww}
\end{equation}
Velocities are most easily found by differentiating the complex potential; $\w_\z = \phi_\x-\ii\phi_\y$, $\ww_\zz = \varphi_\xx-\ii\varphi_\yy$.

\begin{figure}[h!ptb]%
\centering
\includegraphics[width=.5\columnwidth]{./conformalH.pdf}%
\caption{The conformal mapping of a single mode---a wave which is monochromatic wave in the $\zz$-plane.}%
\label{fig:conformalH}%
\end{figure}

\subsection*{Boundary value problem}
The map $\zmap(\zz,t)$ cannot be inverted explicitly. 
This pushes us to restate the boundary value problem in terms of $\zz$-variables such that the entire simulation can be carried out in the $\zz$-plane.

In the $\z$-plane, the boundary value problem reads 
\begin{align*}
h_t &=- \phi_x h_x+\phi_y,\\
%\phi_t &=- \frac12\rbr{\phi_x^2+\phi_y^2}-gh
\phi_t &=- \frac12|w_\z|^2-gh.
\end{align*}
%in physical space.
Re-stated in $\zz$-plane variables%
\footnote{using \eqref{eq:hzz}, \eqref{eq:ww:match} and the Cauchy--Riemann conditions $x_\xx = y_\yy$, $x_\yy=-y_\xx$.}
 it reads%
 \footnote{The dynamic condition can further be re-written
$\varphi\S_t = |\zmap_\zz|^{-2} (\xS_t\xS_\xx + \hzz_t\hzz_\xx)\varphi_\zz - \frac12 |\zmap_\zz|^{-2} \rbr{\varphi_\xx^2-\varphi_\yy^2}  - g \hzz$ as per \refCS{}.} 
\begin{subequations}
\begin{align}
\hzz_t\xS_\xx - \xS_t\hzz_\xx &= \phi_\y\xS_\xx-\phi_\x\hzz_\xx = \ldots = \varphi_\yy, \label{eq:BC_zz:eta} \\
\varphi\S_t &= \Re(\ww_\zz f_t/ f_\zz)  - \frac12 |\ww_\zz/ f_\zz|^2  - g \hzz.\label{eq:BC_zz:phi}
\end{align}%
\label{eq:BC_zz}%
\end{subequations}%
%$|\zmap_\zz|^{2}$ being the transformation Jacobian.%
A final obstacle remains---the left-hand side of \eqref{eq:BC_zz:eta} should be written in terms of a single time derivative.
\refCS{} does this by introducing a modified map
\begin{equation}
\tf(\zz,t)\equiv\zmap_t/\zmap_\zz = |\zmap_\zz|^{-2}[\x_t\x_\xx + \y_t\y_\xx +\ii(\y_t\x_\xx - \x_t\y_\xx ) ]
\label{eq:tf}
\end{equation}
The kinematic condition \eqref{eq:BC_zz:eta} can now be written
\begin{equation}
\Im \tf = \big(|\zmap_\zz|^{-2}\varphi_\yy\big)_{\yy=0} \equiv b(\xx,t).
\label{eq:BC_zz2:eta}%
\end{equation}%
The additional condition that $\y_t=\Im(\tf\zmap_\zz)=0$ at $\yy=-\Hzz$ leads us to the mapping
\begin{equation}
\tf(\zz,t) = \Lsin[b](\zz) + \tilde x_{0}(t).
\label{eq:tfMap}
\end{equation}
(The mean of $b$ leads to a uniform vertical shift of the domain which is of no consequence. We therefore ignore the $j=0$ component when evaluating \eqref{eq:Lsin}.)
The real uniform component $\tilde x_{0}(t)$ is chosen to avoid a gradual horizontal drift of the domain; 
by solving 
\[\int x_t\,\mr d\xx=\int\Re(\tf \zmap_\zz) \,\mr d \xx = 0\]
one finds
\begin{equation}
%\tilde x_{0}(t) = -\sum_{j=-M}^M \frac{\kk_j \Im[\FF_j(|\zmap_\zz|^{-2}\varphi_\yy)\FF_j(\eta)^*]}{\sinh^2\kk_j\Hzz + \delta_j}.
\tilde x_{0}(t) = -2\sum_{j=1}^M \frac{\kk_j \Im[\FF_j(b)\FF_j(\eta)^*]}{\sinh^2\kk_j\Hzz}.
\label{eq:tx0}
\end{equation}
The system now reads
\begin{subequations}
\begin{align}
\eta_t&= \Im(\tf \zmap_\zz), \\
%\varphi\S_t &= \Re(\ww_\zz \tf)  - \frac12|\ww_\zz/\zmap_\zz|^2  - g \hzz,
\varphi\S_t &= \Re(\tf)\varphi_\xx  - \frac12\Re \big[(\ww_\zz/\zmap_\zz)^2\big]  - g \hzz,
%\varphi\S_t &= \Re(\tf)\varphi_\xx  - \frac12 |\zmap_\zz|^{-2} (\varphi_\xx^2-\varphi_\yy^2)  - g \hzz,
\end{align}%
\label{eq:BC_zz2}%
\end{subequations}%
the former equation simply being the definition \eqref{eq:tf} and the latter being \eqref{eq:BC_zz:phi} modified with \eqref{eq:BC_zz2:eta} (which seems to improve numerical stability).
Variables are of course to be evaluated at $\yy=0$.
A MATLAB code implementation is for illustration presented in \lref{Listing}{list:basic} for demonstration.



%\lstinputlisting[caption=Example: Loading a time series.,label=list:load_ts]{\codePath/load_ts.m}
\begin{lstlisting}[basicstyle=\ttfamily\small,label=list:basic,caption=Core implementation in MATLAB (without anti-aliasing)]
	S = -2./(exp(2*k*H)-1); S(1) = 1;
	C = 2./(exp(2*k*H)+1);
	FFTeta = fft(eta);
	f_xi = 1 - ifft(k.*FFTeta.*S);
	JInv = abs(f_xi).^(-2);
	w_xi = ifft(1i.*kx.*fft(phiS).*C);
	FFTb = fft(-JInv.*imag(w_xi));
	tf0 =  1i*sum(imag(FFTb.*conj(FFTeta)).*k./(N*sinh(k*H)+(k==0)).^2);
	tf  =  1i*tf0 + 1i*ifft(FFTb.*S.*(k~=0));
	eta_t  = imag(tf.*f_xi);
	phiS_t = real(tf).*real(w_xi) - .5*JInv.*real(w_xi.^2) - g*eta;
\end{lstlisting}

\subsection*{Stabilising and anti-aliasing}
Anti-aliasing is in transform methods \citep{orszag1970transform} normally achieved through use of the so-called $\nu+1$-\textit{half} rule where any physical space variable is zero-padded up to $(\nu+1)/2$ its original size, $\nu$ being the order of nonlinearity in which the variable takes part \citep{bonnefoy2010}.
The order of nonlinearity in the Chalikov-type scheme is infinite due to the appearance of the inverse Jacobian. \refCS{} do however report that the their solutions are insensitive to increases of $\nu>3$ the the $\nu+1$-\textit{half} rule.
Zero-padding has in our code been implemented as a simulation option.
We have in general observed very little sensitivity to zero-padding in our simulations, and we often skip zero-padding for numerical efficiency in preliminary simulations.
\\


More critical for numerical stability is the use of numerical damping.
Following \citet{chalikov2005modeling}, we extend the time derivative computation with
\begin{subequations}
\begin{align}
\eta_t &\coloneqq \FF^{-1}\{\FF_j(\eta_t) - \mu_j \FF_j(\eta) \}\\
\varphi\S_t &\coloneqq  \FF^{-1}\{\FF_j(\varphi\S_t) - \mu_j \FF_j(\varphi\S) \}
\end{align}%
\label{eq:damping}%
\end{subequations}
where 
\begin{equation}
\mu_j = \begin{cases}
r\times 2\pi M \sqrt{g/L}\rbr{\frac{|j|-M\_d}{M-M\_d}}^2,& j>M\_d\\
0 & \text{otherwise.}
\end{cases}
        %mu = mu = k_cut*sqrt(2*pi*g*kx(2))*N/2 * (2*k/kmax-1).^2.*(k>kmax/2);
\label{eq:damping_nu}
\end{equation}
$-M\leq j \leq M$ are here the spectral modes before zero-padding and $M\_d$ is the first mode with active damping.
The damping is gradually increasing form $M\_d$ into the high-wavenumber end of the spectral domain.
As in \citet{chalikov2005modeling}, $r=0.25$ is chosen.


\subsection*{Initial conditions}
Several types of initial conditions have been tested. 
One usually successful method is to run first the normal HOS scheme with nonlinearity ramping for an initial period, thus yielding initial conditions.
A better method if considering regular waves is to extract the initial conditions form a stationary wave solution generated with the SSGW \citep{clamond2018accurate} method;
the surface potential is easily obtained by integrating
\[\w = \int_{\Omega\S}\! \w_\z \,\mr d z\] 
along the surface line $z\in\Omega\S$ from such a solution. 
Both $\Omega\S$ and  $\w_\z$ are outputs of the SSGW code available online. 
The Chalikov-type HOS scheme described herein can stably maintain this wave solution for some time (stability limitations not tested yet), as shown in \autoref{fig:SSGWinit:contour} and \ref{fig:SSGWinit:eta} for a reasonably shallow wave. 


\begin{figure}[h!ptb]%
\centering
\includegraphics[width=\columnwidth]{../HOS_old/conformalHOS/figures/imagedecayingConformalka0p2_M5_h1p00_Nw1_dt0p25T_nx512.pdf}%
\caption{Wave initiated with the SSGW solution. Steepness $k\bar H/2=0.2$, depth $H=0.1\lambda$.}%
\label{fig:SSGWinit:contour}%
\end{figure}
\begin{figure}[h!ptb]%
\centering
\includegraphics[width=\columnwidth]{../HOS_old/conformalHOS/figures/decayingConformalka0p2_M5_h1p00_Nw1_dt0p25T_nx512.pdf}%
\caption{Wave initiated with the SSGW solution. Steepness $k\bar H/2=0.2$, depth $H=0.1\lambda$.}%
\label{fig:SSGWinit:eta}%
\end{figure}



\subsection*{Normalization and implementation}
We have here presented model and method in dimensional units. 
The practical normalization used by \refCS{}, also adopted in our implementation, is to normalize with gravity and the length scale $L/2\pi$, $L$ being the domain length. 
This renders wavenumbers as integers; $k_j = j$.

Another difference between \refCS{}'s method and traditional HOS implementations is that there is no need to perform the time integration in physical space; instead of $y^{n+1}=y_t^{n}\Delta t$ we can write $\hat y^{n+1}=\hat y_t^{n}\Delta t$ where $\hat y = \FF(y)$.
A benefit with this is that we save ourself an FFT/iFFT operation pair on $\eta$ and $\phi\S$ at each iteration. 
A possible drawback if automated time stepping is used is that the ODE solved can select unnecessarily small time steps.
Both types of input/output are supported in our implementation, as well as an option to choose between a fixed time step Runge--Kutta method adopt MATLAB's \texttt{ODExx} adaptive time step ODE solvers. 


\subsection*{Extension with a background flow field}
Similar to [other memo about current+HOS], we can include a background flow field 
\[
%\phi(\z,t)\to\phi(\z,t)+\phib(\z,t), \qquad
\w(\z,t)\to\w(\z,t)+\wb(\z,t)\]
where $\wb(\z,t)$ is a pre-determined function representing the presence of things like wavemakers or currents.
Repeating the above derivation now yields
\begin{subequations}
\begin{align}
\Im \tf &= |\zmap_\zz|^{-2}\varphi_\yy  - \Im(\wb_\z /\zmap_\zz^*),\label{eq:BCb:kin}\\
\varphi\S_t &= -\Re(\wb_t) + \Re(\ww_\zz\tf) - \frac12\big|\ww_\zz/\zmap_\zz+\wb_\z\big|^2  - g \hzz
\intertext{or, reducing with \eqref{eq:BCb:kin}}
\varphi\S_t &= \Re(\tf)\varphi_\xx  - \frac12\Re \big[(\ww_\zz/\zmap_\zz)^2\big]  - g \hzz
-\Re[\wb_t+\wb_\z\varphi_\xx/\zmap_\zz^*]
\end{align}%
\label{eq:BCb}%
\end{subequations}%
with $\wb_t$ and $\wb_\zz$ evaluated at $z=\zmap(\xx,t)$ and $\zz$-variables at $\yy=0$.
%
This expression has been validated against the previously derived model for the normal HOS method with Taylor expansion, 
\begin{subequations}
\begin{align}
\eta_t &=   - \nabla\eta\cdot\nabla\phi\S  - \nabla\eta\cdot\nabla\phib + \phib_y  + \big(1+|\nabla\eta|^2\big)\phi_y, 
\\
\phi\S_t &= -\Phi_t - \frac12\big|\nabla(\phi\S +\phib)\big|^2 + \frac12\big(1+|\nabla\eta|^2\big)\phi_y^2 - g\eta,
\end{align}%
\label{eq:BCb_normalHos}%
\end{subequations}%
see \autoref{fig:vortex} and related memo.


\begin{figure}[h!ptb]%
\centering
\subfloat[Vortex current]{\includegraphics[width=.5\columnwidth]{../conformalHOS_current/figures/curr_vortex.pdf}}%
\\
\subfloat[Benchmark; \eqref{eq:BCb} vs.\ \eqref{eq:BCb_normalHos}.]{\includegraphics[width=\columnwidth]{../conformalHOS_current/figures/vortex.pdf}}%
\caption{Simulation of wave train over a stationary vortex with normal HOS and Chalikov variant. $ka=0.2$. SSGW initial conditions.}%
\label{fig:vortex}%
\end{figure}




\subsection*{Breaking wave benchmark}

As a more flashy benchmark, we have attempted to reproduce \refCS{}'s solution of a breaking wave. (Run 1 in \refCS{}, figure 2 therein.)
The authors do not specify very precisely what their initial conditions are, other than to say it is a monochromatic wave, in physical space judging by their figure.
Our simulation did not remain stable up to the point of breaking with the parametric description given by \refCS{}, where $M\_d$, equation \eqref{eq:damping_nu}, is set to $M/2$.
Instability was caused by the range of undampened modes $|j|<M\_d$ where $|j|$ is still much greater than one due to the high resolution of this simulation. 
The rate of instability is sensitive to the initial conditions, to which the difference in results may be attributed. 
We can however stabilize our solution by dampening a wider range of modes; setting $M\_d=0$ yields the result shown in \autoref{fig:breaker:eta}.
The numerical damper is now active on all modes, but with little influence on the dominant wavenumbers. 
The bulky shape attained before the wave front is established is likely due the the initial conditions and not the damping.
We then reach the stage where the surface elevation becomes multi-valued in physical space, after which wave breaking is sure to ensue. 
The numerical damping is however strong at these wavenumbers, and the wave never actually breaks.  
Instead, the plunger pulls back, and the simulation survives to continue. 
To some extent, the numerical damping may be regarded as a crude wave breaking modes, although we stress that it is not founded in any physical or empirical evidence. 

A demonstration so the full conformal map solution in a stage of plunging is given in \autoref{fig:breaker:contour}.
\\
\begin{figure}[h!ptb]%
\centering
\includegraphics[width=.6\columnwidth]{../conformalHOS_current/figures/rounded2_Chalikov_ka0p5_M3072_hInf_Nw1_dt0p177T_nx6145_pad1_.pdf}%
\caption{Wave initiated as a steep monochromatic wave $k\bar a=0.5$, cf. \refCS{}'s run 1. Infinite depth, $M=3072$.}%
\label{fig:breaker:eta}%
\end{figure}
\begin{figure}[h!ptb]%
\centering
\includegraphics[width=.75\columnwidth]{../conformalHOS_current/figures/rounded2_contour.pdf}%
\caption{Contour solution of fourth frame ($t=3.38$\,s) of \autoref{fig:breaker:eta}.}%
\label{fig:breaker:contour}%
\end{figure}

The high resolution here used is not necessary for attaining the plunger crest shape, but the wide range of damping is.
\autoref{fig:breaker:etaCoarse} shows the same simulation as in \autoref{fig:breaker:eta}, but with $M=512$.
This simulation is also stable and quite similar up to the point of breaking. We notice however that phase velocity of the two simulations differ during the `breaking event'.
\begin{figure}[h!ptb]%
\centering
\includegraphics[width=.6\columnwidth]{../conformalHOS_current/figures/Chalikov_ka0p5_M512_hInf_Nw1_dt0p177T_nx1025_pad1.pdf}%
\caption{\autoref{fig:breaker:eta}, but with $M=512$.}%
\label{fig:breaker:etaCoarse}%
\end{figure}




\bibliographystyle{plainnat} % abbrvnat,plainnat,unsrtnat
\bibliography{bib,sintef_bib} %You need a file 'literature.bib' for this.


\end{document}
