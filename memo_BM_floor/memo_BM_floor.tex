%\documentclass[a4paper,12pt]{article}
\documentclass[internal]{sintefmemo}
%\documentclass{amsart}
%\usepackage[a4paper, total={17cm, 25cm}]{geometry}

%\usepackage[utf8]{inputenc}
\usepackage[english]{babel}
%\usepackage{amsmath,bm,amsfonts,amssymb}
\usepackage{xcolor}
\usepackage{graphicx}
\usepackage{graphbox} % allows includegraphics[align=c]
\usepackage[round]{natbib}
\usepackage{mathtools}
\usepackage[font=normalsize,margin=2mm]{subfig}
\usepackage{float}
\usepackage{hyperref}
\usepackage{xfrac}
%\usepackage{cprotect}%for \verb in captions
%\usepackage{enumerate}
\usepackage{enumitem}
\hypersetup{colorlinks=true,
			linkcolor=blue,
			filecolor=blue,
			urlcolor=blue,
			citecolor=blue}
\newcommand{\mr}{\mathrm}
\newcommand{\mc}{\mathcal}
\let\SSS\S
\renewcommand{\S}{^\mr{S}}
\newcommand{\ii}{\mr{i}\,}
\newcommand{\ee}{\mr{e}}
\let\underscore\_
\renewcommand{\_}[1]{_\mr{#1}}
\newcommand{\oo}[1]{^{(#1)}}
\let\Re\relax
\let\Im\relax
\DeclareMathOperator\Re{Re}
\DeclareMathOperator\Im{Im}
\newcommand{\w}{w}
\newcommand{\bU}{\bm U}
\newcommand{\h}{\hat}
\newcommand{\br}[3]{\left#1#2\right#3}
\newcommand{\rbr}[1]{\left(#1\right)}
\newcommand{\sbr}[1]{\left[#1\right]}
\newcommand{\cbr}[1]{\left\{#1\right\}}
%\newcommand{\bU}{(\nabla\Phi)_{z=\eta}}


\newcommand{\z}{z}
\newcommand{\x}{x}
\newcommand{\y}{y}
\newcommand{\zz}{\zeta}
\newcommand{\xx}{\xi}
\newcommand{\yy}{\sigma}
\newcommand{\kk}{\kappa}

\newcommand{\zmap}{f}
%\newcommand{\zzmap}{\zmap^{-1}}
\newcommand{\zzmap}{\zmap^{\raisebox{.2ex}{$\scriptscriptstyle-1$}}}

%\newcommand{\ww}{w}
%\renewcommand{\w}{\ww^\mr{P}}
\newcommand{\ww}{\omega}
\renewcommand{\w}{w}

\newcommand{\surf}{\eta}
%\newcommand{\w}{\varpi}
\newcommand{\dd}{\mr d}
\newcommand{\ddfrac}[2]{\frac{\dd #1}{\dd #2}}

\title{Assessment of the impact of depth transitions on wave quality}
\author{Andreas H. Akselsen}
\project{302006355-3 (OSC pre-project)}
\date{\today}
\recipient[information]{SINTEF employees}
%\recipient[information,agreed]{Statsbygg}

\begin{document}


%\maketitle
\frontmatter

%\tableofcontents

\section{Introduction}




\section{Method}
\label{sec:method}
\subsection{Boundary equations for the mapped plane}
%Option \ref{it:chosen} seems the most promising of these options.
The boundary value problem 
\begin{align*}
h_t + \phi_x h_x-\phi_y&=0,\\
\phi_t + \frac12\rbr{\phi_x^2+\phi_y^2}+gh&=0
\end{align*}
can be re-stated in terms of the surface coordinate
\[
\x+\ii h(\x,t) \mapsfrom f[\xx+\ii \eta(\xx,t)].
\]
Introducing
\[\varphi(\xx,\yy,t)=\Re\ww(\zz,t) \]
and the boundary conditions expressed in terms of the surface potential 
\[
\varphi\S(\xx,t)=\varphi[\xx+\ii\eta(\xx,t),t]
\]
we get 
\begin{subequations}
\begin{align}
\eta_t &= |\zmap'|^{-2} \sbr{-   \varphi\S_\xx\eta_\xx + \rbr{1+\eta_\xx^2} \varphi_\yy},\\
\varphi\S_t  &= |\zmap'|^{-2}\sbr{ - \frac12  \rbr{\varphi\S_\xx}^2 + \frac12 \rbr{1+\eta_\xx^2} \varphi_\yy^2 }  - g h,
\end{align}%
\end{subequations}%
after some derivation.
Here $|\zmap'|^{2}$ is equivalent to the transformation Jacobian and $\varphi_\yy$ is evaluated at $\zz=\xx+\ii\eta$.
$h=\Im \zmap(\xx+\ii\eta)$ is readily available. 

\subsection{Mapping}
%The Schwartz--Christoffel transform is a powerful approach for imposing straight angles in a mapping.
A map of a depth transition from $H\_s$ to $H\_d$ with sloping angle $\theta$ is obtained by integrating
\begin{equation}
%\zmap'(\zz) = \frac{H\_s}{\pi \tau_\theta}, \qquad \tau_\theta = \rbr{\frac{\ee^\zz+ (H\_d/H\_s)^{\pi/\theta}}{\ee^\zz+1}}^{\theta/\pi}. 
\zmap'(\zz) = \frac{H\_s}{\pi} \rbr{\frac{\ee^\zz+1}{\ee^\zz+ (H\_d/H\_s)^{\pi/\theta}}}^{\theta/\pi}. 
\label{eq:SCnumStep}
\end{equation}
Any integration path can be used since the map is conformal; we integrate from the top of the domain down to avoid spreading numerical imprecision form the singularities located at $\zeta=-\ii\pi$ and $\zeta=2\ln c-\ii\pi$ into the domain.
An example is shown on \autoref{fig:SCnumStep45deg} for a 45\textdegree down-step.

\begin{figure}[H]%
\centering
\subfloat[$\zz$-plane]{\includegraphics[width=.47\columnwidth]{../conformalMapping/conformalBathymetry/SCnumStep45deg_zz.pdf}}%
\hfill
\subfloat[$\z$-plane]{\includegraphics[width=.47\columnwidth]{../conformalMapping/conformalBathymetry/SCnumStep45deg_z.pdf}}%
\caption{Numerical integration of \eqref{eq:SCnumStep} with $\theta=\pi/4$, $H\_s = 0.5$ and $H\_d = 1.0$.}%
\label{fig:SCnumStep45deg}%
\end{figure}


We can extend this technique to combine several alterations in bathymetry level by adopting
\begin{equation}
\zmap'(\zz) = K \prod_{j=1}^{N_\angle} \rbr{\frac{\exp(\zz-\xi^\angle_j)+1}{\exp(\zz-\xx^\angle_j) + (H_{j+1}/H_j)^{\pi/\theta_j}}}^{\theta_j/\pi}
\label{eq:SCnumMultiStep}
\end{equation}
and fixing the constant $K$ after integration (re-scaling). $N_\angle$ is here the number of depth transitions, $\{H_j\}$ the set of depths, $\{\theta_j\}$ ($>0$) the slopes of the individual transitions and $\{\xi^\angle_j\}$ the horizontal position around which it occurs in the $\zz$-plane. 
%The singularities of \eqref{eq:SCnumMultiStep} are found at $\zz=\xx^\angle_j-\ii \pi$ and $\zz=\xx^\angle_j+\pi/\theta_j\ln(H_{j+1}/H_j)-\ii \pi$.
%\autoref{fig:SCnumStep2x45deg} shows an example similar to the plateau seen earlier. 
%A rougher bathymetry is displayed in \autoref{fig:SCnumMultiStep}. 
%The deformation of some of the sharp corners in this image is due to inaccuracy in the numeral integration near the singularity and is of no consequence for simulation usage.
%Corners can of course be smoothened by defining a higher bathymetry level $\yy=\tilde H < \pi$ in the $\zz$-plane, as shown in \autoref{fig:SCnumMultiStepSmooth}.
%
%\begin{figure}[H]%
%\centering
%\subfloat[$\zz$-plane]{\includegraphics[width=.47\columnwidth]{../conformalMapping/conformalBathymetry/SCnumStep2x45deg_zz.pdf}}%
%\hfill
%\subfloat[$\z$-plane]{\includegraphics[width=.47\columnwidth]{../conformalMapping/conformalBathymetry/SCnumStep2x45deg_z.pdf}}%
%\caption{Numerical integration of \eqref{eq:SCnumMultiStep} with 
%%$\theta_1=\theta_2=\pi/4$; $H_1 = 1.0$, $H_2 = 0.5$, $H_3 = 1.0$ and $\xi^\angle_1=0$, $\xi^\angle_1=5$.
%$\theta_j\in\{\pi/4,\pi/4\}$, $H_j \in\{1.0,0.5,1.0\}$ and $\xx^\angle_j\in\{0,5\}$.
%}%
%\label{fig:SCnumStep2x45deg}%
%\end{figure}
%
%\begin{figure}[H]%
%\centering
%\subfloat[$\zz$-plane]{\includegraphics[width=.47\columnwidth]{../conformalMapping/conformalBathymetry/SCnumStepMulti_zz.pdf}}%
%\hfill
%\subfloat[$\z$-plane]{\includegraphics[width=.47\columnwidth]{../conformalMapping/conformalBathymetry/SCnumStepMulti_z.pdf}}%
%\caption{Numerical integration of \eqref{eq:SCnumMultiStep} with 
%$\theta_j\in\{\pi/4,5\pi/2,5\pi/2\}$, $H_j \in\{0.5,1.5,0.75,1.5\}$ and $\xx^\angle_j\in\{0,8,14\}$.
%}%
%\label{fig:SCnumMultiStep}%
%\end{figure}
%
%\begin{figure}[H]%
%\centering
%\includegraphics[width=.47\columnwidth]{../conformalMapping/conformalBathymetry/SCnumStepMultiSmooth_z.pdf}%
%\caption{As \autoref{fig:SCnumMultiStep}, instead choosing bethymetry at $\yy = 0.1-\pi$.
%}%
%\label{fig:SCnumMultiStepSmooth}%
%\end{figure}




\section{Preliminary results}
\label{sec:results}

\subsection{Wave packets propagating over a step transition in depth}
\label{sec:results:step}
 %\documentclass[a4paper,12pt]{article}
%%\documentclass{amsart}
%\usepackage[a4paper, total={17cm, 25cm}]{geometry}
%\usepackage[utf8]{inputenc}
%\usepackage[english]{babel}
%\usepackage{amsmath,bm,amsfonts,amssymb}
%\usepackage{xcolor}
%\usepackage{graphicx}
%
%\usepackage{graphbox}
%
%\usepackage[round]{natbib}
%\usepackage{mathtools}
%\usepackage[font=normalsize]{subfig}
%\usepackage{float}
%\usepackage{hyperref}
%\usepackage{enumitem}
%\hypersetup{colorlinks=true, 	
			%linkcolor=blue,
			%filecolor=blue,
			%urlcolor=blue,
			%citecolor=blue}
%\newcommand{\mr}{\mathrm}
%\newcommand{\mc}{\mathcal}
%\let\SSS\S
%\renewcommand{\S}{^\mr{S}}
%\newcommand{\ii}{\mr{i}\,}
%\newcommand{\ee}{\mr{e}}
%%\newcommand{\phit}{\psi}
%\newcommand{\phit}{\tilde\phi}
%\newcommand{\br}[3]{\left#1#2\right#3}
%\let\underscore\_
%\renewcommand{\_}[1]{_\mr{#1}}
%\newcommand{\oo}[1]{^{(#1)}}
%\newcommand{\rr}{\bm r}%{x,y}
%\newcommand{\cp}{c\_p}
%\let\Re\relax
%\let\Im\relax
%\DeclareMathOperator\Re{Re}
%\DeclareMathOperator\Im{Im}
%\newcommand{\w}{w}
%\newcommand{\bU}{\bm U}
%\newcommand{\h}{\hat}
%\newcommand{\rbr}[1]{\left(#1\right)}
%\newcommand{\sbr}[1]{\left[#1\right]}
%\newcommand{\cbr}[1]{\left\{#1\right\}}
%\newcommand{\z}{z}
%\newcommand{\x}{x}
%\newcommand{\y}{y}
%\newcommand{\zz}{\zeta}
%\newcommand{\xx}{\xi}
%\newcommand{\yy}{\sigma}
%%\newcommand{\k}{k}
%\newcommand{\kk}{\kappa}
%
%\newcommand{\zmap}{f}
%%\newcommand{\zzmap}{\zmap^{-1}}
%\newcommand{\zzmap}{\zmap^{\raisebox{.2ex}{$\scriptscriptstyle-1$}}}
%
%\newcommand{\ww}{\omega}
%\renewcommand{\w}{w}
%\newcommand{\surf}{\eta}
%\newcommand{\dd}[2]{\frac{\mr d #1}{\mr d #2}}
%
%\begin{document}
%

Results shown here pertain to HOS simulations using the algebraic conformal maps described in \autoref{sec:SC} and the approach of \autoref{sec:zz-planeApproach}.
Simulation over the step shown in \autoref{fig:res:map_logstrip} (equation \ref{eq:map_logstrip}) is presented in \autoref{fig:res:logstrip1} to \ref{fig:res:logstrip3}.
These simulations take about five minutes each to run on a piece-of-shit laptop.

\begin{figure}[h!ptb]%
\centering
\subfloat[Full domain]{\includegraphics[width=.5\columnwidth,align=c]{../conformalMapping/HOS/figures/map/map_logstrip_SSGW_ka0p05_H1p00_0p50_Nw60.pdf}}%
\subfloat[Corner, 1--to--1.]{\includegraphics[width=.5\columnwidth,align=c]{../conformalMapping/HOS/figures/map/mapZoom_logstrip_SSGW_ka0p05_H1p00_0p50_Nw60.pdf}}
\caption{Conformal map, single step; $H\_d = 1.0$, $H\_s = 0.5$}%
\label{fig:res:map_logstrip}%
\end{figure}

\begin{figure}[h!ptb]%
\centering
\includegraphics[width=1\columnwidth]{../conformalMapping/HOS/figures/logstrip_SSGW_ka0p05_M5_H1p00_0p75_Nw60_dt5T_nx3840_pad0_ikCutInf_Md0p5_r0p25.pdf}%
\caption{Surface elevation, $(ka)\_L = 0.05$, $(kH)\_L = 1.00$ ($H\_d=1.0$\,m corresponds to $T\approx2.30$\,s); single step, $H\_s/H\_d = 0.75$. Dashed line indicates $\x$-location of step transition.}%
\label{fig:res:logstrip1}%
\end{figure}
\begin{figure}[h!ptb]%
\centering 
\includegraphics[width=1\columnwidth]{../conformalMapping/HOS/figures/logstrip_SSGW_ka0p05_M5_H1p00_0p50_Nw60_dt5T_nx3840_pad0_ikCutInf_Md0p5_r0p25.pdf}%
\caption{Similar to \autoref{fig:res:logstrip1}, but with $H\_s/H\_d = 0.50$.}%
\label{fig:res:logstrip2}%
\end{figure}
\begin{figure}[h!ptb]%
\centering 															  
\includegraphics[width=1\columnwidth]{../conformalMapping/HOS/figures/logstrip_SSGW_ka0p05_M5_H1p00_0p35_Nw60_dt5T_nx3840_pad0_ikCutInf_Md0p5_r0p25.pdf}%
\caption{Similar to \autoref{fig:res:logstrip1}, but with $H\_s/H\_d = 0.35$.}%
\label{fig:res:logstrip3}%
\end{figure}


\begin{figure}[h!ptb]%
\centering
\includegraphics[width=\columnwidth]{../conformalMapping/HOS/linearStepTheory/contourPlot.pdf}%
\caption{Example of linear theory solution by discontinuity matching. ($H\_d = 1.0$, $H\_s = 0.50$, $kH\_d = 1.00$.) Solid black and dashed red lines are values at left and right side of discontinuity $x=0$, respectively.}%
\label{fig:linearReflection:contour}%
\end{figure}

\begin{figure}[h!ptb]%
\centering
\includegraphics[width=.5\columnwidth]{../conformalMapping/HOS/linearStepTheory/R0.pdf}%
\includegraphics[width=.5\columnwidth]{../conformalMapping/HOS/linearStepTheory/R0_k.pdf}%
\caption{Reflection coefficients form linear theory}%
\label{fig:linearReflection:R0}%
\end{figure}



A comparison to linear theory and to the second-order nonlinear theory of \citet{li_2021_step1} is made in \autoref{tab:compareTheory} for the simulations shown in \autoref{fig:res:logstrip1} to \ref{fig:res:logstrip3}.
(See also the memo \citet{AHA_2021_LiTheory} summarizing \citeauthor{li_2021_step1}'s second-order theory in relation to design decisions of SINTEF Ocean Space Center.)
The relative depth is $(kH)\_L=1.00$,  corresponding to $T/\sqrt{H\_d/g}=7.20$.
Both linear reflected and second-order free transmitted wave packet amplitudes correspond reasonably well to observations.
Discrepancies should likely be attributed to wave packet dispersion, wave packet stability and numerical damping, all of which contributes to the observed amplitudes being smaller than the theoretical ones.
Inaccuracies also arise from estimating the initial primary mode amplitude as half the wave height and form a small portion of the initial energy travelling in the negative direction.

Third and possibly forth-order transmitted wave packets are discernible in the shallowest simulation shown in \autoref{fig:res:logstrip3}.
We can also see in the final panel the emergence of what is likely the second-order sub-harmonic transmitted wave packet. 
Note that wave packet stability and also the numerical resolution and damping may affect results somewhat. 
\\


\begin{table}[h!ptb]%
\centering
\begin{tabular}{c|ccc}
$H\_s/H\_d$ & 0.75 & 0.50 & 0.35\\\hline
$R_0$ & 0.054 & 0.135 & 0.208 \\
$R\_{HOS}^{(1)}$ &  0.052 & 0.132 & 0.20\\\hline
$T_{20}$ & 1.93 & 7.57& 18.58 \\
$T^{(2)}\_{HOS}$ & 1.6 & 6.32 & 14.4
\end{tabular}
\caption{
Comparison with theory of simulation results for $(kH)\_L=1.00$ shown in \autoref{fig:res:logstrip1} to \ref{fig:res:logstrip3}.
Reflection coefficients $R_0$ from linear theory (\autoref{fig:linearReflection:contour}, \ref{fig:linearReflection:R0})
and coefficient $T_{20}$ for second-order transmitted free wave packet from the theory by \citet{li_2021_step1}.
Observed linear reflected and second-order free transmitted wave amplitudes are estimated by visual inspection of the fifth and eight plot panel from below, respectively.
$T^{(2)}\_{HOS} = A_{20}/(2A_0^2 \omega/g)$ is adopted for estimating the second-order coefficients from plots, $A_0$ being the incident packet amplitude and $A_{20}$ the observed second-order free transmitted packet amplitude. 
}
\label{tab:compareTheory}
\end{table}



Wave propagation over the plateau configuration (equation \ref{eq:map_double}) is considered in \autoref{fig:res:double}.
We here have linear reflections associated with both steps of the plateau, as well as the excretion of second-order reflected and transmitted packets.
The main spurious wave packets observed across the plateau are the linear reflected packet from the rear step and the second-order free transmitted packet from the front step. 
These will in turn partially reflect when reaching the opposite step, albeit less intensely. 
Also distinct is the second-order sub-harmonic free wave travelling ahead of the main wave group after interacting with the bathymetry. 
Amplitudes of  sub-harmonic packets can also be estimated with the theory of \citet{li_2021_step1}, although we do not do so here. 
Passive absorption of sub-harmonic  (surge-type) waves is inefficient such that these may persist in a wave tank for a long time.

Images also contain some pollution form a wave packet associated with imprecision in the initial conditions. This packet travels in the negative direction through the periodic boundary.
The simulation has been terminated at the point in time when the linear wave reflected at the up-step reaches the down-step via the periodic boundary. 


\begin{figure}[h!ptb]%
\centering
\subfloat[Map]{\includegraphics[width=1\columnwidth]{../conformalMapping/HOS/figures/map/map2_double_SSGW_ka0p05_H1p00_0p50_Nw60.pdf}}\\
\subfloat[Surface elevation]{\includegraphics[width=1\columnwidth]{../conformalMapping/HOS/figures/double_SSGW_ka0p05_M5_H1p00_0p50_Nw60_dt7p5T_nx3840_pad0_ikCutInf_Md0p5_r0p25.pdf}}%
\caption{Double step, $(ka)_0 = 0.05$, $(kH)_0 = 1.00$, $H\_s/H\_d = 0.50$. Plateau length is 35\% of domain length.}%
\label{fig:res:double}%
\end{figure}




%\big|\zeta_{0T,f}^{(22,0)}\big| &= \frac{2\omega_0}{g}|T_{20}| A_0^2.


%\end{document} 

\subsection{Wave packets propagating over a slope transition in depth}
\label{sec:results:slope}
 %\documentclass[a4paper,12pt]{article}
%%\documentclass{amsart}
%\usepackage[a4paper, total={17cm, 25cm}]{geometry}
%\usepackage[utf8]{inputenc}
%\usepackage[english]{babel}
%\usepackage{amsmath,bm,amsfonts,amssymb}
%\usepackage{xcolor}
%\usepackage{graphicx}
%
%\usepackage{graphbox}
%
%\usepackage[round]{natbib}
%\usepackage{mathtools}
%\usepackage[font=normalsize]{subfig}
%\usepackage{float}
%\usepackage{hyperref}
%\usepackage{enumitem}
%\hypersetup{colorlinks=true, 	
			%linkcolor=blue,
			%filecolor=blue,
			%urlcolor=blue,
			%citecolor=blue}
%\newcommand{\mr}{\mathrm}
%\newcommand{\mc}{\mathcal}
%\let\SSS\S
%\renewcommand{\S}{^\mr{S}}
%\newcommand{\ii}{\mr{i}\,}
%\newcommand{\ee}{\mr{e}}
%%\newcommand{\phit}{\psi}
%\newcommand{\phit}{\tilde\phi}
%\newcommand{\br}[3]{\left#1#2\right#3}
%\let\underscore\_
%\renewcommand{\_}[1]{_\mr{#1}}
%\newcommand{\oo}[1]{^{(#1)}}
%\newcommand{\rr}{\bm r}%{x,y}
%\newcommand{\cp}{c\_p}
%\let\Re\relax
%\let\Im\relax
%\DeclareMathOperator\Re{Re}
%\DeclareMathOperator\Im{Im}
%\newcommand{\w}{w}
%\newcommand{\bU}{\bm U}
%\newcommand{\h}{\hat}
%\newcommand{\rbr}[1]{\left(#1\right)}
%\newcommand{\sbr}[1]{\left[#1\right]}
%\newcommand{\cbr}[1]{\left\{#1\right\}}
%\newcommand{\z}{z}
%\newcommand{\x}{x}
%\newcommand{\y}{y}
%\newcommand{\zz}{\zeta}
%\newcommand{\xx}{\xi}
%\newcommand{\yy}{\sigma}
%%\newcommand{\k}{k}
%\newcommand{\kk}{\kappa}
%
%\newcommand{\zmap}{f}
%%\newcommand{\zzmap}{\zmap^{-1}}
%\newcommand{\zzmap}{\zmap^{\raisebox{.2ex}{$\scriptscriptstyle-1$}}}
%
%\newcommand{\ww}{\omega}
%\renewcommand{\w}{w}
%\newcommand{\surf}{\eta}
%\newcommand{\dd}[2]{\frac{\mr d #1}{\mr d #2}}
%
%\begin{document}
%

Results for bethymetries of a linear transition in water depth are presented here.
Conformal maps are integrated numerically, as described in \autoref{sec:SCnum}.
Results from the numerical map integration method have of course been benchmarked up against the algebraic map results of the previous section.
\\

Maps of slopes $\theta=\pi/4$, $\pi/20$ and $\pi/40$ are shown in \autoref{fig:res:map_slope} with corresponding simulations displayed in \autoref{fig:res:slope1} to \ref{fig:res:slope3}, all with $H\_s/H\_d = 0.5$.
Compared with a step transition (\autoref{fig:res:logstrip2}), very little difference is noted in terms of spurious waves  with 45\textdegree slope (\autoref{fig:res:slope1}). 
Reducing the slope to $\theta=\pi/20$ (or 9\textdegree) reduces the amplitude of the linear reflected wave packet, as shown in \autoref{fig:res:slope2}. 
No notable effect is seen on the second-order transmitted free wave packet.
Further reducing the slope to $\theta=\pi/40$ (or 4.5\textdegree) yields some reduction also in the amplitude of this spurious transmitted packet.
Notice however that the spurious packets are wider than the carrier packet.



\begin{figure}[h!ptb]%
\centering
\subfloat[$\theta=\pi/4$.]{\includegraphics[width=.33\columnwidth]{../HOS_bathymetry/figures/map/mapZoom_SSGW_ka0p05_H1p00_0p50_nH2_ang1_0p5_Nw60.pdf}}%
\subfloat[$\theta=\pi/20$.]{\includegraphics[width=.33\columnwidth]{../HOS_bathymetry/figures/map/mapZoom_SSGW_ka0p05_H1p00_0p50_nH2_ang1_0p1_Nw60.pdf}}%
\subfloat[$\theta=\pi/40$.]{\includegraphics[width=.33\columnwidth]{../HOS_bathymetry/figures/map/mapZoom_SSGW_ka0p05_H1p00_0p50_nH2_ang1_0p05_Nw60.pdf}}%
\caption{Conformal map, single slope; $H\_d = 1.0$, $H\_s = 0.5$}%
\label{fig:res:map_slope}%
\end{figure}

\begin{figure}[h!ptb]%
\centering
\includegraphics[width=1\columnwidth]{../HOS_bathymetry/figures/SSGW_ka0p05_M5_H1p00_0p50_nH2_ang1_0p5_Nw60_dt5T_nx3840_pad0_ikCutInf_Md0p5_r0p25.pdf}%
\caption{Surface elevation, $(ka)_0 = 0.05$, $(kH)_0 = 1.00$ ($H\_d=1.0$\,m corresponds to $T\approx2.30$\,s); single slope with $\theta=\pi/4$ (45\textdegree); $H\_s/H\_d = 0.5$. Dashed lines indicate $\x$-location of slope transition beginning and end.}%
\label{fig:res:slope1}%
\end{figure}
\begin{figure}[h!ptb]%
\centering 
\includegraphics[width=1\columnwidth]{../HOS_bathymetry/figures/SSGW_ka0p05_M5_H1p00_0p50_nH2_ang1_0p1_Nw60_dt5T_nx3840_pad0_ikCutInf_Md0p5_r0p25.pdf}%
\caption{Similar to \autoref{fig:res:slope1}, but with $\theta=\pi/20$  (9\textdegree).}%
\label{fig:res:slope2}%
\end{figure}
\begin{figure}[h!ptb]%
\centering 															  
\includegraphics[width=1\columnwidth]{../HOS_bathymetry/figures/SSGW_ka0p05_M5_H1p00_0p50_nH2_ang1_0p05_Nw60_dt5T_nx3840_pad0_ikCutInf_Md0p5_r0p25.pdf}%
\caption{Similar to \autoref{fig:res:slope1}, but with $\theta=\pi/40$ (4.5\textdegree).}%
\label{fig:res:slope3}%
\end{figure}






%\big|\zeta_{0T,f}^{(22,0)}\big| &= \frac{2\omega_0}{g}|T_{20}| A_0^2.


%\end{document}


\subsection{Effect of depth transition on a wave spectrum}
\label{sec:results:spectrum}
We now move on to irregular waves.
The new HOS code has been extended with options for a closed domain, a wavemaker boundary and a numerical beach. 
This allows for simulations of a continuously incoming wave filed that is absorbed on the far end of the domain. 
The wavemaker boundary is linear and can so be imprecise when compared to nonlinear wavemaker kinematics. 
Wavemaker inaccuracies are however of little relevance since we are interested in the spatial evolution of whatever energy spectrum will be generated. 
An example of the spatial surface elevation during several stages of simulation is shown in \autoref{fig:ts:flat}.
See \citet{SFo2018_HOS,bonnefoy2010,bonnefoy2006A_BM,ducrozet2006_BM} for details on numerical wavemaker and beach.
\\

%\begin{figure}[H]%
%\centering
%\includegraphics[width=.5\columnwidth]{../HOS_bathymetry/figures/map/mapZoom_81300_linear_ka0_H3p00_0p50_theta90_Nw1.pdf}%
%\caption{Bathymetry: Map of step form depth 3 meters to 0.5 meters. Wavemaker at $x=-25$\,m, beach covers $x=8.3$ to 25.0\,m.}%
%\label{fig:map:step}%
%\end{figure}



\begin{figure}[H]%
\centering
\includegraphics[width=.9\columnwidth]{../HOS_bathymetry/figures/81300_closed_linear_T2p50_ka0_M5_H0p50_theta_Nw1_dt1T_nx1024_pad0_ikCut256_Md0_r0.pdf}%
\caption{Time record of surface elevation for flat bed simulation (0.5 meter depth).}%
\label{fig:ts:flat}%
\end{figure}

%\begin{figure}[H]%
%\centering
%\subfloat[Flat bed]{
%\includegraphics[width=.33\columnwidth]{../HOS_bathymetry/figures/S_81300_closed_linear_T2p50_ka0_M5_H0p50_theta_Nw1_dt1T_nx1024_pad0_ikCut256_Md0_r0.pdf}}%
%\subfloat[Bathymetry step, wavemaker signal attuned to depth 0.5\,m]{
%\includegraphics[width=.33\columnwidth]{../HOS_bathymetry/figures/S_81300_closed_linear_T2p50_ka0_M5_H3p00_0p50_theta90_Nw1_dt1T_nx1024_pad0_ikCut256_Md0_r0.pdf}}%
%\subfloat[Bathymetry step, wavemaker signal attuned to depth 3.0\,m]{
%\includegraphics[width=.33\columnwidth]{../HOS_bathymetry/figures/S_81500_closed_linear_T2p50_ka0_M5_H3p00_0p50_theta90_Nw1_dt1T_nx1024_pad0_ikCut256_Md0_r0.pdf}}%
%\label{fig:S}%
%\end{figure}

%\begin{figure}[H]%
%\centering
%\subfloat[Flat bed]{
%\includegraphics[width=.5\columnwidth]{../HOS_bathymetry/figures/S_81300_closed_linear_T2p50_ka0_M5_H0p50_theta_Nw1_dt1T_nx1024_pad0_ikCut256_Md0_r0.pdf}}%
%\subfloat[Bathymetry step]{
%\includegraphics[width=.5\columnwidth]{../HOS_bathymetry/figures/S_81500_closed_linear_T2p50_ka0_M5_H3p00_0p50_theta90_Nw1_dt1T_nx1024_pad0_ikCut256_Md0_r0.pdf}}%
%\caption{Power spectra measured at in the domain centre, at $x=0$\,m.
%$T\_p=2.0$\,s, $H\_s=0.075$\,m.}
%\label{fig:S}%
%\end{figure}



%\begin{figure}[H]%
%\centering
%\subfloat[Flat bed]{
%\includegraphics[width=.5\columnwidth]{../HOS_bathymetry/figures/S_81200_closed_M5_H0p50_theta_Nw1_nx1536_L75_Lb50_pad0_ikCut384_Md0_r0.pdf}}%
%\subfloat[Bathymetry step]{
%\includegraphics[width=.5\columnwidth]{../HOS_bathymetry/figures/S_81600_closed_M5_H3p00_0p50_theta90_Nw1_nx1536_L75_Lb50_pad0_ikCut384_Md0_r0.pdf}}%
%\caption{Power spectra measured at in the domain centre, at $x=0$\,m.
%$T\_p=2.0$\,s, $H\_s=0.125$\,m.}
%\label{fig:S}%
%\end{figure}



%% THESE MAY BE USED, BUT COMBINE WITH OTHER SLOPES FIRST!
%\begin{figure}[H]%
%\centering
%\subfloat[$x=5.0$\,m.]{
%\includegraphics[width=.5\columnwidth]{../HOS_bathymetry/figures/S_5_83010_closed_M5_H0p50_theta_Nw1_nx2048_L50_Lb16p7_pad0_ikCut512_Md0_r0.pdf}}%
%\subfloat[$x=10.0$\,m.]{
%\includegraphics[width=.5\columnwidth]{../HOS_bathymetry/figures/S_10_83010_closed_M5_H0p50_theta_Nw1_nx2048_L50_Lb16p7_pad0_ikCut512_Md0_r0.pdf}}\\
%\subfloat[$x=15.0$\,m.]{
%\includegraphics[width=.5\columnwidth]{../HOS_bathymetry/figures/S_15_83010_closed_M5_H0p50_theta_Nw1_nx2048_L50_Lb16p7_pad0_ikCut512_Md0_r0.pdf}}%
%\subfloat[$x=20.0$\,m.]{
%\includegraphics[width=.5\columnwidth]{../HOS_bathymetry/figures/S_20_83010_closed_M5_H0p50_theta_Nw1_nx2048_L50_Lb16p7_pad0_ikCut512_Md0_r0.pdf}}%
%\caption{Power spectra measured at various locations. Flat bottom.
%$T\_p=2.0$\,s, $H\_s=0.100$\,m. (Wave breaking/simulation crash with $H\_s=0.125$\,m.)}
%\label{fig:S}%
%\end{figure}

%\begin{figure}[H]%
%\centering
%\subfloat[$x=5.0$\,m.]{
%\includegraphics[width=.5\columnwidth]{../HOS_bathymetry/figures/S_5_82000_closed_M5_H3p00_0p50_theta90_Nw1_nx2048_L50_Lb16p7_pad0_ikCut512_Md0_r0.pdf}}%
%\subfloat[$x=10.0$\,m.]{
%\includegraphics[width=.5\columnwidth]{../HOS_bathymetry/figures/S_10_82000_closed_M5_H3p00_0p50_theta90_Nw1_nx2048_L50_Lb16p7_pad0_ikCut512_Md0_r0.pdf}}\\
%\subfloat[$x=15.0$\,m.]{
%\includegraphics[width=.5\columnwidth]{../HOS_bathymetry/figures/S_15_82000_closed_M5_H3p00_0p50_theta90_Nw1_nx2048_L50_Lb16p7_pad0_ikCut512_Md0_r0.pdf}}%
%\subfloat[$x=20.0$\,m.]{
%\includegraphics[width=.5\columnwidth]{../HOS_bathymetry/figures/S_20_82000_closed_M5_H3p00_0p50_theta90_Nw1_nx2048_L50_Lb16p7_pad0_ikCut512_Md0_r0.pdf}}%
%\caption{Power spectra measured at various locations. Step discontinuity at $x=12.2$\,m.
%$T\_p=2.0$\,s, $H\_s=0.125$\,m.}
%\label{fig:S}%
%\end{figure}





%\begin{figure}[H]%
%\centering
%\subfloat[$x=5.0$\,m.]{
%\includegraphics[width=.5\columnwidth]{../HOS_bathymetry/figures/S_5_82110_closed_M5_H0p50_theta_Nw1_nx2048_L50_Lb16p7_pad0_ikCut512_Md0_r0.pdf}}%
%\subfloat[$x=10.0$\,m.]{
%\includegraphics[width=.5\columnwidth]{../HOS_bathymetry/figures/S_10_82110_closed_M5_H0p50_theta_Nw1_nx2048_L50_Lb16p7_pad0_ikCut512_Md0_r0.pdf}}\\
%\subfloat[$x=15.0$\,m.]{
%\includegraphics[width=.5\columnwidth]{../HOS_bathymetry/figures/S_15_82110_closed_M5_H0p50_theta_Nw1_nx2048_L50_Lb16p7_pad0_ikCut512_Md0_r0.pdf}}%
%\subfloat[$x=20.0$\,m.]{ 
%\includegraphics[width=.5\columnwidth]{../HOS_bathymetry/figures/S_20_82110_closed_M5_H0p50_theta_Nw1_nx2048_L50_Lb16p7_pad0_ikCut512_Md0_r0.pdf}}%
%\caption{Power spectra measured at various locations. Flat bottom.
%$T\_p=2.5$\,s, $H\_s=0.125$\,m.}
%\label{fig:S}%
%\end{figure}

%\begin{figure}[H]%
%\centering
%\subfloat[$x=5.0$\,m.]{
%\includegraphics[width=.5\columnwidth]{../HOS_bathymetry/figures/S_5_82100_closed_M5_H3p00_0p50_theta90_Nw1_nx2048_L50_Lb16p7_pad0_ikCut512_Md0_r0.pdf}}%
%\subfloat[$x=10.0$\,m.]{
%\includegraphics[width=.5\columnwidth]{../HOS_bathymetry/figures/S_10_82100_closed_M5_H3p00_0p50_theta90_Nw1_nx2048_L50_Lb16p7_pad0_ikCut512_Md0_r0.pdf}}\\
%\subfloat[$x=15.0$\,m.]{
%\includegraphics[width=.5\columnwidth]{../HOS_bathymetry/figures/S_15_82100_closed_M5_H3p00_0p50_theta90_Nw1_nx2048_L50_Lb16p7_pad0_ikCut512_Md0_r0.pdf}}%
%\subfloat[$x=20.0$\,m.]{
%\includegraphics[width=.5\columnwidth]{../HOS_bathymetry/figures/S_20_82100_closed_M5_H3p00_0p50_theta90_Nw1_nx2048_L50_Lb16p7_pad0_ikCut512_Md0_r0.pdf}}%
%\caption{Power spectra measured at various locations. Step discontinuity at $x=12.2$\,m.
%$T\_p=2.5$\,s, $H\_s=0.125$\,m.}
%\label{fig:S}%
%\end{figure}


We now consider a depth transition example where the water depth decreases from 3.0 meters to 0.5 meters some distance form the wavemaker. 
Three transition slopes, $\theta = 90$\textdegree, $45$\textdegree and $15$\textdegree, are compared.
Power spectra measured as various domain locations are shown in \autoref{fig:Tp2p5:slopes}.
A JONSWAP spectrum with $T\_p=2.5$\,s, $H\_s=0.125$\,m is here the wavemaker target.

Energy around the peak period is observed to deviate more from target with abrupt depth transition than with a gradual transition. 
It is possible that this finding is related to the first-order reflection of wave packets observed in \autoref{sec:results:step}  and  \ref{sec:results:slope}, with reflected waves bouncing back within the deep water section of the domain and interacting with the wavemaker. 
All transition slops generate a secondary energy peak around the double frequency $2/T\_p$, analogous to the second-order wave packets seen earlier. 
Reduced slope angles do not appear no mitigate this development.
The second order energy peaks diminish further away from the depth transition, likely due to further nonlinear development. 
It is possible that numerical damping effects, for example for the spectral cut-off in the HOS scheme, also influence this development. 
 
For comparison, a similar spectrum from simulation domains with no depth transition is shown in \autoref{fig:Tp2p5:flat}.
These domain have depths of 3.0 and 0.5 meters all over and the wavemaker signal is adjusted accordingly. 
A second order energy peak is seen close to the wavemaker in the shallow case. It diminished with increasing distance. 
It is possibly generated by the wavemaker, both is a physical sense and in the sense of errors from the wavemaker approximation.
Notable low-frequency energy is also observed in the shallow water case, which possesses stronger nonlinearities. 
Again, beach damping of low-frequency waves is inefficient, so these surges may travel beck and forth across the numerical wave tank. 
\\

Similar observations are made in  \autoref{fig:Tp2p0:slopes}\ref{fig:Tp2p0:flat} with the target spectrum $T\_p=2.0$\,s, $H\_s=0.100$\,m.
The bathymetry effect is smaller at this period, yet still notable in the energy deficit in the region $T\sim 6$\,s.

\begin{figure}[H]%
\centering
\subfloat[Bathymetries]{
\includegraphics[width=.33\columnwidth]{../HOS_bathymetry/figures/map/mapZoom_82100_linear_ka0_H3p00_0p50_theta90_Nw1.pdf}
\includegraphics[width=.33\columnwidth]{../HOS_bathymetry/figures/map/mapZoom_82000_linear_ka0_H3p00_0p50_theta45_Nw1.pdf}
\includegraphics[width=.33\columnwidth]{../HOS_bathymetry/figures/map/mapZoom_82100_linear_ka0_H3p00_0p50_theta15_Nw1.pdf}
}\\%
\subfloat[$x=0.0$\,m.]{
\includegraphics[width=.5\columnwidth]{../HOS_bathymetry/figures/powerSpec/82100/x_wp0.pdf}}%
\subfloat[$x=5.0$\,m.]{
\includegraphics[width=.5\columnwidth]{../HOS_bathymetry/figures/powerSpec/82100/x_wp5.pdf}}\\
\subfloat[$x=10.0$\,m.]{
\includegraphics[width=.5\columnwidth]{../HOS_bathymetry/figures/powerSpec/82100/x_wp10.pdf}}%
\subfloat[$x=15.0$\,m.]{
\includegraphics[width=.5\columnwidth]{../HOS_bathymetry/figures/powerSpec/82100/x_wp15.pdf}}\\
\subfloat[$x=20.0$\,m.]{
\includegraphics[width=.5\columnwidth]{../HOS_bathymetry/figures/powerSpec/82100/x_wp20.pdf}}%
\subfloat[$x=25.0$\,m.]{
\includegraphics[width=.5\columnwidth]{../HOS_bathymetry/figures/powerSpec/82100/x_wp25.pdf}}%
\caption{Power spectra measured at various locations, comparing transitional ramps.
Depth transition from 3.0 to 0.5 meters.
$T\_p=2.5$\,s, $H\_s=0.125$\,m.}
\label{fig:Tp2p5:slopes}%
\end{figure}

\begin{figure}[H]%
\centering
\subfloat[$x=0.0$\,m.]{
\includegraphics[width=.5\columnwidth]{../HOS_bathymetry/figures/powerSpec/821x0_2xflatOnly/x_wp0.pdf}}%
\subfloat[$x=5.0$\,m.]{
\includegraphics[width=.5\columnwidth]{../HOS_bathymetry/figures/powerSpec/821x0_2xflatOnly/x_wp5.pdf}}\\%
\subfloat[$x=10.0$\,m.]{
\includegraphics[width=.5\columnwidth]{../HOS_bathymetry/figures/powerSpec/821x0_2xflatOnly/x_wp10.pdf}}%
\subfloat[$x=15.0$\,m.]{
\includegraphics[width=.5\columnwidth]{../HOS_bathymetry/figures/powerSpec/821x0_2xflatOnly/x_wp15.pdf}}\\%
\subfloat[$x=20.0$\,m.]{
\includegraphics[width=.5\columnwidth]{../HOS_bathymetry/figures/powerSpec/821x0_2xflatOnly/x_wp20.pdf}}%
\subfloat[$x=25.0$\,m.]{
\includegraphics[width=.5\columnwidth]{../HOS_bathymetry/figures/powerSpec/821x0_2xflatOnly/x_wp25.pdf}}%
\caption{Power spectra measured at various locations, flat bed at 0.5 meters depth.
$T\_p=2.5$\,s, $H\_s=0.125$\,m.}
\label{fig:Tp2p5:flat}%
\end{figure}




%%%%%%%%%%%%%

\begin{figure}[H]%
\centering
\subfloat[$x=0.0$\,m.]{
\includegraphics[width=.5\columnwidth]{../HOS_bathymetry/figures/powerSpec/83000/x_wp0.pdf}}%
\subfloat[$x=5.0$\,m.]{
\includegraphics[width=.5\columnwidth]{../HOS_bathymetry/figures/powerSpec/83000/x_wp5.pdf}}\\
\subfloat[$x=10.0$\,m.]{
\includegraphics[width=.5\columnwidth]{../HOS_bathymetry/figures/powerSpec/83000/x_wp10.pdf}}%
\subfloat[$x=15.0$\,m.]{
\includegraphics[width=.5\columnwidth]{../HOS_bathymetry/figures/powerSpec/83000/x_wp15.pdf}}\\
\subfloat[$x=20.0$\,m.]{
\includegraphics[width=.5\columnwidth]{../HOS_bathymetry/figures/powerSpec/83000/x_wp20.pdf}}%
\subfloat[$x=25.0$\,m.]{
\includegraphics[width=.5\columnwidth]{../HOS_bathymetry/figures/powerSpec/83000/x_wp25.pdf}}%
\caption{Power spectra measured at various locations, comparing transitional ramps.
Depth transition from 3.0 to 0.5 meters.
$T\_p=2.0$\,s, $H\_s=0.100$\,m.}
\label{fig:Tp2p0:slopes}%
\end{figure}

\begin{figure}[H]%
\centering
\subfloat[$x=0.0$\,m.]{
\includegraphics[width=.5\columnwidth]{../HOS_bathymetry/figures/powerSpec/830x0_2xflatOnly/x_wp0.pdf}}%
\subfloat[$x=5.0$\,m.]{
\includegraphics[width=.5\columnwidth]{../HOS_bathymetry/figures/powerSpec/830x0_2xflatOnly/x_wp5.pdf}}\\%
\subfloat[$x=10.0$\,m.]{
\includegraphics[width=.5\columnwidth]{../HOS_bathymetry/figures/powerSpec/830x0_2xflatOnly/x_wp10.pdf}}%
\subfloat[$x=15.0$\,m.]{
\includegraphics[width=.5\columnwidth]{../HOS_bathymetry/figures/powerSpec/830x0_2xflatOnly/x_wp15.pdf}}\\%
\subfloat[$x=20.0$\,m.]{
\includegraphics[width=.5\columnwidth]{../HOS_bathymetry/figures/powerSpec/830x0_2xflatOnly/x_wp20.pdf}}%
\subfloat[$x=25.0$\,m.]{
\includegraphics[width=.5\columnwidth]{../HOS_bathymetry/figures/powerSpec/830x0_2xflatOnly/x_wp25.pdf}}%
\caption{Power spectra measured at various locations, flat bed at 0.5 meters depth.
$T\_p=2.0$\,s, $H\_s=0.100$\,m.}
\label{fig:Tp2p0:flat}%
\end{figure}
 
 
\bibliographystyle{abbrvnat} % abbrvnat,plainnat,unsrtnat
\bibliography{../sintef_bib}


\end{document}
