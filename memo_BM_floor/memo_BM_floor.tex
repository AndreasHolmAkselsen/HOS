%\documentclass[a4paper,12pt]{article}
\documentclass[internal]{sintefmemo}
%\documentclass{amsart}
%\usepackage[a4paper, total={17cm, 25cm}]{geometry}

%\usepackage[utf8]{inputenc}
\usepackage[english]{babel}
%\usepackage{amsmath,bm,amsfonts,amssymb}
\usepackage{xcolor}
\usepackage{graphicx}
\usepackage{graphbox} % allows includegraphics[align=c]
\usepackage[round]{natbib}
\usepackage{mathtools}
\usepackage[font=normalsize,margin=2mm]{subfig}
\usepackage{float}
\usepackage{hyperref}
\usepackage{xfrac}
%\usepackage{cprotect}%for \verb in captions
%\usepackage{enumerate}
\usepackage{enumitem}
\hypersetup{colorlinks=true,
			linkcolor=blue,
			filecolor=blue,
			urlcolor=blue,
			citecolor=blue}
\newcommand{\mr}{\mathrm}
\newcommand{\mc}{\mathcal}
\let\SSS\S
\renewcommand{\S}{^\mr{S}}
\newcommand{\ii}{\mr{i}\,}
\newcommand{\ee}{\mr{e}}
\let\underscore\_
\renewcommand{\_}[1]{_\mr{#1}}
\newcommand{\oo}[1]{^{(#1)}}
\let\Re\relax
\let\Im\relax
\DeclareMathOperator\Re{Re}
\DeclareMathOperator\Im{Im}
\newcommand{\w}{w}
\newcommand{\bU}{\bm U}
\newcommand{\h}{\hat}
\newcommand{\br}[3]{\left#1#2\right#3}
\newcommand{\rbr}[1]{\left(#1\right)}
\newcommand{\sbr}[1]{\left[#1\right]}
\newcommand{\cbr}[1]{\left\{#1\right\}}
%\newcommand{\bU}{(\nabla\Phi)_{z=\eta}}

\usepackage{pifont}% http://ctan.org/pkg/pifont
\newcommand{\cmark}{\text{\ding{51}}}% or \checkmark
\newcommand{\xmark}{\text{\ding{55}}}%

\newcommand{\z}{z}
\newcommand{\x}{x}
\newcommand{\y}{y}
\newcommand{\zz}{\zeta}
\newcommand{\xx}{\xi}
\newcommand{\yy}{\sigma}
\newcommand{\kk}{\kappa}

\newcommand{\zmap}{f}
%\newcommand{\zzmap}{\zmap^{-1}}
\newcommand{\zzmap}{\zmap^{\raisebox{.2ex}{$\scriptscriptstyle-1$}}}

%\newcommand{\ww}{w}
%\renewcommand{\w}{\ww^\mr{P}}
\newcommand{\ww}{\omega}
\renewcommand{\w}{w}

\newcommand{\surf}{\eta}
%\newcommand{\w}{\varpi}
\newcommand{\dd}{\mr d}
\newcommand{\ddfrac}[2]{\frac{\dd #1}{\dd #2}}

\title{Assessment of the impact of depth transitions on wave quality}
\author{Andreas H. Akselsen}
\project{302006355-3 (OSC pre-project)}
\date{\today}
\recipient[information]{SINTEF employees}
%\recipient[information,agreed]{Statsbygg}

\begin{document}


%\maketitle
\frontmatter
\tableofcontents

\section{Introduction}




\section{Method}
\label{sec:method}

This study is based on the higher order spectral method adjusted using conformal mapping to acount for depth transitions. 
Development of this original method started at SINTEF Ocean in late 2021 for presicely this purpose.
A quick description now follows. 
More information is provided in the memo \citet{AHA_2021_conformalHOS} and hopfully in future publications.

\subsection{Boundary equations for the mapped plane}
%Option \ref{it:chosen} seems the most promising of these options.
The boundary value problem 
\begin{align*}
h_t + \phi_x h_x-\phi_y&=0,\\
\phi_t + \frac12\rbr{\phi_x^2+\phi_y^2}+gh&=0
\end{align*}
can be re-stated in terms of the surface coordinate
\[
\x+\ii h(\x,t) \mapsfrom f[\xx+\ii \eta(\xx,t)].
\]
Introducing
\[\varphi(\xx,\yy,t)=\Re\ww(\zz,t) \]
and the boundary conditions expressed in terms of the surface potential 
\[
\varphi\S(\xx,t)=\varphi[\xx+\ii\eta(\xx,t),t]
\]
we get 
\begin{subequations}
\begin{align}
\eta_t &= |\zmap'|^{-2} \sbr{-   \varphi\S_\xx\eta_\xx + \rbr{1+\eta_\xx^2} \varphi_\yy},\\
\varphi\S_t  &= |\zmap'|^{-2}\sbr{ - \frac12  \rbr{\varphi\S_\xx}^2 + \frac12 \rbr{1+\eta_\xx^2} \varphi_\yy^2 }  - g h,
\end{align}%
\end{subequations}%
after some derivation.
Here $|\zmap'|^{2}$ is equivalent to the transformation Jacobian and $\varphi_\yy$ is evaluated at $\zz=\xx+\ii\eta$.
$h=\Im \zmap(\xx+\ii\eta)$ is readily available. 

\subsection{Mapping}
%The Schwartz--Christoffel transform is a powerful approach for imposing straight angles in a mapping.
A map of a depth transition from $H\_s$ to $H\_d$ with sloping angle $\theta$ is obtained by integrating
\begin{equation}
%\zmap'(\zz) = \frac{H\_s}{\pi \tau_\theta}, \qquad \tau_\theta = \rbr{\frac{\ee^\zz+ (H\_d/H\_s)^{\pi/\theta}}{\ee^\zz+1}}^{\theta/\pi}. 
\zmap'(\zz) = \frac{H\_s}{\pi} \rbr{\frac{\ee^\zz+1}{\ee^\zz+ (H\_d/H\_s)^{\pi/\theta}}}^{\theta/\pi}. 
\label{eq:SCnumStep}
\end{equation}
Any integration path can be used since the map is conformal; we integrate from the top of the domain down to avoid spreading numerical imprecision form the singularities located at $\zeta=-\ii\pi$ and $\zeta=2\ln c-\ii\pi$ into the domain.
An example is shown on \autoref{fig:SCnumStep45deg} for a 45\textdegree down-step.

\begin{figure}[H]%
\centering
\subfloat[$\zz$-plane]{\includegraphics[width=.47\columnwidth]{../conformalMapping/conformalBathymetry/SCnumStep45deg_zz.pdf}}%
\hfill
\subfloat[$\z$-plane]{\includegraphics[width=.47\columnwidth]{../conformalMapping/conformalBathymetry/SCnumStep45deg_z.pdf}}%
\caption{Numerical integration of \eqref{eq:SCnumStep} with $\theta=\pi/4$, $H\_s = 0.5$ and $H\_d = 1.0$.}%
\label{fig:SCnumStep45deg}%
\end{figure}


We can extend this technique to combine several alterations in bathymetry level by adopting
\begin{equation}
\zmap'(\zz) = K \prod_{j=1}^{N_\angle} \rbr{\frac{\exp(\zz-\xi^\angle_j)+1}{\exp(\zz-\xx^\angle_j) + (H_{j+1}/H_j)^{\pi/\theta_j}}}^{\theta_j/\pi}
\label{eq:SCnumMultiStep}
\end{equation}
and fixing the constant $K$ after integration (re-scaling). $N_\angle$ is here the number of depth transitions, $\{H_j\}$ the set of depths, $\{\theta_j\}$ ($>0$) the slopes of the individual transitions and $\{\xi^\angle_j\}$ the horizontal position around which it occurs in the $\zz$-plane. 
%The singularities of \eqref{eq:SCnumMultiStep} are found at $\zz=\xx^\angle_j-\ii \pi$ and $\zz=\xx^\angle_j+\pi/\theta_j\ln(H_{j+1}/H_j)-\ii \pi$.
%\autoref{fig:SCnumStep2x45deg} shows an example similar to the plateau seen earlier. 
%A rougher bathymetry is displayed in \autoref{fig:SCnumMultiStep}. 
%The deformation of some of the sharp corners in this image is due to inaccuracy in the numeral integration near the singularity and is of no consequence for simulation usage.
%Corners can of course be smoothened by defining a higher bathymetry level $\yy=\tilde H < \pi$ in the $\zz$-plane, as shown in \autoref{fig:SCnumMultiStepSmooth}.
%
%\begin{figure}[H]%
%\centering
%\subfloat[$\zz$-plane]{\includegraphics[width=.47\columnwidth]{../conformalMapping/conformalBathymetry/SCnumStep2x45deg_zz.pdf}}%
%\hfill
%\subfloat[$\z$-plane]{\includegraphics[width=.47\columnwidth]{../conformalMapping/conformalBathymetry/SCnumStep2x45deg_z.pdf}}%
%\caption{Numerical integration of \eqref{eq:SCnumMultiStep} with 
%%$\theta_1=\theta_2=\pi/4$; $H_1 = 1.0$, $H_2 = 0.5$, $H_3 = 1.0$ and $\xi^\angle_1=0$, $\xi^\angle_1=5$.
%$\theta_j\in\{\pi/4,\pi/4\}$, $H_j \in\{1.0,0.5,1.0\}$ and $\xx^\angle_j\in\{0,5\}$.
%}%
%\label{fig:SCnumStep2x45deg}%
%\end{figure}
%
%\begin{figure}[H]%
%\centering
%\subfloat[$\zz$-plane]{\includegraphics[width=.47\columnwidth]{../conformalMapping/conformalBathymetry/SCnumStepMulti_zz.pdf}}%
%\hfill
%\subfloat[$\z$-plane]{\includegraphics[width=.47\columnwidth]{../conformalMapping/conformalBathymetry/SCnumStepMulti_z.pdf}}%
%\caption{Numerical integration of \eqref{eq:SCnumMultiStep} with 
%$\theta_j\in\{\pi/4,5\pi/2,5\pi/2\}$, $H_j \in\{0.5,1.5,0.75,1.5\}$ and $\xx^\angle_j\in\{0,8,14\}$.
%}%
%\label{fig:SCnumMultiStep}%
%\end{figure}
%
%\begin{figure}[H]%
%\centering
%\includegraphics[width=.47\columnwidth]{../conformalMapping/conformalBathymetry/SCnumStepMultiSmooth_z.pdf}%
%\caption{As \autoref{fig:SCnumMultiStep}, instead choosing bethymetry at $\yy = 0.1-\pi$.
%}%
%\label{fig:SCnumMultiStepSmooth}%
%\end{figure}


\section{Spurious waves generated by a train of regular waves}
\label{sec:train}
 %\documentclass[a4paper,12pt]{article}
%%\documentclass{amsart}
%\usepackage[a4paper, total={17cm, 25cm}]{geometry}
%
%\usepackage[utf8]{inputenc}
\usepackage[english]{babel}
%\usepackage{amsmath,bm,amsfonts,amssymb}
\usepackage{xcolor}
\usepackage{graphicx}
\usepackage{graphbox} % allows includegraphics[align=c]
\usepackage[round]{natbib}
\usepackage{mathtools}
\usepackage[font=normalsize,margin=2mm]{subfig}
\usepackage{float}
\usepackage{hyperref}
\usepackage{xfrac}
%\usepackage{cprotect}%for \verb in captions
%\usepackage{enumerate}
\usepackage{enumitem}
\hypersetup{colorlinks=true,
			linkcolor=blue,
			filecolor=blue,
			urlcolor=blue,
			citecolor=blue}
\newcommand{\mr}{\mathrm}
\newcommand{\mc}{\mathcal}
\let\SSS\S
\renewcommand{\S}{^\mr{S}}
\newcommand{\ii}{\mr{i}\,}
\newcommand{\ee}{\mr{e}}
\let\underscore\_
\renewcommand{\_}[1]{_\mr{#1}}
\newcommand{\oo}[1]{^{(#1)}}
\let\Re\relax
\let\Im\relax
\DeclareMathOperator\Re{Re}
\DeclareMathOperator\Im{Im}
\newcommand{\w}{w}
\newcommand{\bU}{\bm U}
\newcommand{\h}{\hat}
\newcommand{\br}[3]{\left#1#2\right#3}
\newcommand{\rbr}[1]{\left(#1\right)}
\newcommand{\sbr}[1]{\left[#1\right]}
\newcommand{\cbr}[1]{\left\{#1\right\}}
%\newcommand{\bU}{(\nabla\Phi)_{z=\eta}}

\usepackage{pifont}% http://ctan.org/pkg/pifont
\newcommand{\cmark}{\text{\ding{51}}}% or \checkmark
\newcommand{\xmark}{\text{\ding{55}}}%

\newcommand{\z}{z}
\newcommand{\x}{x}
\newcommand{\y}{y}
\newcommand{\zz}{\zeta}
\newcommand{\xx}{\xi}
\newcommand{\yy}{\sigma}
\newcommand{\kk}{\kappa}

\newcommand{\zmap}{f}
%\newcommand{\zzmap}{\zmap^{-1}}
\newcommand{\zzmap}{\zmap^{\raisebox{.2ex}{$\scriptscriptstyle-1$}}}

%\newcommand{\ww}{w}
%\renewcommand{\w}{\ww^\mr{P}}
\newcommand{\ww}{\omega}
\renewcommand{\w}{w}

\newcommand{\surf}{\eta}
%\newcommand{\w}{\varpi}
\newcommand{\dd}{\mr d}
\newcommand{\ddfrac}[2]{\frac{\dd #1}{\dd #2}}

%
%\begin{document}

We will in this section examine generation of spurious waves ensuing from a Strokes wave propagating across a water depth transition. Both abrupt and slanting transitions are examined, from deep to shallow and shallow to deep.
Simulations are carried out by initiating a sufficiently long Stokes wave train in front of a transition.  
SSGW is used for the initial conditions and the train is tapered in both ends.
A window in time and space, typically 3--10 wavelengths $\times$ wave periods, is then interpolated form the simulation history. The window is placed such that it contains both the fundamental wave and the active spurious harmonics, avoiding tapering ramps.\\

The parameter space if interest includes the wave periods, water depth on deep and shallow side, wave steepnerss and slope of water depth transition.
We reduce the parameter space as follows: Deep side water depth is chosen as deep water,
i.e., we fixed $h\_d/\lambda\_d=0.5$ ($k\_dh\_d=\pi$, $\tanh\pi\approx0.996$).
Wave steepness is fixed as the largest steepness that we in practice can simulate without encountering wave breaking. 
(Wave breaking will result in simulation breakdown unless a wave breaking model is implemented.)
An overview of the stability map, estimated by trial, is presented in \autoref{tab:trainStab}
\\

\begin{table}[H]%
\centering
\begin{tabular}{c|ccccc}
$(ka)\_d\Big\backslash h\_s/\lambda\_d$		&0.25	 &0.20      &0.15&0.10&0.05\\\hline
0.300  	& $\cmark^*$ & $\xmark$ &  &&	\\
0.275  &$\cmark^*$&$\cmark^*$&&&\\
0.250  &$\cmark^*$&$\cmark^*$&&&\\
0.225 &$\cmark$& $\cmark$& $\xmark$ &&\\
0.200 &&&$\cmark$&$\xmark$&\\
0.150 &&&&$\xmark$&\\
0.100  &&&&$\cmark$& $\xmark$\\
0.075 &&&&& $\xmark$\\
0.050 &&&&&$\cmark$
\end{tabular}
\caption{Stability map for varying wave steepness of waves travelling across a step depth transition. $\cmark$: Non-breaking, $\cmark^*$: Survives transition but breaks early due to modulational instability, $\xmark$: Breaks at depth transition.}
\label{tab:trainStab}
\end{table}


\subsection{Transmitted waves over step transition}

%\autoref{fig:H0p3_0p3_ka0p2} to \ref{fig:H1p0_0p1_ka0p05} shows wave trains propagating over steps form deep to intermediate water depths.
%Wave steepness is restricted by wave breaking which is not yet handled by the simulator. 
We consider deep-to-shallow water depth transitions using maps such as presented in \autoref{fig:map:deepToShallow}.
\autoref{fig:H1p0_0p5_ka0p25_nw3_np3} to \ref{fig:H1p0_0p1_ka0p05_nw10_np10} shows surface elevation in the region after the depth transition in a time window that contains spurious waves released as a result of the transition.
Also shown are spectral amplitudes in space and time.
Breaking occurs at the step discontinuity for shallowest water examples. 
Wave breaking occurs at the front of the wave train due to modulational instability with deeper waters, limiting the number of crests that we are able to include in the analysis space-time window. Larger wave steepnesses will be possible with a wave breaking model in place. 
The strength of the spurious free harmonics are for the shallow water transition seen to be of magnitudes similar to the corresponding bound harmonics. 
Spurious harmonics are understandably weaker with the less shallow transitions. Again, stronger spurious harmonics would likely be observed for these cases was it not for the modulational instability that places a limit on wave steepness.



\begin{figure}[H]%
\centering
\subfloat[Step transition]{\includegraphics[width=.5\columnwidth]{../HOS_bathymetry/figures/map/map_train_SSGW_ka0p05_H1p00_0p10_theta90_Lx100.pdf}}%
\subfloat[1.0\textdegree{} slope transition starting at depth $0.15\lambda$]{\includegraphics[width=.5\columnwidth]{../HOS_bathymetry/figures/map/map_train_SSGW_ka0p05_H1p00_0p30_0p10_theta90_1_Lx100.pdf}}%
\caption{Conformal map for deep-to-shallow depth transition.}%
\label{fig:map:deepToShallow}%
\end{figure}


\newcommand{\triIm}[2]{
\begin{figure}[H]%
\centering
\subfloat[Surface elevation]{
\includegraphics[width=.33\textwidth]{../HOS_bathymetry/figures/trainAly/#1.pdf}%
\includegraphics[width=.33\textwidth]{../HOS_bathymetry/figures/trainAly/#1_xt.pdf}%
}%
\\
\subfloat[Fourier transform of surface elevation (wave amplitudes).]{
\includegraphics[width=.33\textwidth]{../HOS_bathymetry/figures/trainAly/#1_xw.pdf}%
\includegraphics[width=.33\textwidth]{../HOS_bathymetry/figures/trainAly/#1_kt.pdf}%
\includegraphics[width=.33\textwidth]{../HOS_bathymetry/figures/trainAly/#1_kw.pdf}%
}%
\caption{#2}%
\label{fig:#1}%
\end{figure}
}
\triIm{H0p3_0p3_ka0p2}{Reference: transmitted wave, no step. $h/\lambda = 0.15$, $ka = 0.2$.}
\triIm{H1p0_0p5_ka0p25_nw3_np3}{Transmitted wave, $h\_s/\lambda\_d =0.25$, $(ka)\_d = 0.25$.}%$H\_d/\lambda = 0.5$,
\triIm{H1p0_0p4_ka0p25_nw3_np3}{Transmitted wave, $h\_s/\lambda\_d =0.20$, $(ka)\_d = 0.25$.}
%\triIm{H1p0_0p4_ka0p225_nw5_np5}{$h\_s/\lambda\_d =0.20$, $(ka)\_d = 0.225$.}
\triIm{H1p0_0p3_ka0p2_nw8_np8}{Transmitted wave, $h\_s/\lambda\_d =0.15$, $(ka)\_d = 0.20$.}
\triIm{H1p0_0p2_ka0p1_nw10_np10}{Transmitted wave, $h\_s/\lambda\_d =0.10$, $(ka)\_d = 0.10$.}
\triIm{H1p0_0p1_ka0p05_nw10_np10}{Transmitted wave, $h\_s/\lambda\_d =0.05$, $(ka)\_d = 0.05$.}




\subsection{Reflected waves from step transition}

\autoref{fig:refl_H1p0_0p3_ka0p2_nw5_np5} to \ref{fig:refl_H1p0_0p1_ka0p05_nw5_np5} displays the wavenumber--frequency distribution on the reflection side of the shallow water steps. The first and third quadrant of in $k$--$\omega$ space indicates reflected amplitudes. 

\triIm{refl_H1p0_0p3_ka0p2_nw5_np5}{Reflected wave, $h\_s/\lambda\_d =0.15$, $(ka)\_d = 0.20$.}
\triIm{refl_H1p0_0p2_ka0p1_nw8_np8}{Reflected wave, $h\_s/\lambda\_d =0.10$, $(ka)\_d = 0.10$.}
\triIm{refl_H1p0_0p1_ka0p05_nw5_np5}{Reflected wave, $h\_s/\lambda\_d =0.05$, $(ka)\_d = 0.05$.}
%\triIm{refl_H1p0_0p1_ka0p05_nw8_np8}{Reflected wave, $h\_s/\lambda\_d =0.05$, $(ka)\_d = 0.05$.}




\subsection{Transmitted waves over sloping transition}
Transmission across sloping depth transitions is examined in \autoref{fig:45degDeep_H1p0_0p1_ka0p05_nw10_np10} to \ref{fig:1deg_H1p0_0p3_ka0p05_nw10_np10}.
Only the largest depth transition down to $h\_s/\lambda\_d =0.05$ is considered.
Increasing slopes of the depth transition appears to only very gradually affect the overall energy of the spurious harmonics. 
Instead, the spurious energy is seen to spread around the higher harmonics.
A good transition with little contamination if first obtained at slope 1\textdegree{}(!), see \autoref{fig:1deg_H1p0_0p3_ka0p05_nw10_np10}.

It is worthwhile to examine the surface elevation snapshots in the transition region, for example \autoref{fig:4p5deg_H1p0_0p3_ka0p05_nw10_np10}.
Waves are seen to contract across the slope transition, but are not `ready' for the shallow water depth encountered thereafter, resulting in wave crest deformations.
\\

\triIm{45degDeep_H1p0_0p1_ka0p05_nw10_np10}{Transmitted wave, 45.0\textdegree{} slope. $h\_s/\lambda\_d =0.05$, $(ka)\_d = 0.05$, cf.\ \autoref{fig:H1p0_0p1_ka0p05_nw10_np10}.}
\triIm{9degDeep_H1p0_0p1_ka0p05_nw10_np10}{Transmitted wave, 9.0\textdegree{} slope. $h\_s/\lambda\_d =0.05$, $(ka)\_d = 0.05$.}
\triIm{4p5deg_H1p0_0p3_ka0p05_nw10_np10}{Transmitted wave, 4.5\textdegree{} slope starting at depth $0.15\lambda$. $h\_s/\lambda\_d =0.05$, $(ka)\_d = 0.05$.}
\triIm{2deg_H1p0_0p3_ka0p05_nw10_np10}{Transmitted wave, 2.0\textdegree{} slope starting at depth $0.15\lambda$. $h\_s/\lambda\_d =0.05$, $(ka)\_d = 0.05$.}
\triIm{1deg_H1p0_0p3_ka0p05_nw10_np10}{Transmitted wave, 1.0\textdegree{} slope starting at depth $0.15\lambda$. $h\_s/\lambda\_d =0.05$, $(ka)\_d = 0.05$.}

Transition wave breaking is also affected by having the steepness of the depth transition. 
Waves are in \autoref{tab:trainStab} presented as breaking with a step transition for $(ka)\_d \geq 0.075$, identified by the breakdown of our simulation. 
Wave breaking is however avoided with the $1.0\textdegree{}$ transition slope. Results for this wave steepness are displayed in \autoref{fig:1degka075_H1p0_0p3_0p1_ka0p075_nw10_np10}.
Wave again break for wave steepness $(ka)\_d \geq 0.100$ or a $2.0\textdegree{}$ transition slope.

\triIm{1degka075_H1p0_0p3_0p1_ka0p075_nw10_np10}{Like \autoref{fig:1deg_H1p0_0p3_ka0p05_nw10_np10} (1.0\textdegree{}) but with a steeper wave $(ka)\_d = 0.075$.}



\subsection{Reflected waves from sloping transition}
Reflection form the sloping depth transition is examined in \autoref{fig:refl45degDeep_H1_0p1_ka0p05_nw5_np5} to \ref{fig:refl9deg_H1_0p3_0p1_ka0p05_nw5_np5}. 
Contrary to the transmitted disturbances, steeper sloping depth transitions do appear to be beneficial for reducing reflected waves. Very little reflection remains with the 9.0\textdegree{} slope as shown in \autoref{fig:refl9degDeep_H1_0p1_ka0p05_nw5_np5}. We see however that transition slopes need to extend deeper into the water to achieve full effect (\autoref{fig:refl9deg_H1_0p3_0p1_ka0p05_nw5_np5}).
This is essentially because the wavelength of the strongest reflected wave is the same as the incident wave, while the strongest spurious transmitted wave is a shorter a second order harmonic.

\triIm{refl45degDeep_H1_0p1_ka0p05_nw5_np5}{Reflected wave, 45.0\textdegree{} slope. $h\_s/\lambda\_d =0.05$, $(ka)\_d = 0.05$, cf.\ \autoref{fig:refl_H1p0_0p1_ka0p05_nw5_np5}.}
\triIm{refl9degDeep_H1_0p1_ka0p05_nw5_np5}{Reflected wave, 9.0\textdegree{} slope. $h\_s/\lambda\_d =0.05$, $(ka)\_d = 0.05$.}
\triIm{refl9deg_H1_0p3_0p1_ka0p05_nw5_np5}{Reflected wave, 9.0\textdegree{} slope starting at depth $0.15\lambda$. $h\_s/\lambda\_d =0.05$, $(ka)\_d = 0.05$.}



\subsection{Down-step}


We consider next the inverse geometries with the wave train propagating from shallow to deep water, such as shown in \autoref{fig:map:shallowToDeep}. 
\autoref{fig:downStrepRefl_H0p3_1p0_ka0p2_nw4_np4} to \ref{fig:downStrepRefl_H0p1_1p0_ka0p05_nw5_np5}
holds the results for reflected waves in a step transition from shallow depths $h\_s/\lambda\_d =0.15$ to $h\_s/\lambda\_d =0.05$, respectively.
These are seen to differ from the down-step case (\autoref{fig:refl_H1p0_0p3_ka0p2_nw5_np5} to \ref{fig:refl_H1p0_0p1_ka0p05_nw5_np5}) in that higher order harmonics dominate more.

\begin{figure}[H]%
\centering
\subfloat[Step transition]{\includegraphics[width=.5\columnwidth]{../HOS_bathymetry/figures/map/map_train_SSGW_ka0p1_H0p20_1p00_theta90_Lx150.pdf}}%
\subfloat[1.0\textdegree{} slope transition]{\includegraphics[width=.5\columnwidth]{../HOS_bathymetry/figures/map/map_train_SSGW_ka0p05_H0p10_1p00_theta1_Lx150.pdf}}%
\caption{Conformal map for shallow-to-deep depth transition.}%
\label{fig:map:shallowToDeep}%
\end{figure}


\triIm{downStrepRefl_H0p3_1p0_ka0p2_nw4_np4}{Reflected wave from down-step. $h\_s/\lambda\_d =0.15$, $(ka)\_d = 0.20$.}
\triIm{downStrepRefl_H0p2_1p0_ka0p1_nw4_np4}{Reflected wave from down-step. $h\_s/\lambda\_d =0.10$, $(ka)\_d = 0.10$.}
\triIm{downStrepRefl_H0p1_1p0_ka0p05_nw5_np5}{Reflected wave from down-step. $h\_s/\lambda\_d =0.05$, $(ka)\_d = 0.05$.}


Similar to \autoref{fig:45degDeep_H1p0_0p1_ka0p05_nw10_np10} to \ref{fig:downStrepRefl1degDeep_H0p1_0p3_1p0_ka0p05_nw3_np3}, we also consider in \autoref{fig:downStrepRefl45degDeep_H0p1_1p0_ka0p05_nw5_np5} to \ref{fig:downStrepRefl1deg_H0p1_0p3_1p0_ka0p05_nw3_np3} the effect of down-sloping depth transitions.
We let these slopes extend all the way down to deep water depths. (Abrupt depth transitions after depth $0.15\lambda$ causes additional reflection, as observed earlier.)

\triIm{downStrepRefl45degDeep_H0p1_1p0_ka0p05_nw5_np5}{Reflected wave from 45.0\textdegree{} down-sloap. $h\_s/\lambda\_d =0.05$, $(ka)\_d = 0.05$.}
\triIm{downStrepRefl9degDeep_H0p1_1p0_ka0p05_nw5_np5}{Reflected wave from 9.0\textdegree{} down-sloap, abruptly increasing after depth $0.15\lambda$. $h\_s/\lambda\_d =0.05$, $(ka)\_d = 0.05$.}
\triIm{downStrepRefl4p5degDeep_H0p1_1p0_ka0p05_nw4_np4}{Reflected wave from from 4.5\textdegree{} down-sloap. $h\_s/\lambda\_d =0.05$, $(ka)\_d = 0.05$.}
\triIm{downStrepRefl2degDeep_H0p1_1p0_ka0p05_nw3_np3}{Reflected wave from from 2.0\textdegree{} down-sloap. $h\_s/\lambda\_d =0.05$, $(ka)\_d = 0.05$.}
\triIm{downStrepRefl1degDeep_H0p1_1p0_ka0p05_nw3_np3}{Reflected wave from from 1.0\textdegree{} down-sloap. $h\_s/\lambda\_d =0.05$, $(ka)\_d = 0.05$.}



%\end{document}




\section{Spurious waves generated by wave packets propagating over a depth transition}

\subsection{Step transitions}
\label{sec:packetStep}
\input{packetStep.tex} 

\subsection{Sloping transitions}
\label{sec:packetSlope}
 %\documentclass[a4paper,12pt]{article}
%%\documentclass{amsart}
%\usepackage[a4paper, total={17cm, 25cm}]{geometry}
%
%\usepackage[utf8]{inputenc}
\usepackage[english]{babel}
%\usepackage{amsmath,bm,amsfonts,amssymb}
\usepackage{xcolor}
\usepackage{graphicx}
\usepackage{graphbox} % allows includegraphics[align=c]
\usepackage[round]{natbib}
\usepackage{mathtools}
\usepackage[font=normalsize,margin=2mm]{subfig}
\usepackage{float}
\usepackage{hyperref}
\usepackage{xfrac}
%\usepackage{cprotect}%for \verb in captions
%\usepackage{enumerate}
\usepackage{enumitem}
\hypersetup{colorlinks=true,
			linkcolor=blue,
			filecolor=blue,
			urlcolor=blue,
			citecolor=blue}
\newcommand{\mr}{\mathrm}
\newcommand{\mc}{\mathcal}
\let\SSS\S
\renewcommand{\S}{^\mr{S}}
\newcommand{\ii}{\mr{i}\,}
\newcommand{\ee}{\mr{e}}
\let\underscore\_
\renewcommand{\_}[1]{_\mr{#1}}
\newcommand{\oo}[1]{^{(#1)}}
\let\Re\relax
\let\Im\relax
\DeclareMathOperator\Re{Re}
\DeclareMathOperator\Im{Im}
\newcommand{\w}{w}
\newcommand{\bU}{\bm U}
\newcommand{\h}{\hat}
\newcommand{\br}[3]{\left#1#2\right#3}
\newcommand{\rbr}[1]{\left(#1\right)}
\newcommand{\sbr}[1]{\left[#1\right]}
\newcommand{\cbr}[1]{\left\{#1\right\}}
%\newcommand{\bU}{(\nabla\Phi)_{z=\eta}}

\usepackage{pifont}% http://ctan.org/pkg/pifont
\newcommand{\cmark}{\text{\ding{51}}}% or \checkmark
\newcommand{\xmark}{\text{\ding{55}}}%

\newcommand{\z}{z}
\newcommand{\x}{x}
\newcommand{\y}{y}
\newcommand{\zz}{\zeta}
\newcommand{\xx}{\xi}
\newcommand{\yy}{\sigma}
\newcommand{\kk}{\kappa}

\newcommand{\zmap}{f}
%\newcommand{\zzmap}{\zmap^{-1}}
\newcommand{\zzmap}{\zmap^{\raisebox{.2ex}{$\scriptscriptstyle-1$}}}

%\newcommand{\ww}{w}
%\renewcommand{\w}{\ww^\mr{P}}
\newcommand{\ww}{\omega}
\renewcommand{\w}{w}

\newcommand{\surf}{\eta}
%\newcommand{\w}{\varpi}
\newcommand{\dd}{\mr d}
\newcommand{\ddfrac}[2]{\frac{\dd #1}{\dd #2}}

%\begin{document}
%

%Results for bethymetries of a linear transition in water depth are presented here.
%Conformal maps are integrated numerically, as described in \autoref{sec:SCnum}.
%Results from the numerical map integration method have of course been benchmarked up against the algebraic map results of the previous section.
%\\

Maps of slopes $\theta=\pi/4$, $\pi/20$ and $\pi/40$ are shown in \autoref{fig:res:map_slope} with corresponding simulations displayed in \autoref{fig:res:slope1} to \ref{fig:res:slope3}, all with $H\_s/H\_d = 0.5$.
Compared with a step transition (\autoref{fig:res:logstrip2}), very little difference is noted in terms of spurious waves  with 45\textdegree slope (\autoref{fig:res:slope1}). 
Reducing the slope to $\theta=\pi/20$ (or 9\textdegree) reduces the amplitude of the linear reflected wave packet, as shown in \autoref{fig:res:slope2}. 
No notable effect is seen on the second-order transmitted free wave packet.
Further reducing the slope to $\theta=\pi/40$ (or 4.5\textdegree) yields some reduction also in the amplitude of this spurious transmitted packet.
Notice however that the spurious packets are wider than the carrier packet.



\begin{figure}[H]%
\centering
\subfloat[$\theta=\pi/4$.]{\includegraphics[width=.33\columnwidth]{../HOS_bathymetry/figures/map/mapZoom_SSGW_ka0p05_H1p00_0p50_nH2_ang1_0p5_Nw60.pdf}}%
\subfloat[$\theta=\pi/20$.]{\includegraphics[width=.33\columnwidth]{../HOS_bathymetry/figures/map/mapZoom_SSGW_ka0p05_H1p00_0p50_nH2_ang1_0p1_Nw60.pdf}}%
\subfloat[$\theta=\pi/40$.]{\includegraphics[width=.33\columnwidth]{../HOS_bathymetry/figures/map/mapZoom_SSGW_ka0p05_H1p00_0p50_nH2_ang1_0p05_Nw60.pdf}}%
\caption{Conformal map, single slope; $H\_d = 1.0$, $H\_s = 0.5$}%
\label{fig:res:map_slope}%
\end{figure}

\begin{figure}[H]%
\centering
\includegraphics[width=1\columnwidth]{../HOS_bathymetry/figures/SSGW_ka0p05_M5_H1p00_0p50_nH2_ang1_0p5_Nw60_dt5T_nx3840_pad0_ikCutInf_Md0p5_r0p25.pdf}%
\caption{Surface elevation, $(ka)_0 = 0.05$, $(kH)_0 = 1.00$ ($H\_d=1.0$\,m corresponds to $T\approx2.30$\,s); single slope with $\theta=\pi/4$ (45\textdegree); $H\_s/H\_d = 0.5$. Dashed lines indicate $\x$-location of slope transition beginning and end.}%
\label{fig:res:slope1}%
\end{figure}
\begin{figure}[H]%
\centering 
\includegraphics[width=1\columnwidth]{../HOS_bathymetry/figures/SSGW_ka0p05_M5_H1p00_0p50_nH2_ang1_0p1_Nw60_dt5T_nx3840_pad0_ikCutInf_Md0p5_r0p25.pdf}%
\caption{Similar to \autoref{fig:res:slope1}, but with $\theta=\pi/20$  (9\textdegree).}%
\label{fig:res:slope2}%
\end{figure}
\begin{figure}[H]%
\centering 															  
\includegraphics[width=1\columnwidth]{../HOS_bathymetry/figures/SSGW_ka0p05_M5_H1p00_0p50_nH2_ang1_0p05_Nw60_dt5T_nx3840_pad0_ikCutInf_Md0p5_r0p25.pdf}%
\caption{Similar to \autoref{fig:res:slope1}, but with $\theta=\pi/40$ (4.5\textdegree).}%
\label{fig:res:slope3}%
\end{figure}






%\big|\zeta_{0T,f}^{(22,0)}\big| &= \frac{2\omega_0}{g}|T_{20}| A_0^2.


%\end{document}


\section{Effect of depth transition on wave spectra}
\label{sec:spectra}

 %\documentclass[a4paper,12pt]{article}
%%\documentclass{amsart}
%\usepackage[a4paper, total={17cm, 25cm}]{geometry}
%
%\usepackage[utf8]{inputenc}
\usepackage[english]{babel}
%\usepackage{amsmath,bm,amsfonts,amssymb}
\usepackage{xcolor}
\usepackage{graphicx}
\usepackage{graphbox} % allows includegraphics[align=c]
\usepackage[round]{natbib}
\usepackage{mathtools}
\usepackage[font=normalsize,margin=2mm]{subfig}
\usepackage{float}
\usepackage{hyperref}
\usepackage{xfrac}
%\usepackage{cprotect}%for \verb in captions
%\usepackage{enumerate}
\usepackage{enumitem}
\hypersetup{colorlinks=true,
			linkcolor=blue,
			filecolor=blue,
			urlcolor=blue,
			citecolor=blue}
\newcommand{\mr}{\mathrm}
\newcommand{\mc}{\mathcal}
\let\SSS\S
\renewcommand{\S}{^\mr{S}}
\newcommand{\ii}{\mr{i}\,}
\newcommand{\ee}{\mr{e}}
\let\underscore\_
\renewcommand{\_}[1]{_\mr{#1}}
\newcommand{\oo}[1]{^{(#1)}}
\let\Re\relax
\let\Im\relax
\DeclareMathOperator\Re{Re}
\DeclareMathOperator\Im{Im}
\newcommand{\w}{w}
\newcommand{\bU}{\bm U}
\newcommand{\h}{\hat}
\newcommand{\br}[3]{\left#1#2\right#3}
\newcommand{\rbr}[1]{\left(#1\right)}
\newcommand{\sbr}[1]{\left[#1\right]}
\newcommand{\cbr}[1]{\left\{#1\right\}}
%\newcommand{\bU}{(\nabla\Phi)_{z=\eta}}

\usepackage{pifont}% http://ctan.org/pkg/pifont
\newcommand{\cmark}{\text{\ding{51}}}% or \checkmark
\newcommand{\xmark}{\text{\ding{55}}}%

\newcommand{\z}{z}
\newcommand{\x}{x}
\newcommand{\y}{y}
\newcommand{\zz}{\zeta}
\newcommand{\xx}{\xi}
\newcommand{\yy}{\sigma}
\newcommand{\kk}{\kappa}

\newcommand{\zmap}{f}
%\newcommand{\zzmap}{\zmap^{-1}}
\newcommand{\zzmap}{\zmap^{\raisebox{.2ex}{$\scriptscriptstyle-1$}}}

%\newcommand{\ww}{w}
%\renewcommand{\w}{\ww^\mr{P}}
\newcommand{\ww}{\omega}
\renewcommand{\w}{w}

\newcommand{\surf}{\eta}
%\newcommand{\w}{\varpi}
\newcommand{\dd}{\mr d}
\newcommand{\ddfrac}[2]{\frac{\dd #1}{\dd #2}}

%
%\begin{document}
%



We now move on to irregular waves.
The new HOS code has been extended with options for a closed domain, a wavemaker boundary and a numerical beach. 
This allows for simulations of a continuously incoming wave filed that is absorbed on the far end of the domain. 
The wavemaker boundary is linear and can so be imprecise when compared to nonlinear wavemaker kinematics. 
Wavemaker inaccuracies are however of little relevance since we are interested in the spatial evolution of whatever energy spectrum will be generated. 
An example of the spatial surface elevation during several stages of simulation is shown in \autoref{fig:ts:flat}.
See \citet{SFo2018_HOS,bonnefoy2010,bonnefoy2006A_BM,ducrozet2006_BM} for details on numerical wavemaker and beach.
\\

%\begin{figure}[H]%
%\centering
%\includegraphics[width=.5\columnwidth]{../HOS_bathymetry/figures/map/mapZoom_81300_linear_ka0_H3p00_0p50_theta90_Nw1.pdf}%
%\caption{Bathymetry: Map of step form depth 3 meters to 0.5 meters. Wavemaker at $x=-25$\,m, beach covers $x=8.3$ to 25.0\,m.}%
%\label{fig:map:step}%
%\end{figure}



\begin{figure}[H]%
\centering
\includegraphics[width=.9\columnwidth]{../HOS_bathymetry/figures/81300_closed_linear_T2p50_ka0_M5_H0p50_theta_Nw1_dt1T_nx1024_pad0_ikCut256_Md0_r0.pdf}%
\caption{Time record of surface elevation for flat bed simulation (0.5 meter depth).}%
\label{fig:ts:flat}%
\end{figure}

%\begin{figure}[H]%
%\centering
%\subfloat[Flat bed]{
%\includegraphics[width=.33\columnwidth]{../HOS_bathymetry/figures/S_81300_closed_linear_T2p50_ka0_M5_H0p50_theta_Nw1_dt1T_nx1024_pad0_ikCut256_Md0_r0.pdf}}%
%\subfloat[Bathymetry step, wavemaker signal attuned to depth 0.5\,m]{
%\includegraphics[width=.33\columnwidth]{../HOS_bathymetry/figures/S_81300_closed_linear_T2p50_ka0_M5_H3p00_0p50_theta90_Nw1_dt1T_nx1024_pad0_ikCut256_Md0_r0.pdf}}%
%\subfloat[Bathymetry step, wavemaker signal attuned to depth 3.0\,m]{
%\includegraphics[width=.33\columnwidth]{../HOS_bathymetry/figures/S_81500_closed_linear_T2p50_ka0_M5_H3p00_0p50_theta90_Nw1_dt1T_nx1024_pad0_ikCut256_Md0_r0.pdf}}%
%\label{fig:S}%
%\end{figure}

%\begin{figure}[H]%
%\centering
%\subfloat[Flat bed]{
%\includegraphics[width=.5\columnwidth]{../HOS_bathymetry/figures/S_81300_closed_linear_T2p50_ka0_M5_H0p50_theta_Nw1_dt1T_nx1024_pad0_ikCut256_Md0_r0.pdf}}%
%\subfloat[Bathymetry step]{
%\includegraphics[width=.5\columnwidth]{../HOS_bathymetry/figures/S_81500_closed_linear_T2p50_ka0_M5_H3p00_0p50_theta90_Nw1_dt1T_nx1024_pad0_ikCut256_Md0_r0.pdf}}%
%\caption{Power spectra measured at in the domain centre, at $x=0$\,m.
%$T\_p=2.0$\,s, $H\_s=0.075$\,m.}
%\label{fig:S}%
%\end{figure}



%\begin{figure}[H]%
%\centering
%\subfloat[Flat bed]{
%\includegraphics[width=.5\columnwidth]{../HOS_bathymetry/figures/S_81200_closed_M5_H0p50_theta_Nw1_nx1536_L75_Lb50_pad0_ikCut384_Md0_r0.pdf}}%
%\subfloat[Bathymetry step]{
%\includegraphics[width=.5\columnwidth]{../HOS_bathymetry/figures/S_81600_closed_M5_H3p00_0p50_theta90_Nw1_nx1536_L75_Lb50_pad0_ikCut384_Md0_r0.pdf}}%
%\caption{Power spectra measured at in the domain centre, at $x=0$\,m.
%$T\_p=2.0$\,s, $H\_s=0.125$\,m.}
%\label{fig:S}%
%\end{figure}



%% THESE MAY BE USED, BUT COMBINE WITH OTHER SLOPES FIRST!
%\begin{figure}[H]%
%\centering
%\subfloat[$x=5.0$\,m.]{
%\includegraphics[width=.5\columnwidth]{../HOS_bathymetry/figures/S_5_83010_closed_M5_H0p50_theta_Nw1_nx2048_L50_Lb16p7_pad0_ikCut512_Md0_r0.pdf}}%
%\subfloat[$x=10.0$\,m.]{
%\includegraphics[width=.5\columnwidth]{../HOS_bathymetry/figures/S_10_83010_closed_M5_H0p50_theta_Nw1_nx2048_L50_Lb16p7_pad0_ikCut512_Md0_r0.pdf}}\\
%\subfloat[$x=15.0$\,m.]{
%\includegraphics[width=.5\columnwidth]{../HOS_bathymetry/figures/S_15_83010_closed_M5_H0p50_theta_Nw1_nx2048_L50_Lb16p7_pad0_ikCut512_Md0_r0.pdf}}%
%\subfloat[$x=20.0$\,m.]{
%\includegraphics[width=.5\columnwidth]{../HOS_bathymetry/figures/S_20_83010_closed_M5_H0p50_theta_Nw1_nx2048_L50_Lb16p7_pad0_ikCut512_Md0_r0.pdf}}%
%\caption{Power spectra measured at various locations. Flat bottom.
%$T\_p=2.0$\,s, $H\_s=0.100$\,m. (Wave breaking/simulation crash with $H\_s=0.125$\,m.)}
%\label{fig:S}%
%\end{figure}

%\begin{figure}[H]%
%\centering
%\subfloat[$x=5.0$\,m.]{
%\includegraphics[width=.5\columnwidth]{../HOS_bathymetry/figures/S_5_82000_closed_M5_H3p00_0p50_theta90_Nw1_nx2048_L50_Lb16p7_pad0_ikCut512_Md0_r0.pdf}}%
%\subfloat[$x=10.0$\,m.]{
%\includegraphics[width=.5\columnwidth]{../HOS_bathymetry/figures/S_10_82000_closed_M5_H3p00_0p50_theta90_Nw1_nx2048_L50_Lb16p7_pad0_ikCut512_Md0_r0.pdf}}\\
%\subfloat[$x=15.0$\,m.]{
%\includegraphics[width=.5\columnwidth]{../HOS_bathymetry/figures/S_15_82000_closed_M5_H3p00_0p50_theta90_Nw1_nx2048_L50_Lb16p7_pad0_ikCut512_Md0_r0.pdf}}%
%\subfloat[$x=20.0$\,m.]{
%\includegraphics[width=.5\columnwidth]{../HOS_bathymetry/figures/S_20_82000_closed_M5_H3p00_0p50_theta90_Nw1_nx2048_L50_Lb16p7_pad0_ikCut512_Md0_r0.pdf}}%
%\caption{Power spectra measured at various locations. Step discontinuity at $x=12.2$\,m.
%$T\_p=2.0$\,s, $H\_s=0.125$\,m.}
%\label{fig:S}%
%\end{figure}





%\begin{figure}[H]%
%\centering
%\subfloat[$x=5.0$\,m.]{
%\includegraphics[width=.5\columnwidth]{../HOS_bathymetry/figures/S_5_82110_closed_M5_H0p50_theta_Nw1_nx2048_L50_Lb16p7_pad0_ikCut512_Md0_r0.pdf}}%
%\subfloat[$x=10.0$\,m.]{
%\includegraphics[width=.5\columnwidth]{../HOS_bathymetry/figures/S_10_82110_closed_M5_H0p50_theta_Nw1_nx2048_L50_Lb16p7_pad0_ikCut512_Md0_r0.pdf}}\\
%\subfloat[$x=15.0$\,m.]{
%\includegraphics[width=.5\columnwidth]{../HOS_bathymetry/figures/S_15_82110_closed_M5_H0p50_theta_Nw1_nx2048_L50_Lb16p7_pad0_ikCut512_Md0_r0.pdf}}%
%\subfloat[$x=20.0$\,m.]{ 
%\includegraphics[width=.5\columnwidth]{../HOS_bathymetry/figures/S_20_82110_closed_M5_H0p50_theta_Nw1_nx2048_L50_Lb16p7_pad0_ikCut512_Md0_r0.pdf}}%
%\caption{Power spectra measured at various locations. Flat bottom.
%$T\_p=2.5$\,s, $H\_s=0.125$\,m.}
%\label{fig:S}%
%\end{figure}

%\begin{figure}[H]%
%\centering
%\subfloat[$x=5.0$\,m.]{
%\includegraphics[width=.5\columnwidth]{../HOS_bathymetry/figures/S_5_82100_closed_M5_H3p00_0p50_theta90_Nw1_nx2048_L50_Lb16p7_pad0_ikCut512_Md0_r0.pdf}}%
%\subfloat[$x=10.0$\,m.]{
%\includegraphics[width=.5\columnwidth]{../HOS_bathymetry/figures/S_10_82100_closed_M5_H3p00_0p50_theta90_Nw1_nx2048_L50_Lb16p7_pad0_ikCut512_Md0_r0.pdf}}\\
%\subfloat[$x=15.0$\,m.]{
%\includegraphics[width=.5\columnwidth]{../HOS_bathymetry/figures/S_15_82100_closed_M5_H3p00_0p50_theta90_Nw1_nx2048_L50_Lb16p7_pad0_ikCut512_Md0_r0.pdf}}%
%\subfloat[$x=20.0$\,m.]{
%\includegraphics[width=.5\columnwidth]{../HOS_bathymetry/figures/S_20_82100_closed_M5_H3p00_0p50_theta90_Nw1_nx2048_L50_Lb16p7_pad0_ikCut512_Md0_r0.pdf}}%
%\caption{Power spectra measured at various locations. Step discontinuity at $x=12.2$\,m.
%$T\_p=2.5$\,s, $H\_s=0.125$\,m.}
%\label{fig:S}%
%\end{figure}


We now consider a depth transition example where the water depth decreases from 3.0 meters to 0.5 meters some distance form the wavemaker. 
Three transition slopes, $\theta = 90$\textdegree, $45$\textdegree and $15$\textdegree, are compared.
Power spectra measured as various domain locations are shown in \autoref{fig:Tp2p5:slopes}.
A JONSWAP spectrum with $T\_p=2.5$\,s, $H\_s=0.125$\,m is here the wavemaker target.

Energy around the peak period is observed to deviate more from target with abrupt depth transition than with a gradual transition. 
It is possible that this finding is related to the first-order reflection of wave packets observed in \autoref{sec:packetStep}  and  \ref{sec:packetSlope}, with reflected waves bouncing back within the deep water section of the domain and interacting with the wavemaker. 
All transition slops generate a secondary energy peak around the double frequency $2/T\_p$, analogous to the second-order wave packets seen earlier. 
Reduced slope angles do not appear no mitigate this development.
The second order energy peaks diminish further away from the depth transition, likely due to further nonlinear development. 
It is possible that numerical damping effects, for example for the spectral cut-off in the HOS scheme, also influence this development. 
 
For comparison, a similar spectrum from simulation domains with no depth transition is shown in \autoref{fig:Tp2p5:flat}.
These domain have depths of 3.0 and 0.5 meters all over and the wavemaker signal is adjusted accordingly. 
A second order energy peak is seen close to the wavemaker in the shallow case. It diminished with increasing distance. 
It is possibly generated by the wavemaker, both is a physical sense and in the sense of errors from the wavemaker approximation.
Notable low-frequency energy is also observed in the shallow water case, which possesses stronger nonlinearities. 
Again, beach damping of low-frequency waves is inefficient, so these surges may travel beck and forth across the numerical wave tank. 
\\

Similar observations are made in  \autoref{fig:Tp2p0:slopes}\ref{fig:Tp2p0:flat} with the target spectrum $T\_p=2.0$\,s, $H\_s=0.100$\,m.
The bathymetry effect is smaller at this peak period, yet still notable in the energy deficit around 6\,Hz.


\begin{figure}[H]%
\centering
\subfloat[Bathymetries]{
\includegraphics[width=.33\columnwidth]{../HOS_bathymetry/figures/map/mapZoom_82100_linear_ka0_H3p00_0p50_theta90_Nw1.pdf}
\includegraphics[width=.33\columnwidth]{../HOS_bathymetry/figures/map/mapZoom_82000_linear_ka0_H3p00_0p50_theta45_Nw1.pdf}
\includegraphics[width=.33\columnwidth]{../HOS_bathymetry/figures/map/mapZoom_82100_linear_ka0_H3p00_0p50_theta15_Nw1.pdf}
}\\%
\subfloat[$x=0.0$\,m.]{
\includegraphics[width=.5\columnwidth]{../HOS_bathymetry/figures/powerSpec/82100/x_wp0.pdf}}%
\subfloat[$x=5.0$\,m.]{
\includegraphics[width=.5\columnwidth]{../HOS_bathymetry/figures/powerSpec/82100/x_wp5.pdf}}\\
\subfloat[$x=10.0$\,m.]{
\includegraphics[width=.5\columnwidth]{../HOS_bathymetry/figures/powerSpec/82100/x_wp10.pdf}}%
\subfloat[$x=15.0$\,m.]{
\includegraphics[width=.5\columnwidth]{../HOS_bathymetry/figures/powerSpec/82100/x_wp15.pdf}}\\
\subfloat[$x=20.0$\,m.]{
\includegraphics[width=.5\columnwidth]{../HOS_bathymetry/figures/powerSpec/82100/x_wp20.pdf}}%
\subfloat[$x=25.0$\,m.]{
\includegraphics[width=.5\columnwidth]{../HOS_bathymetry/figures/powerSpec/82100/x_wp25.pdf}}%
\caption{Power spectra measured at various locations, comparing transitional ramps.
Depth transition from 3.0 to 0.5 meters.
$T\_p=2.5$\,s, $H\_s=0.125$\,m.}
\label{fig:Tp2p5:slopes}%
\end{figure}

\begin{figure}[H]%
\centering
\subfloat[$x=0.0$\,m.]{
\includegraphics[width=.5\columnwidth]{../HOS_bathymetry/figures/powerSpec/821x0_2xflatOnly/x_wp0.pdf}}%
\subfloat[$x=5.0$\,m.]{
\includegraphics[width=.5\columnwidth]{../HOS_bathymetry/figures/powerSpec/821x0_2xflatOnly/x_wp5.pdf}}\\%
\subfloat[$x=10.0$\,m.]{
\includegraphics[width=.5\columnwidth]{../HOS_bathymetry/figures/powerSpec/821x0_2xflatOnly/x_wp10.pdf}}%
\subfloat[$x=15.0$\,m.]{
\includegraphics[width=.5\columnwidth]{../HOS_bathymetry/figures/powerSpec/821x0_2xflatOnly/x_wp15.pdf}}\\%
\subfloat[$x=20.0$\,m.]{
\includegraphics[width=.5\columnwidth]{../HOS_bathymetry/figures/powerSpec/821x0_2xflatOnly/x_wp20.pdf}}%
\subfloat[$x=25.0$\,m.]{
\includegraphics[width=.5\columnwidth]{../HOS_bathymetry/figures/powerSpec/821x0_2xflatOnly/x_wp25.pdf}}%
\caption{Power spectra measured at various locations, flat bed at 0.5 meters depth.
$T\_p=2.5$\,s, $H\_s=0.125$\,m.}
\label{fig:Tp2p5:flat}%
\end{figure}




%%%%%%%%%%%%%

\begin{figure}[H]%
\centering
\subfloat[$x=0.0$\,m.]{
\includegraphics[width=.5\columnwidth]{../HOS_bathymetry/figures/powerSpec/83000/x_wp0.pdf}}%
\subfloat[$x=5.0$\,m.]{
\includegraphics[width=.5\columnwidth]{../HOS_bathymetry/figures/powerSpec/83000/x_wp5.pdf}}\\
\subfloat[$x=10.0$\,m.]{
\includegraphics[width=.5\columnwidth]{../HOS_bathymetry/figures/powerSpec/83000/x_wp10.pdf}}%
\subfloat[$x=15.0$\,m.]{
\includegraphics[width=.5\columnwidth]{../HOS_bathymetry/figures/powerSpec/83000/x_wp15.pdf}}\\
\subfloat[$x=20.0$\,m.]{
\includegraphics[width=.5\columnwidth]{../HOS_bathymetry/figures/powerSpec/83000/x_wp20.pdf}}%
\subfloat[$x=25.0$\,m.]{
\includegraphics[width=.5\columnwidth]{../HOS_bathymetry/figures/powerSpec/83000/x_wp25.pdf}}%
\caption{Power spectra measured at various locations, comparing transitional ramps.
Depth transition from 3.0 to 0.5 meters.
$T\_p=2.0$\,s, $H\_s=0.100$\,m.}
\label{fig:Tp2p0:slopes}%
\end{figure}

\begin{figure}[H]%
\centering
\subfloat[$x=0.0$\,m.]{
\includegraphics[width=.5\columnwidth]{../HOS_bathymetry/figures/powerSpec/830x0_2xflatOnly/x_wp0.pdf}}%
\subfloat[$x=5.0$\,m.]{
\includegraphics[width=.5\columnwidth]{../HOS_bathymetry/figures/powerSpec/830x0_2xflatOnly/x_wp5.pdf}}\\%
\subfloat[$x=10.0$\,m.]{
\includegraphics[width=.5\columnwidth]{../HOS_bathymetry/figures/powerSpec/830x0_2xflatOnly/x_wp10.pdf}}%
\subfloat[$x=15.0$\,m.]{
\includegraphics[width=.5\columnwidth]{../HOS_bathymetry/figures/powerSpec/830x0_2xflatOnly/x_wp15.pdf}}\\%
\subfloat[$x=20.0$\,m.]{
\includegraphics[width=.5\columnwidth]{../HOS_bathymetry/figures/powerSpec/830x0_2xflatOnly/x_wp20.pdf}}%
\subfloat[$x=25.0$\,m.]{
\includegraphics[width=.5\columnwidth]{../HOS_bathymetry/figures/powerSpec/830x0_2xflatOnly/x_wp25.pdf}}%
\caption{Power spectra measured at various locations, flat bed at 0.5 meters depth.
$T\_p=2.0$\,s, $H\_s=0.100$\,m.}
\label{fig:Tp2p0:flat}%
\end{figure}
 

%\end{document}
 
\section{Conclusions}
\begin{itemize}
	\item Very shallow transitions ($\sim 1\textdegree{}$) are needed to alleviate transmitted contamination from depth transition.
	\item Less shallow transitions ($\sim 10\textdegree{}$) are needed to eliminated reflected waves from depth transition.This slope must however extend deeper than is necessary for the transmitted components. 
	%\item In terms of an elevated floor, reflected waves from the up-step has little effect other than to modulate the 
\end{itemize}


\bibliographystyle{abbrvnat} % abbrvnat,plainnat,unsrtnat
\bibliography{../sintef_bib}


\end{document}
