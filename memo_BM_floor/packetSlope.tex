 %\documentclass[a4paper,12pt]{article}
%%\documentclass{amsart}
%\usepackage[a4paper, total={17cm, 25cm}]{geometry}
%
%\usepackage[utf8]{inputenc}
\usepackage[english]{babel}
%\usepackage{amsmath,bm,amsfonts,amssymb}
\usepackage{xcolor}
\usepackage{graphicx}
\usepackage{graphbox} % allows includegraphics[align=c]
\usepackage[round]{natbib}
\usepackage{mathtools}
\usepackage[font=normalsize,margin=2mm]{subfig}
\usepackage{float}
\usepackage{hyperref}
\usepackage{xfrac}
%\usepackage{cprotect}%for \verb in captions
%\usepackage{enumerate}
\usepackage{enumitem}
\hypersetup{colorlinks=true,
			linkcolor=blue,
			filecolor=blue,
			urlcolor=blue,
			citecolor=blue}
\newcommand{\mr}{\mathrm}
\newcommand{\mc}{\mathcal}
\let\SSS\S
\renewcommand{\S}{^\mr{S}}
\newcommand{\ii}{\mr{i}\,}
\newcommand{\ee}{\mr{e}}
\let\underscore\_
\renewcommand{\_}[1]{_\mr{#1}}
\newcommand{\oo}[1]{^{(#1)}}
\let\Re\relax
\let\Im\relax
\DeclareMathOperator\Re{Re}
\DeclareMathOperator\Im{Im}
\newcommand{\w}{w}
\newcommand{\bU}{\bm U}
\newcommand{\h}{\hat}
\newcommand{\br}[3]{\left#1#2\right#3}
\newcommand{\rbr}[1]{\left(#1\right)}
\newcommand{\sbr}[1]{\left[#1\right]}
\newcommand{\cbr}[1]{\left\{#1\right\}}
%\newcommand{\bU}{(\nabla\Phi)_{z=\eta}}

\usepackage{pifont}% http://ctan.org/pkg/pifont
\newcommand{\cmark}{\text{\ding{51}}}% or \checkmark
\newcommand{\xmark}{\text{\ding{55}}}%

\newcommand{\z}{z}
\newcommand{\x}{x}
\newcommand{\y}{y}
\newcommand{\zz}{\zeta}
\newcommand{\xx}{\xi}
\newcommand{\yy}{\sigma}
\newcommand{\kk}{\kappa}

\newcommand{\zmap}{f}
%\newcommand{\zzmap}{\zmap^{-1}}
\newcommand{\zzmap}{\zmap^{\raisebox{.2ex}{$\scriptscriptstyle-1$}}}

%\newcommand{\ww}{w}
%\renewcommand{\w}{\ww^\mr{P}}
\newcommand{\ww}{\omega}
\renewcommand{\w}{w}

\newcommand{\surf}{\eta}
%\newcommand{\w}{\varpi}
\newcommand{\dd}{\mr d}
\newcommand{\ddfrac}[2]{\frac{\dd #1}{\dd #2}}

%\begin{document}
%

%Results for bethymetries of a linear transition in water depth are presented here.
%Conformal maps are integrated numerically, as described in \autoref{sec:SCnum}.
%Results from the numerical map integration method have of course been benchmarked up against the algebraic map results of the previous section.
%\\

Maps of slopes $\theta=\pi/4$, $\pi/20$ and $\pi/40$ are shown in \autoref{fig:res:map_slope} with corresponding simulations displayed in \autoref{fig:res:slope1} to \ref{fig:res:slope3}, all with $H\_s/H\_d = 0.5$.
Compared with a step transition (\autoref{fig:res:logstrip2}), very little difference is noted in terms of spurious waves  with 45\textdegree slope (\autoref{fig:res:slope1}). 
Reducing the slope to $\theta=\pi/20$ (or 9\textdegree) reduces the amplitude of the linear reflected wave packet, as shown in \autoref{fig:res:slope2}. 
No notable effect is seen on the second-order transmitted free wave packet.
Further reducing the slope to $\theta=\pi/40$ (or 4.5\textdegree) yields some reduction also in the amplitude of this spurious transmitted packet.
Notice however that the spurious packets are wider than the carrier packet.



\begin{figure}[H]%
\centering
\subfloat[$\theta=\pi/4$.]{\includegraphics[width=.33\columnwidth]{../HOS_bathymetry/figures/map/mapZoom_SSGW_ka0p05_H1p00_0p50_nH2_ang1_0p5_Nw60.pdf}}%
\subfloat[$\theta=\pi/20$.]{\includegraphics[width=.33\columnwidth]{../HOS_bathymetry/figures/map/mapZoom_SSGW_ka0p05_H1p00_0p50_nH2_ang1_0p1_Nw60.pdf}}%
\subfloat[$\theta=\pi/40$.]{\includegraphics[width=.33\columnwidth]{../HOS_bathymetry/figures/map/mapZoom_SSGW_ka0p05_H1p00_0p50_nH2_ang1_0p05_Nw60.pdf}}%
\caption{Conformal map, single slope; $H\_d = 1.0$, $H\_s = 0.5$}%
\label{fig:res:map_slope}%
\end{figure}

\begin{figure}[H]%
\centering
\includegraphics[width=1\columnwidth]{../HOS_bathymetry/figures/SSGW_ka0p05_M5_H1p00_0p50_nH2_ang1_0p5_Nw60_dt5T_nx3840_pad0_ikCutInf_Md0p5_r0p25.pdf}%
\caption{Surface elevation, $(ka)_0 = 0.05$, $(kH)_0 = 1.00$ ($H\_d=1.0$\,m corresponds to $T\approx2.30$\,s); single slope with $\theta=\pi/4$ (45\textdegree); $H\_s/H\_d = 0.5$. Dashed lines indicate $\x$-location of slope transition beginning and end.}%
\label{fig:res:slope1}%
\end{figure}
\begin{figure}[H]%
\centering 
\includegraphics[width=1\columnwidth]{../HOS_bathymetry/figures/SSGW_ka0p05_M5_H1p00_0p50_nH2_ang1_0p1_Nw60_dt5T_nx3840_pad0_ikCutInf_Md0p5_r0p25.pdf}%
\caption{Similar to \autoref{fig:res:slope1}, but with $\theta=\pi/20$  (9\textdegree).}%
\label{fig:res:slope2}%
\end{figure}
\begin{figure}[H]%
\centering 															  
\includegraphics[width=1\columnwidth]{../HOS_bathymetry/figures/SSGW_ka0p05_M5_H1p00_0p50_nH2_ang1_0p05_Nw60_dt5T_nx3840_pad0_ikCutInf_Md0p5_r0p25.pdf}%
\caption{Similar to \autoref{fig:res:slope1}, but with $\theta=\pi/40$ (4.5\textdegree).}%
\label{fig:res:slope3}%
\end{figure}






%\big|\zeta_{0T,f}^{(22,0)}\big| &= \frac{2\omega_0}{g}|T_{20}| A_0^2.


%\end{document}