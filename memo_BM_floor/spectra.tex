
 %\documentclass[a4paper,12pt]{article}
%%\documentclass{amsart}
%\usepackage[a4paper, total={17cm, 25cm}]{geometry}
%
%\usepackage[utf8]{inputenc}
\usepackage[english]{babel}
%\usepackage{amsmath,bm,amsfonts,amssymb}
\usepackage{xcolor}
\usepackage{graphicx}
\usepackage{graphbox} % allows includegraphics[align=c]
\usepackage[round]{natbib}
\usepackage{mathtools}
\usepackage[font=normalsize,margin=2mm]{subfig}
\usepackage{float}
\usepackage{hyperref}
\usepackage{xfrac}
%\usepackage{cprotect}%for \verb in captions
%\usepackage{enumerate}
\usepackage{enumitem}
\hypersetup{colorlinks=true,
			linkcolor=blue,
			filecolor=blue,
			urlcolor=blue,
			citecolor=blue}
\newcommand{\mr}{\mathrm}
\newcommand{\mc}{\mathcal}
\let\SSS\S
\renewcommand{\S}{^\mr{S}}
\newcommand{\ii}{\mr{i}\,}
\newcommand{\ee}{\mr{e}}
\let\underscore\_
\renewcommand{\_}[1]{_\mr{#1}}
\newcommand{\oo}[1]{^{(#1)}}
\let\Re\relax
\let\Im\relax
\DeclareMathOperator\Re{Re}
\DeclareMathOperator\Im{Im}
\newcommand{\w}{w}
\newcommand{\bU}{\bm U}
\newcommand{\h}{\hat}
\newcommand{\br}[3]{\left#1#2\right#3}
\newcommand{\rbr}[1]{\left(#1\right)}
\newcommand{\sbr}[1]{\left[#1\right]}
\newcommand{\cbr}[1]{\left\{#1\right\}}
%\newcommand{\bU}{(\nabla\Phi)_{z=\eta}}

\usepackage{pifont}% http://ctan.org/pkg/pifont
\newcommand{\cmark}{\text{\ding{51}}}% or \checkmark
\newcommand{\xmark}{\text{\ding{55}}}%

\newcommand{\z}{z}
\newcommand{\x}{x}
\newcommand{\y}{y}
\newcommand{\zz}{\zeta}
\newcommand{\xx}{\xi}
\newcommand{\yy}{\sigma}
\newcommand{\kk}{\kappa}

\newcommand{\zmap}{f}
%\newcommand{\zzmap}{\zmap^{-1}}
\newcommand{\zzmap}{\zmap^{\raisebox{.2ex}{$\scriptscriptstyle-1$}}}

%\newcommand{\ww}{w}
%\renewcommand{\w}{\ww^\mr{P}}
\newcommand{\ww}{\omega}
\renewcommand{\w}{w}

\newcommand{\surf}{\eta}
%\newcommand{\w}{\varpi}
\newcommand{\dd}{\mr d}
\newcommand{\ddfrac}[2]{\frac{\dd #1}{\dd #2}}

%
%\begin{document}
%



We now move on to irregular waves.
The new HOS code has been extended with options for a closed domain, a wavemaker boundary and a numerical beach. 
This allows for simulations of a continuously incoming wave filed that is absorbed on the far end of the domain. 
The wavemaker boundary is linear and can so be imprecise when compared to nonlinear wavemaker kinematics. 
Wavemaker inaccuracies are however of little relevance since we are interested in the spatial evolution of whatever energy spectrum will be generated. 
An example of the spatial surface elevation during several stages of simulation is shown in \autoref{fig:ts:flat}.
See \citet{SFo2018_HOS,bonnefoy2010,bonnefoy2006A_BM,ducrozet2006_BM} for details on numerical wavemaker and beach.
\\

%\begin{figure}[H]%
%\centering
%\includegraphics[width=.5\columnwidth]{../HOS_bathymetry/figures/map/mapZoom_81300_linear_ka0_H3p00_0p50_theta90_Nw1.pdf}%
%\caption{Bathymetry: Map of step form depth 3 meters to 0.5 meters. Wavemaker at $x=-25$\,m, beach covers $x=8.3$ to 25.0\,m.}%
%\label{fig:map:step}%
%\end{figure}



\begin{figure}[H]%
\centering
\includegraphics[width=.9\columnwidth]{../HOS_bathymetry/figures/81300_closed_linear_T2p50_ka0_M5_H0p50_theta_Nw1_dt1T_nx1024_pad0_ikCut256_Md0_r0.pdf}%
\caption{Time record of surface elevation for flat bed simulation (0.5 meter depth).}%
\label{fig:ts:flat}%
\end{figure}

%\begin{figure}[H]%
%\centering
%\subfloat[Flat bed]{
%\includegraphics[width=.33\columnwidth]{../HOS_bathymetry/figures/S_81300_closed_linear_T2p50_ka0_M5_H0p50_theta_Nw1_dt1T_nx1024_pad0_ikCut256_Md0_r0.pdf}}%
%\subfloat[Bathymetry step, wavemaker signal attuned to depth 0.5\,m]{
%\includegraphics[width=.33\columnwidth]{../HOS_bathymetry/figures/S_81300_closed_linear_T2p50_ka0_M5_H3p00_0p50_theta90_Nw1_dt1T_nx1024_pad0_ikCut256_Md0_r0.pdf}}%
%\subfloat[Bathymetry step, wavemaker signal attuned to depth 3.0\,m]{
%\includegraphics[width=.33\columnwidth]{../HOS_bathymetry/figures/S_81500_closed_linear_T2p50_ka0_M5_H3p00_0p50_theta90_Nw1_dt1T_nx1024_pad0_ikCut256_Md0_r0.pdf}}%
%\label{fig:S}%
%\end{figure}

%\begin{figure}[H]%
%\centering
%\subfloat[Flat bed]{
%\includegraphics[width=.5\columnwidth]{../HOS_bathymetry/figures/S_81300_closed_linear_T2p50_ka0_M5_H0p50_theta_Nw1_dt1T_nx1024_pad0_ikCut256_Md0_r0.pdf}}%
%\subfloat[Bathymetry step]{
%\includegraphics[width=.5\columnwidth]{../HOS_bathymetry/figures/S_81500_closed_linear_T2p50_ka0_M5_H3p00_0p50_theta90_Nw1_dt1T_nx1024_pad0_ikCut256_Md0_r0.pdf}}%
%\caption{Power spectra measured at in the domain centre, at $x=0$\,m.
%$T\_p=2.0$\,s, $H\_s=0.075$\,m.}
%\label{fig:S}%
%\end{figure}



%\begin{figure}[H]%
%\centering
%\subfloat[Flat bed]{
%\includegraphics[width=.5\columnwidth]{../HOS_bathymetry/figures/S_81200_closed_M5_H0p50_theta_Nw1_nx1536_L75_Lb50_pad0_ikCut384_Md0_r0.pdf}}%
%\subfloat[Bathymetry step]{
%\includegraphics[width=.5\columnwidth]{../HOS_bathymetry/figures/S_81600_closed_M5_H3p00_0p50_theta90_Nw1_nx1536_L75_Lb50_pad0_ikCut384_Md0_r0.pdf}}%
%\caption{Power spectra measured at in the domain centre, at $x=0$\,m.
%$T\_p=2.0$\,s, $H\_s=0.125$\,m.}
%\label{fig:S}%
%\end{figure}



%% THESE MAY BE USED, BUT COMBINE WITH OTHER SLOPES FIRST!
%\begin{figure}[H]%
%\centering
%\subfloat[$x=5.0$\,m.]{
%\includegraphics[width=.5\columnwidth]{../HOS_bathymetry/figures/S_5_83010_closed_M5_H0p50_theta_Nw1_nx2048_L50_Lb16p7_pad0_ikCut512_Md0_r0.pdf}}%
%\subfloat[$x=10.0$\,m.]{
%\includegraphics[width=.5\columnwidth]{../HOS_bathymetry/figures/S_10_83010_closed_M5_H0p50_theta_Nw1_nx2048_L50_Lb16p7_pad0_ikCut512_Md0_r0.pdf}}\\
%\subfloat[$x=15.0$\,m.]{
%\includegraphics[width=.5\columnwidth]{../HOS_bathymetry/figures/S_15_83010_closed_M5_H0p50_theta_Nw1_nx2048_L50_Lb16p7_pad0_ikCut512_Md0_r0.pdf}}%
%\subfloat[$x=20.0$\,m.]{
%\includegraphics[width=.5\columnwidth]{../HOS_bathymetry/figures/S_20_83010_closed_M5_H0p50_theta_Nw1_nx2048_L50_Lb16p7_pad0_ikCut512_Md0_r0.pdf}}%
%\caption{Power spectra measured at various locations. Flat bottom.
%$T\_p=2.0$\,s, $H\_s=0.100$\,m. (Wave breaking/simulation crash with $H\_s=0.125$\,m.)}
%\label{fig:S}%
%\end{figure}

%\begin{figure}[H]%
%\centering
%\subfloat[$x=5.0$\,m.]{
%\includegraphics[width=.5\columnwidth]{../HOS_bathymetry/figures/S_5_82000_closed_M5_H3p00_0p50_theta90_Nw1_nx2048_L50_Lb16p7_pad0_ikCut512_Md0_r0.pdf}}%
%\subfloat[$x=10.0$\,m.]{
%\includegraphics[width=.5\columnwidth]{../HOS_bathymetry/figures/S_10_82000_closed_M5_H3p00_0p50_theta90_Nw1_nx2048_L50_Lb16p7_pad0_ikCut512_Md0_r0.pdf}}\\
%\subfloat[$x=15.0$\,m.]{
%\includegraphics[width=.5\columnwidth]{../HOS_bathymetry/figures/S_15_82000_closed_M5_H3p00_0p50_theta90_Nw1_nx2048_L50_Lb16p7_pad0_ikCut512_Md0_r0.pdf}}%
%\subfloat[$x=20.0$\,m.]{
%\includegraphics[width=.5\columnwidth]{../HOS_bathymetry/figures/S_20_82000_closed_M5_H3p00_0p50_theta90_Nw1_nx2048_L50_Lb16p7_pad0_ikCut512_Md0_r0.pdf}}%
%\caption{Power spectra measured at various locations. Step discontinuity at $x=12.2$\,m.
%$T\_p=2.0$\,s, $H\_s=0.125$\,m.}
%\label{fig:S}%
%\end{figure}





%\begin{figure}[H]%
%\centering
%\subfloat[$x=5.0$\,m.]{
%\includegraphics[width=.5\columnwidth]{../HOS_bathymetry/figures/S_5_82110_closed_M5_H0p50_theta_Nw1_nx2048_L50_Lb16p7_pad0_ikCut512_Md0_r0.pdf}}%
%\subfloat[$x=10.0$\,m.]{
%\includegraphics[width=.5\columnwidth]{../HOS_bathymetry/figures/S_10_82110_closed_M5_H0p50_theta_Nw1_nx2048_L50_Lb16p7_pad0_ikCut512_Md0_r0.pdf}}\\
%\subfloat[$x=15.0$\,m.]{
%\includegraphics[width=.5\columnwidth]{../HOS_bathymetry/figures/S_15_82110_closed_M5_H0p50_theta_Nw1_nx2048_L50_Lb16p7_pad0_ikCut512_Md0_r0.pdf}}%
%\subfloat[$x=20.0$\,m.]{ 
%\includegraphics[width=.5\columnwidth]{../HOS_bathymetry/figures/S_20_82110_closed_M5_H0p50_theta_Nw1_nx2048_L50_Lb16p7_pad0_ikCut512_Md0_r0.pdf}}%
%\caption{Power spectra measured at various locations. Flat bottom.
%$T\_p=2.5$\,s, $H\_s=0.125$\,m.}
%\label{fig:S}%
%\end{figure}

%\begin{figure}[H]%
%\centering
%\subfloat[$x=5.0$\,m.]{
%\includegraphics[width=.5\columnwidth]{../HOS_bathymetry/figures/S_5_82100_closed_M5_H3p00_0p50_theta90_Nw1_nx2048_L50_Lb16p7_pad0_ikCut512_Md0_r0.pdf}}%
%\subfloat[$x=10.0$\,m.]{
%\includegraphics[width=.5\columnwidth]{../HOS_bathymetry/figures/S_10_82100_closed_M5_H3p00_0p50_theta90_Nw1_nx2048_L50_Lb16p7_pad0_ikCut512_Md0_r0.pdf}}\\
%\subfloat[$x=15.0$\,m.]{
%\includegraphics[width=.5\columnwidth]{../HOS_bathymetry/figures/S_15_82100_closed_M5_H3p00_0p50_theta90_Nw1_nx2048_L50_Lb16p7_pad0_ikCut512_Md0_r0.pdf}}%
%\subfloat[$x=20.0$\,m.]{
%\includegraphics[width=.5\columnwidth]{../HOS_bathymetry/figures/S_20_82100_closed_M5_H3p00_0p50_theta90_Nw1_nx2048_L50_Lb16p7_pad0_ikCut512_Md0_r0.pdf}}%
%\caption{Power spectra measured at various locations. Step discontinuity at $x=12.2$\,m.
%$T\_p=2.5$\,s, $H\_s=0.125$\,m.}
%\label{fig:S}%
%\end{figure}


We now consider a depth transition example where the water depth decreases from 3.0 meters to 0.5 meters some distance form the wavemaker. 
Three transition slopes, $\theta = 90$\textdegree, $45$\textdegree and $15$\textdegree, are compared.
Power spectra measured as various domain locations are shown in \autoref{fig:Tp2p5:slopes}.
A JONSWAP spectrum with $T\_p=2.5$\,s, $H\_s=0.125$\,m is here the wavemaker target.

Energy around the peak period is observed to deviate more from target with abrupt depth transition than with a gradual transition. 
It is possible that this finding is related to the first-order reflection of wave packets observed in \autoref{sec:packetStep}  and  \ref{sec:packetSlope}, with reflected waves bouncing back within the deep water section of the domain and interacting with the wavemaker. 
All transition slops generate a secondary energy peak around the double frequency $2/T\_p$, analogous to the second-order wave packets seen earlier. 
Reduced slope angles do not appear no mitigate this development.
The second order energy peaks diminish further away from the depth transition, likely due to further nonlinear development. 
It is possible that numerical damping effects, for example for the spectral cut-off in the HOS scheme, also influence this development. 
 
For comparison, a similar spectrum from simulation domains with no depth transition is shown in \autoref{fig:Tp2p5:flat}.
These domain have depths of 3.0 and 0.5 meters all over and the wavemaker signal is adjusted accordingly. 
A second order energy peak is seen close to the wavemaker in the shallow case. It diminished with increasing distance. 
It is possibly generated by the wavemaker, both is a physical sense and in the sense of errors from the wavemaker approximation.
Notable low-frequency energy is also observed in the shallow water case, which possesses stronger nonlinearities. 
Again, beach damping of low-frequency waves is inefficient, so these surges may travel beck and forth across the numerical wave tank. 
\\

Similar observations are made in  \autoref{fig:Tp2p0:slopes}\ref{fig:Tp2p0:flat} with the target spectrum $T\_p=2.0$\,s, $H\_s=0.100$\,m.
The bathymetry effect is smaller at this peak period, yet still notable in the energy deficit around 6\,Hz.


\begin{figure}[H]%
\centering
\subfloat[Bathymetries]{
\includegraphics[width=.33\columnwidth]{../HOS_bathymetry/figures/map/mapZoom_82100_linear_ka0_H3p00_0p50_theta90_Nw1.pdf}
\includegraphics[width=.33\columnwidth]{../HOS_bathymetry/figures/map/mapZoom_82000_linear_ka0_H3p00_0p50_theta45_Nw1.pdf}
\includegraphics[width=.33\columnwidth]{../HOS_bathymetry/figures/map/mapZoom_82100_linear_ka0_H3p00_0p50_theta15_Nw1.pdf}
}\\%
\subfloat[$x=0.0$\,m.]{
\includegraphics[width=.5\columnwidth]{../HOS_bathymetry/figures/powerSpec/82100/x_wp0.pdf}}%
\subfloat[$x=5.0$\,m.]{
\includegraphics[width=.5\columnwidth]{../HOS_bathymetry/figures/powerSpec/82100/x_wp5.pdf}}\\
\subfloat[$x=10.0$\,m.]{
\includegraphics[width=.5\columnwidth]{../HOS_bathymetry/figures/powerSpec/82100/x_wp10.pdf}}%
\subfloat[$x=15.0$\,m.]{
\includegraphics[width=.5\columnwidth]{../HOS_bathymetry/figures/powerSpec/82100/x_wp15.pdf}}\\
\subfloat[$x=20.0$\,m.]{
\includegraphics[width=.5\columnwidth]{../HOS_bathymetry/figures/powerSpec/82100/x_wp20.pdf}}%
\subfloat[$x=25.0$\,m.]{
\includegraphics[width=.5\columnwidth]{../HOS_bathymetry/figures/powerSpec/82100/x_wp25.pdf}}%
\caption{Power spectra measured at various locations, comparing transitional ramps.
Depth transition from 3.0 to 0.5 meters.
$T\_p=2.5$\,s, $H\_s=0.125$\,m.}
\label{fig:Tp2p5:slopes}%
\end{figure}

\begin{figure}[H]%
\centering
\subfloat[$x=0.0$\,m.]{
\includegraphics[width=.5\columnwidth]{../HOS_bathymetry/figures/powerSpec/821x0_2xflatOnly/x_wp0.pdf}}%
\subfloat[$x=5.0$\,m.]{
\includegraphics[width=.5\columnwidth]{../HOS_bathymetry/figures/powerSpec/821x0_2xflatOnly/x_wp5.pdf}}\\%
\subfloat[$x=10.0$\,m.]{
\includegraphics[width=.5\columnwidth]{../HOS_bathymetry/figures/powerSpec/821x0_2xflatOnly/x_wp10.pdf}}%
\subfloat[$x=15.0$\,m.]{
\includegraphics[width=.5\columnwidth]{../HOS_bathymetry/figures/powerSpec/821x0_2xflatOnly/x_wp15.pdf}}\\%
\subfloat[$x=20.0$\,m.]{
\includegraphics[width=.5\columnwidth]{../HOS_bathymetry/figures/powerSpec/821x0_2xflatOnly/x_wp20.pdf}}%
\subfloat[$x=25.0$\,m.]{
\includegraphics[width=.5\columnwidth]{../HOS_bathymetry/figures/powerSpec/821x0_2xflatOnly/x_wp25.pdf}}%
\caption{Power spectra measured at various locations, flat bed at 0.5 meters depth.
$T\_p=2.5$\,s, $H\_s=0.125$\,m.}
\label{fig:Tp2p5:flat}%
\end{figure}




%%%%%%%%%%%%%

\begin{figure}[H]%
\centering
\subfloat[$x=0.0$\,m.]{
\includegraphics[width=.5\columnwidth]{../HOS_bathymetry/figures/powerSpec/83000/x_wp0.pdf}}%
\subfloat[$x=5.0$\,m.]{
\includegraphics[width=.5\columnwidth]{../HOS_bathymetry/figures/powerSpec/83000/x_wp5.pdf}}\\
\subfloat[$x=10.0$\,m.]{
\includegraphics[width=.5\columnwidth]{../HOS_bathymetry/figures/powerSpec/83000/x_wp10.pdf}}%
\subfloat[$x=15.0$\,m.]{
\includegraphics[width=.5\columnwidth]{../HOS_bathymetry/figures/powerSpec/83000/x_wp15.pdf}}\\
\subfloat[$x=20.0$\,m.]{
\includegraphics[width=.5\columnwidth]{../HOS_bathymetry/figures/powerSpec/83000/x_wp20.pdf}}%
\subfloat[$x=25.0$\,m.]{
\includegraphics[width=.5\columnwidth]{../HOS_bathymetry/figures/powerSpec/83000/x_wp25.pdf}}%
\caption{Power spectra measured at various locations, comparing transitional ramps.
Depth transition from 3.0 to 0.5 meters.
$T\_p=2.0$\,s, $H\_s=0.100$\,m.}
\label{fig:Tp2p0:slopes}%
\end{figure}

\begin{figure}[H]%
\centering
\subfloat[$x=0.0$\,m.]{
\includegraphics[width=.5\columnwidth]{../HOS_bathymetry/figures/powerSpec/830x0_2xflatOnly/x_wp0.pdf}}%
\subfloat[$x=5.0$\,m.]{
\includegraphics[width=.5\columnwidth]{../HOS_bathymetry/figures/powerSpec/830x0_2xflatOnly/x_wp5.pdf}}\\%
\subfloat[$x=10.0$\,m.]{
\includegraphics[width=.5\columnwidth]{../HOS_bathymetry/figures/powerSpec/830x0_2xflatOnly/x_wp10.pdf}}%
\subfloat[$x=15.0$\,m.]{
\includegraphics[width=.5\columnwidth]{../HOS_bathymetry/figures/powerSpec/830x0_2xflatOnly/x_wp15.pdf}}\\%
\subfloat[$x=20.0$\,m.]{
\includegraphics[width=.5\columnwidth]{../HOS_bathymetry/figures/powerSpec/830x0_2xflatOnly/x_wp20.pdf}}%
\subfloat[$x=25.0$\,m.]{
\includegraphics[width=.5\columnwidth]{../HOS_bathymetry/figures/powerSpec/830x0_2xflatOnly/x_wp25.pdf}}%
\caption{Power spectra measured at various locations, flat bed at 0.5 meters depth.
$T\_p=2.0$\,s, $H\_s=0.100$\,m.}
\label{fig:Tp2p0:flat}%
\end{figure}
 

%\end{document}